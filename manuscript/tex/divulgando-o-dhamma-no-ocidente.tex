\chapter{Divulgando o Dhamma no ocidente}

Conforme o número de estrangeiros que desejavam tornarem-se monges
começou a crescer, Luang Pó contemplou a possibilidade de criação de um
monastério exclusivamente para praticantes ocidentais. Parte da razão
era dar a eles um ambiente mais conducente à prática, onde não apenas
podiam se comunicar em sua própria língua como também organizar o
monastério de uma maneira mais compatível com a sua cultura. Outro fator
importante era que ele já tinha em mente o futuro, o momento em que
esses monges teriam que voltar às suas terras natais, e pensava que era
importante que ganhassem experiência em administrar um monastério com
todas suas atividades, como o trabalho de construção, manutenção,
administração das finanças, interação com as comunidades leiga e
monástica, etc. E de fato foi o que aconteceu: a maioria daqueles que
serviram como abades deste primeiro monastério, chamado Wat Pah
Nanachat, mais tarde se tornaram líderes de comunidades em países
ocidentais como Inglaterra, Estados Unidos e Austrália.

O projeto de criação do monastério tornou-se realidade em 1975, quando
Ajahn Sumedho levou um grupo de monges a um pequeno pedaço de floresta
no vilarejo de Bung Wai em busca de bambu seco para utilizar como lenha
para queimar suas tigelas, um processo que protege as tigelas de
ferrugem (quando são feitas de ferro) e que também serve para enegrecer
o metal, tornando-o menos atrativo. O vilarejo ficava a apenas alguns
quilômetros de Wat Nong Pah Pong e muitos de seus moradores conheciam
Ajahn Chah. Após essa visita, os moradores desejaram ter os monges por
perto e foram até Wat Nong Pah Pong fazer um pedido formal a Ajahn Chah
para que os enviasse para viver lá e estabelecer um monastério. Após
ouvir os comentários positivos feitos por Ajahn Sumedho sobre o local,
Luang Pó deu permissão para a criação da nova filial. A primeira ideia
que teve era nomear o local ``Wat Pah América'', mas uma vez que muitos
discípulos vinham de outros países do ocidente, ele o nomeou ``Wat Pah
Nanachat'' (Monastério de Floresta Internacional) e apontou Ajahn
Sumedho como seu primeiro abade.

Em 1976, ao retornar de uma visita a seus pais nos Estados Unidos, Ajahn
Sumedho parou em Londres e se hospedou no English Sangha Trust, uma
organização cujo propósito era criar condições para estabelecer uma
sangha monástica Theravada na Inglaterra. Eles convidaram Ajahn Sumedho
a permanecer lá, mas, apesar de ele ter recusado o convite, prometeu
passar a informação a Ajahn Chah para que ele avaliasse as
possibilidades. Pouco tempo depois, George Sharp, presidente da
organização, foi à Tailândia pessoalmente convidar Luang Pó a visitar a
Inglaterra e contemplar a ideia de criar um monastério no país. Luang Pó
não aceitou imediatamente, mas pediu que George primeiro passasse alguns
dias em Wat Nong Pah Pong, seguindo a rotina diária do monastério tal
qual faziam os demais leigos que lá se hospedavam, ou seja, dormir no
chão, acordar às três da manhã, comer uma única refeição ao dia, tomar
parte nas \emph{pūjas}, na limpeza e demais atividades do monastério.
Somente após ele ter passado alguns dias lá e Luang Pó ter tido a
oportunidade de observar seu comportamento para saber se era munido de
diligência, sinceridade e humildade, foi que decidiu aceitar o convite
para ir com Ajahn Sumedho à Inglaterra, e a viagem foi agendada para
maio de 1977.

Durante aquela viagem, Luang Pó ensinou em vários centros de meditação e
fez contato direto com a sociedade ocidental. Fez também uma visita
curta à França. Ele teve uma boa impressão geral sobre tudo o que viu, a
única exceção sendo o hábito dos ocidentais de pensarem demais e, como
consequência, fazerem uma bagunça total dentro de suas cabeças. Durante
uma de suas visitas a um grupo budista, ele recebeu uma pergunta de uma
distinta senhora inglesa que havia passado muitos anos estudando o
Abhidhamma. Ela pediu que ele por gentileza explicasse certos aspectos
difíceis daquele sistema de psicologia para que ela pudesse continuar
seus estudos, mas Luang Pó simplesmente respondeu: ``Você, senhora, é
como uma pessoa que cria galinhas no quintal, mas recolhe a titica ao
invés dos ovos.''

Ao final da viagem, Luang Pó decidiu aceitar o convite para criar uma
filial de Wat Nong Pah Pong na Inglaterra, mas só informou Ajahn Sumedho
no último dia da viagem, pouco antes de retornar à Tailândia. Naquele
dia ele apenas disse casualmente: ``Sumedho, não precisa voltar para a
Tailândia, você fica aqui.'', e estava encerrada a conversa. Ajahn Chah
voltou à Tailândia sozinho, deixando para trás Ajahn Sumedho, atônito.
Felizmente ele já estava acostumado ao modo de ser de Luang Pó e sabia
que, com ele, tudo era possível.

Daquele dia em diante, Ajahn Sumedho se estabeleceu inicialmente em
Londres e, desse esforço, muitos monastérios surgiram na Inglaterra,
como Cittaviveka, Amaravati, Aruna Ratanagiri e Forest Hermitage. E, uma
vez criado o precedente de um monastério ter sido estabelecido com
sucesso dentro de uma sociedade ocidental, muitos outros países seguiram
o exemplo, como Estados Unidos, Nova Zelândia, Austrália, Itália,
Portugal, Alemanha, Canadá, Suíça e outros.

Luang Pó chegou à Tailândia em julho de 1977, mas essa não seria sua
última visita ao ocidente. Em 1979 ele fez uma segunda viagem à
Inglaterra, desta vez para visitar o recém-estabelecido monastério de
Cittaviveka e seu abade, Ajahn Sumedho. Após passar algum tempo por lá,
viajou aos Estados Unidos. Durante sua estadia, ensinou em vários locais
e a comunidade leiga teve muitas oportunidades de interagir com ele. Eis
um exemplo de perguntas feitas por ocidentais e das respostas dadas por
Ajahn Chah:

Pergunta: ``Eu digo que, mesmo envolvidos com coisas mundanas, mas
fazendo-o com determinação e dedicação total, nossa mente é capaz de
alcançar \emph{samādhi} profundo. Por exemplo, um músico, quando absorto
na música que está tocando, pode alcançar \emph{jhāna}, êxtase e
felicidade, unificação mental. A única diferença é que o foco dele não
era um objeto de meditação, só isso.''

Luang Pó ri e diz: ``Não, não. Ninguém consegue alcançar \emph{jhāna}
tocando música\ldots{} só vocês gringos devem conseguir uma coisa
dessas! Que \emph{jhāna}? Você não sabe do que está falando!''

Pergunta: ``Neste mundo há muita confusão; como podemos ajudar o mundo?
Ainda há esperança para o mundo no futuro?''

Luang Pó: ``Você pergunta sobre o mundo, mas você conhece o mundo? O
mundo são as portas dos sentidos. Interiormente são os olhos, os
ouvidos, o nariz, a língua, o corpo e a mente. Exteriormente são as
formas, os sons, os odores, os sabores, os objetos físicos e os objetos
mentais. Em \emph{pāli}, mundo se escreve ``\emph{loko}'' e significa
``escuridão''. O oposto se chama ``\emph{aloko}'', a luz. A prática do
Dhamma nos faz alcançar a luz que é superior à escuridão do mundo.
Entendeu?''

Pergunta: ``Quando você senta em meditação, como prepara sua mente?''

Luang Pó: ``Cuido dela onde quer que ela esteja.''\footnote{Tradução
  alternativa: ``Cuido dela no lugar onde ela mora.''}

Pergunta: ``Eu aceito que seu ensinamento é a Verdade, mas é difícil
para um leigo que está preso a coisas mundanas praticar de acordo com
ele.''

Luang Pó pega sua bengala, aponta para o peito da pessoa e diz: ``Se a
ponta da minha bengala estivesse em chamas -- pense, o que você diria?
-- se a bengala pudesse falar, ela diria: `Estou sofrendo muito! Está
muito quente!' e ficaria aqui sentada sem fazer nada? Ficaria ali só
resmungando? Ela iria levantar e sair correndo imediatamente! Se virmos
o sofrimento de verdade, não faremos nada a não ser praticar e buscar o
caminho de saída até o alcançarmos.''

Pergunta: ``O que fazer para conseguir abandonar as \emph{kilesas}?''

Luang Pó: ``Não se apresse em abandonar \emph{kilesas}; tenha calma,
olhe para o sofrimento, olhe a causa dele. Olhe bem e então conseguirá
removê-lo por completo. É como quando comemos: mastigamos devagar, com
cuidado e assim a comida digere bem e por completo.''

Pergunta: ``Como fazer para alcançar o estágio de \emph{sotāpanna} o
mais rápido possível?''

Luang Pó: ``Alcançar o estágio de \emph{sotāpanna} é um dos objetivos da
prática, mas requer resiliência, não é algo fácil. Não se apresse muito,
não é só praticando uma ou duas noites que você já vai alcançar
\emph{nibbāna}. Ninguém possui um poder mágico que possa fazê-lo
alcançar a iluminação rápido desse jeito. Leva um pouco de tempo, é
necessário ter dedicação em praticar com firmeza e continuidade para
poder enxergar os resultados.''

Pergunta: ``Você é um \emph{arahant}?''

Luang Pó: ``Sua pergunta é digna de resposta e vou responder da seguinte
maneira: sou como uma árvore na floresta. Pássaros veem à árvore,
sentam-se em seus galhos e comem seus frutos. Para os pássaros, os
frutos podem ser doces ou azedos, ou o que seja, mas a árvore não tem
nada a ver com isso. Os pássaros dizem `doce', ou dizem `azedo' - do
ponto de vista da árvore, isso é apenas o tagarelar dos pássaros.''

Naquela mesma noite também discutiram as virtudes relativas dos
\emph{arahants} e dos \emph{bodhisattas}. Luang Pó encerrou a discussão
dizendo: ``Não seja um \emph{arahant}. Não seja um Buddha. Não seja
absolutamente nada. `Ser' algo cria problemas: portanto, não seja nada.
Você não precisa ser nada, ele não precisa ser nada, eu não preciso ser
nada\ldots{} Se você for um \emph{arahant}, você vai sofrer. Se você for
um \emph{bodhisatta}, você vai sofrer. Se você for qualquer coisa, você
vai sofrer.''

Mas Luang Pó também teve seus momentos de conflito com a cultura
ocidental. Ele às vezes expressava exasperação com o absurdo número de
escolas e sectos, não apenas de diferentes religiões, mas também de
budismo, presentes no ocidente. Outro aspecto que parecia lhe preocupar
era a tendência, nos Estados Unidos, de substituírem professores de
Dhamma monásticos por leigos, o que na opinião dele resultaria em perda
de qualidade e distorção do ensinamento do Buddha.

Apesar de todas as diferenças, ele apresentava uma capacidade
inigualável de lidar com esses choques culturais. Durante um estágio de
sua viagem pelos Estados Unidos, ele teve a oportunidade de descansar
numa área montanhosa onde os pais de um discípulo monástico tinham um
chalé. Numa ocasião anterior à sua estadia lá, Luang Pó havia conversado
com os demais monges sobre como a sociedade ocidental estava perdendo o
pudor em expor o corpo humano e como chegaria o dia em que não haveria
mais nada a fazer a não ser expor o corpo completamente nu. Aconteceu
que, durante aquela estadia nas montanhas, um dos monges estava olhando
uma pilha de revistas velhas que lá se encontrava e entre elas havia
algumas edições da revista Playboy. Ele foi contar a Luang Pó a
respeito:

``Luang Pó, lembra como você estava falando sobre exposição e as coisas
que as pessoas andam fazendo? Bom, de fato aconteceu! Tem umas revistas
aqui\ldots{}''

``É mesmo? Vamos dar uma olhada.''

``Você tem certeza de que quer ver, Luang Pó?''

``Sim!''

Então o monge trouxe duas revistas. Luang Pó pegou seus óculos,
limpou-os e começou a estudar as revistas de capa a capa. Ele olhava com
interesse e curiosidade, mas permaneceu em silêncio durante todo o
período. Tendo encerrado a segunda revista, ele as jogou num canto com
aparente repulsa e disse: ``Vamos!'', indicando que queria ir caminhar
ao ar livre. Luang Pó e o monge caminharam em silêncio, e ele pareceu
sombrio e pensativo durante todo o trajeto, até voltarem para perto do
chalé. Eles então se sentaram num banco à margem do rio e continuaram em
silêncio; era como se algo pesasse em sua mente. Bem nesse momento, um
grupo de leigos chegou, vindo de Seattle e Luang Pó perguntou:

``Quem são esses?''

``Oh, essas são as pessoas que conhecemos no aeroporto.'' No dia
anterior, quando desembarcaram no aeroporto de Seattle, um grupo de
pessoas da cidade notou a presença dos monges e veio conversar. Quando
ficaram sabendo que eles se hospedariam nas redondezas por alguns dias,
pediram permissão para visitar e oferecer uma refeição à sangha.

Luang Pó exclamou ``Ah!'' e riu um pouco, mostrando que a chegada
daquelas pessoas lhe causava interesse. Após algum tempo, ele disse:
``Você achou que aquelas revistas não tinham importância, mas elas me
causaram uma forte impressão. Eu já estava pronto a subir no avião e
voltar para Tailândia! Estava tão enojado por aquelas revistas com todas
essas mulheres que não conhecem seu verdadeiro valor, sua verdadeira
herança, o que elas são capazes de fazer com um nascimento humano, que
já estava pronto a voltar, pensando que era um esforço inútil -- qual o
propósito em ficar aqui?''

Isso foi tudo que ele disse, mas o discípulo logo compreendeu que,
quando Luang Pó viu o grupo de pessoas chegando para oferecer a refeição
e ouvir o Dhamma, sua esperança em ensinar retornou, assim como ocorreu
com o Buddha quando ele considerou que haveria alguns ``com pouca poeira
em seus olhos'' que conseguiriam compreender o Dhamma para o qual ele
havia despertado.

Luang Pó e o monge permaneceram sentados por mais alguns minutos, rindo
felizes, e aquele rir aos poucos se transformou em uma grande gargalhada
que deixou os dois monges com lágrimas de alegria nos olhos. Durante a
refeição, Luang Pó estava de volta a seu jeito habitual de ser,
ensinando e conversando de forma descontraída com o grupo de visitantes.
Após a partida do grupo, mais uma vez ele e seu discípulo sentaram-se em
silêncio por um instante e começaram a rir. Luang Pó retornou à
Tailândia em julho de 1979.

