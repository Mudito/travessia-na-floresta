\chapter{Estabelecendo Wat Nong Pah Pong}

A vida de peregrinações que Luang Pó levava acabou em 1954, quando ele
decidiu se estabelecer em Dong Pah Pong. Daquele dia em diante, ele
permaneceu ali até o dia de sua morte, em 1992, com exceção de alguns
períodos curtos de viagens e três diferentes ocasiões em que passou o
\emph{vassa} em monastérios filiados a Wat Nong Pah Pong. Ele nunca mais
retornou à vida de austeridade nas florestas e montanhas de um monge
peregrino. Estando Wat Nong Pah Pong estabelecido, sua vida passou a ser
administrar, organizar e ensinar; liderar monges e leigos na prática do
Dhamma era sua principal ocupação. Mas antes que tudo chegasse a esse
estágio, ele e seus discípulos tiveram que enfrentar muitos desafios.

Em 8 de março de 1954, ao final da tarde, Luang Pó e seus discípulos
chegaram à densa floresta que os aldeões na época chamavam de Dong Pah
Pong. Ela se localizava a cerca de três quilômetros de Ban Kó, o
vilarejo onde Luang Pó havia nascido. Dong Pah Pong era um local
recluso, com espessa folhagem. Ao chegarem, os monges se espalharam pela
floresta e acamparam ao som das cigarras. Naquela noite, enquanto
sentava em meditação, Luang Pó soube que sua vida estava agora tomando
um novo rumo. Aquela floresta era um local importante para ele havia
muito tempo:

``Quando eu era pequeno, ouvia meu pai contar sobre quando Tahn Ajahn
Sao\footnote{Ajahn Sao Kantasīlo: junto com Ajahn Man Bhuridatto, divide
  o título de patriarca da recente tradição da floresta tailandesa.}
veio acampar aqui. Meu pai foi ouvir o ensinamento dele. Eu era pequeno,
mais ainda consigo lembrar. Esse tipo de lembrança estava próxima a meu
coração o tempo todo, eu sempre pensava nisso porque esse era um lugar
abandonado; eu via os enormes pés de manga, todos muito antigos.

Meu pai me disse que veio prestar reverência aos monges
\emph{kammatthāna}, veio vê-los comer, e eles colocavam toda a comida na
tigela: o arroz, a mistura e os doces eram postos juntos dentro da
tigela. Meu pai nunca tinha visto aquilo (e pensou): `Eh! Que monges são
esses?'

Ele me contou isso quando eu era criança; ele os chamava de monges
\emph{kammatthāna}. Eles não ensinavam como os monges comuns. Quem
quisesse ouvir um sermão não conseguia, pois eles simplesmente falavam
normalmente; portanto, a pessoa não ouvia um sermão, ouvia apenas os
monges falando. Eles eram monges \emph{kammatthāna} que vieram praticar
aqui. Quando eu mesmo me tornei um monge \emph{kammatthāna}, essa
lembrança estava sempre no meu coração. Sempre que virava o rosto em
direção à minha terra natal, pensava nesta floresta, sem exceções. Após
ter passado tempo suficiente em peregrinações, voltei para me
estabelecer aqui.

Certa vez, Ajahn Di, de Pibun Mangsahan, e Tahn Chao Khun Chin (Luang Pó
Put Tāniyo)\footnote{Ambos renomados discípulos de Luang Pu Man e Luang
  Pu Sao.} foram convidados a morar aqui. Eles também queriam ficar, mas
disseram que não poderiam fazê-lo. Ajahn Di disse que esse lugar não era
deles, Than Chao Khun Chin disse várias vezes: `Não posso ficar aqui, o
lugar não é meu. Não demora muito e o dono do local chegará.'''

Quando amanheceu, o grupo examinou a área. A folhagem era tão densa que
quase não encontravam um espaço vazio para colocar seus pertences. Os
aldeões que vieram lhes dar boas vindas prepararam um abrigo temporário
perto de um grande pé de manga (ao sul de onde hoje fica o salão de
\emph{uposatha}). Somente após terem examinado as condições e decidido
que era um local apropriado para estabelecer um monastério, começaram a
construir estruturas permanentes. Com a ajuda dos habitantes de Ban Kó e
Ban Glang, construíram quatro cabanas com telhado de palha, piso de
bambu e paredes de folhas de palmeira. Em seguida, cavaram um poço e
usaram a terra removida como piso para um pequeno salão para reuniões, e
esse salão continuou sendo utilizado pela sangha por muitos anos. Luang
Pó falou sobre como era Wat Nong Pah Pong naqueles primeiros dias:

``As coisas eram muito difíceis antigamente em Wat Nong Pah Pong. Aqui
costumava ser uma grande floresta, era habitação de elefantes e tigres.
Havia um lago onde os diversos animais da floresta bebiam água. Quando
vim para cá não havia nada, a não ser a floresta. Nem fale em ruas ou
estradas, era difícil chegar aqui! As casas das pessoas ficavam longe,
pois elas não tinham coragem de morar perto. Acreditavam que um espírito
muito feroz morava aqui. A lenda dizia que essa pessoa era um caçador de
elefantes, que ele e seus empregados caçavam elefantes para vender. Com
o passar do tempo ele passava cada vez mais tempo por aqui e, no final,
se estabeleceu na floresta de vez, cuidando deste local. Por isso ainda
havia um pouco de floresta quando vim morar aqui; se não fosse assim, a
floresta já teria sido destruída há muito tempo. Moradores de Ban Pueng
e Ban Bok tentaram tomar pedaços de terra para lavrar, mas sempre
acontecia alguma coisa (para impedi-los). Aqueles que vinham cortar
lenha nesta floresta costumavam morrer repentinamente. Havia muita
mandioca e jacatupé crescendo aqui, mas ninguém tinha coragem de botar a
mão neles. Só depois que vim morar aqui é que as demais pessoas vieram
lavrar as terras próximas.''

Após Luang Pó e seu grupo terem passado dez dias lá, numa noite de lua
cheia, por volta das sete da noite, cerca de dez aldeões vieram ouvir o
Dhamma. Luang Pó alertou a todos para que permanecessem pacíficos, não
se assustassem se algo estranho acontecesse e pediu que não fizessem
barulho. Alguns minutos após Luang Pó começar o sermão, uma esfera de
luz, similar a um cometa, apareceu na direção nordeste e flutuou além da
vista em direção ao sudeste. A luz era brilhante o suficiente para
deixar a floresta iluminada como se fosse dia. Todos acharam que era um
sinal muito auspicioso para o novo monastério, mas Luang Pó permanecia
impassível, apenas continuou discursando sobre o Dhamma como se nada
tivesse acontecido. Os aldeões continuaram sentados em silêncio e,
apesar de estarem cheios de assombro e curiosidade, ninguém se atreveu a
dizer nada.

Após o ocorrido, Luang Pó se recusava a fazer qualquer comentário sobre
aquele evento, e essa sempre foi sua atitude quando ensinando os leigos:
mesmo coisas fantásticas são normais, não há razão para muita excitação.
Na manhã seguinte, Luang Pó e os aldeões foram marcar a área do novo
monastério e utilizaram os locais onde a luz surgiu e desapareceu como
pontos de referência. Ele estimou cerca de 90 acres e deu ordens para
que uma trilha fosse capinada marcando o perímetro do monastério.

Mas a aparição de luzes continuou ocorrendo. Uma discípula leiga que
estava entre aqueles que presenciaram aquela primeira aparição, contou
detalhes de uma segunda ocasião:

``Antigamente as estradas não eram tão convenientes como hoje em dia.
Nós morávamos em Ban Kó e queríamos ouvir um ensinamento naquela noite:
para isso tínhamos que usar um caminho estreito que cruzava um bosque.
Alguns trechos eram muito fechados por mato alto e nos perdemos quando
chegamos perto de Nong Ngu Luam.

Paramos para deliberar e tentar encontrar o caminho para chegar onde os
monges moravam e bem nesse instante vimos uma luz brilhando no alto de
um pé de manga. Então começamos a andar, atravessando o mato alto e os
cipós que se enroscavam nas árvores e obstruíam o caminho. Andávamos
tendo aquela luz como objetivo, achando que alguém havia acendido uma
lanterna de querosene. Aquilo nos fez esquecer nosso cansaço, pois
pensávamos que Luang Pó estava ajudando a nós que estávamos perdidos.
Conseguimos chegar ao local de prática, mas assim que chegamos não vimos
luz em lugar algum. Todos acharam aquilo muito estranho.''

A principal razão que fez Luang Pó aceitar o convite para se estabelecer
ali foi o desejo de retribuir sua dívida de gratidão com sua mãe. Por
causa disso, não muito tempo após ter criado o monastério, ele recebeu
sua mãe como a primeira monja de Wat Nong Pah Pong; logo em seguida,
três outras amigas dela se juntaram à sangha e o número total de
monásticos morando lá durante o ano de 1954 era nove: quatro monges, um
noviço e quatro monjas. Durante os primeiros dez anos o número de monges
manteve-se entre quinze e vinte; já o número de monjas crescia todo ano
e, por volta de 1964, elas eram pouco mais de vinte. O nome ``Wat Nong
Pah Pong'' foi escolhido por Luang Pó utilizando parte do nome já
conhecido pelos moradores da região, cujo significado diz respeito à
característica do local: há uma floresta (\emph{pah}) e uma área alagada
(\emph{nong}) repleta de pés de sorgo (\emph{pong}); mas um nome mais
curto que muitos utilizam é ``Wat Pah Pong''.

Os primeiros anos de Wat Nong Pah Pong foram anos de dificuldades e
luta. Por ser localizado numa área pouco desenvolvida, as pessoas que
moravam nas redondezas eram pobres e mal tinham o suficiente para manter
suas próprias vidas. A maioria não estava acostumada com monges
\emph{kammatthāna} e alguns os viam com desconfiança e apreensão. As
pessoas de outras áreas que vinham ao monastério eram muito poucas,
devido à dificuldade de acesso e ao fato de que Luang Pó ainda não era
famoso e venerado como nos anos posteriores de sua vida. Por isso, quase
nada vinha de fora do monastério, e eles em geral tinham que se manter
com aquilo que a natureza provia. Não havia nenhuma das facilidades
modernas como eletricidade ou lanternas a pilha, e mesmo velas e
fósforos eram difíceis de obter. Água para beber e lavar tinha que ser
tirada do poço e carregada até grandes jarros de barro que ficavam
espalhados pelo monastério. A única bebida que de vez em quando tinham
para o período da tarde\footnote{O Buddha proibiu os monges se
  alimentarem no período da tarde, mas os autorizou a consumir sucos de
  frutas, substâncias medicinais e tônicos.} era chá de
\emph{borapet}.\footnote{Um cipó medicinal muito amargo.} A vida dos
monges naqueles dias era difícil e dolorosa; o uso de tudo era
racionado. Luang Pó contou a respeito:

``No começo, quando viemos para cá, não tínhamos chinelos, não tínhamos
fósforos, só tínhamos pedras para fazer faísca. Colocávamos palha dentro
de um bambu com uma ponta aberta e uma casca de limão fechando a outra;
batíamos as pedras para tirar faíscas e acender a palha dentro do bambu
-- era o que usávamos naquela época, tal como usam isqueiros hoje em
dia.

À noite, quando descíamos de nossas cabanas, estava completamente
escuro. Levantávamos as mãos em \emph{añjali} acima de nossas cabeças (e
dizíamos) `\emph{Sādhu!} Que o poder do Buddha, do Dhamma e da Sangha
(me proteja). \emph{Sabbe sattā sukhitā hontu. Sabbe sattā averā hontu.
Sabbe sattā abyāpajjhā hontu}.\footnote{``Que todos os seres sejam
  felizes. Que todos os seres sejam livres de inimizade. Que todos os
  seres sejam livres de opressão.'' (pāli)} Que todos os seres
sejam felizes e não venham para perto de mim!' À noite era escuro desse
jeito, não se via nada e não tínhamos lanternas; não demorava muito e
pisávamos em algo. Eu praticava meditação andando no escuro e várias
vezes pisei em cobras, mas nunca levei picada. Por aqui tinha muita
cobra cascavel.''

Os mantos que vestiam tinham que ser remendados quando ficavam gastos.
Eles iam remendando até que estivesse realmente velho e arruinado antes
de pedir permissão para costurar um novo, porque era muito difícil obter
tecido. Um dos monges daquela primeira geração de discípulos contou
sobre as dificuldades que enfrentavam:

``Tendo feito um manto, ele tinha que durar vários anos. Eu estava lá e
ajudava nesse serviço. Antigamente costurávamos manualmente. Durante os
primeiros três ou quatro anos eu costurava os mantos com minhas próprias
mãos. Às vezes pedia que os monges viessem ajudar a cortar, costurar e
tingir o tecido com tintura de jaca. Levava meses para terminar um único
manto, e para cada monge tínhamos que costurar dois mantos. Não era tão
fácil como ferver água para fazer chá ou café: a tintura tinha que ser
fervida e evaporada até que a cor estivesse correta. Tínhamos que
procurar pano branco, cortá-lo e costurá-lo. Tínhamos que fazer nossas
próprias capas para nossas tigelas (de esmola). Era muito difícil. Se
fôssemos vários monges, como hoje em dia, não sei como conseguiríamos
fazer tudo isso. Era difícil, mas no final conseguimos. A maioria das
cabanas eram baixas, feitas com telhado de palha e outras folhas.
Cortávamos as folhas, que fixávamos em armações, e construíamos a
cabana. Naquela época, \emph{kammatthāna} era assim\ldots{}''

A comida doada era bastante limitada: em geral, apenas arroz. Muito
raramente lhes eram doadas frutas encontradas nas redondezas, como
bananas. As pessoas do nordeste, especialmente no campo, tinham o
costume de só doar arroz durante \emph{pindapāta}, porque o normal era
mais tarde levarem outros pratos para oferecer no monastério. Porém,
como Wat Nong Pah Pong ficava longe e o acesso era difícil, mudaram de
costume e só ofereciam arroz em \emph{pindapāta}, sem fazer o esforço de
levar outros pratos para oferecer no monastério. Como resultado, os
monges tinham que sobreviver com o que estava disponível; após a
\emph{pindapāta} os noviços eram instruídos a recolher quaisquer folhas
comestíveis que pudessem encontrar no caminho de volta ao monastério.

Toda a comida era dividida com a comunidade. Ao retornar de
\emph{pindapāta}, os monges esvaziavam suas tigelas em bacias e a comida
era então passada de um a um, começando pelos monges mais velhos e
terminando com as monjas mais novas. Todos pegavam pouca comida, fazendo
um esforço para que o alimento disponível fosse suficiente para chegar
até o final da fila. É dito que uma vez três bananas foram oferecidas e
Luang Pó mandou que elas fossem cortadas em rodelas finíssimas para que
fosse possível a todos receber pelo menos uma parte. Luang Pó falou um
pouco sobre o assunto:

``No que diz respeito a comida e alimentação, não pensávamos em
desperdiçar tempo preparando comida desse ou daquele jeito: só
pensávamos que, se havia arroz, já era suficiente. Para aqueles que
tinham fé em praticar, quando chegava a tarde eu mandava ferver água,
mas não tinha nada para misturar: não tinha açúcar, não tinha chocolate,
café ou qualquer outra coisa, então fervíamos \emph{borapet} e
distribuíamos. Terminado, só havia silêncio, ninguém reclamava que
estava amargo; não tinha outra coisa para beber, então bebíamos aquilo.
Uma vez um monge foi a Ayutthaya e trouxe um pacote de café em sua
sacola. Mas só tinha café, não tinha açúcar, então tivemos que ferver
café sem açúcar e distribuir. Silêncio total, ninguém abriu a boca para
reclamar. Tínhamos café, então tomávamos café; o açúcar ficava para mais
tarde, se um dia alguém oferecesse. Qual o problema? Todos tomavam em
silêncio, não havia problema algum.''

Mais tarde, quando Wat Nong Pah Pong ganhou fama e renome, o monastério
passou a ser bem suprido de alimentos, mas Luang Pó ainda gostava de
falar sobre as dificuldades dos primeiros anos como uma forma de alertar
para que os monges e noviços não se deixassem levar por luxo ou
conforto:

``Comer todos os dias, mesmo que seja arroz puro, ainda é melhor do que
não comer. Quando comia arroz puro, me lembrava dos cães: em áreas
pobres, os donos de cachorros só dão uma pequena porção de arroz ao dia
para eles comerem. Não tem mistura alguma, só há arroz puro, e os cães
não morrem, ainda vivem tranquilos. Pelo contrário, ainda são cães
diligentes: assim que ouvem um som estranho, começam a latir, acordam
rápido. Quando o dono os leva para caçar, correm rápido, pois são
magros. Mas cães bem cuidados pelos donos costumam ser preguiçosos.
Mesmo que alguém chegue bem perto, eles ainda não latem, dormem o tempo
todo. Mesmo que alguém venha perto a ponto de pisar na cabeça deles,
ainda não despertam.''

No que diz respeito a medicamentos, a sangha utilizava apenas frutas,
ervas e raízes que podiam ser encontradas na floresta. Apesar de Luang
Pó ser flexível com relação a esse assunto, na maior parte dos casos
eles contavam apenas com o ``remédio do Dhamma'', ou seja, \emph{khanti
pāramī} e \emph{samādhi}:

``Com relação a medicamentos e doenças, um monge uma vez ficou doente
com apendicite. A barriga doía, mas ele não admitia ir ao hospital;
aliás, naquela época ninguém ia ao hospital. Pode-se dizer que tinham
muita resiliência. Mesmo eu fiquei doente durante três anos e nunca fui
ao hospital, nem uma só vez. Lutava bem aqui. E então, o que fazer?
Fervíamos \emph{borapet} com sal e bebíamos, comíamos
\emph{samó}.\footnote{Uma fruta medicinal tailandesa.} E era
muito bom: o corpo sofria e doía, mas se ainda não havia chegado a hora
da morte, não morria. Era só um pouco inconveniente, porque antigamente
não tínhamos remédios, se um monge ou noviço ficasse doente (eu dizia):
`Bom, aguente! Monges \emph{kammatthāna,} vocês não precisam ter medo,
se vocês morrerem eu me encarrego de cremá-los. Se eu morrer, que vocês
me cremem -- não precisa guardar o corpo,\footnote{Na Tailândia é comum
  guardar o cadáver por vários dias, às vezes meses ou anos, se o corpo
  for de um mestre muito reverenciado.} pois é sofrimento.' Falávamos
desse jeito, alertávamos uns aos outros dessa forma. Ninguém era frouxo
ou vacilante, eram corajosos e capazes de verdade. Não me preocupava se
haveria algum monge vacilando ou fraquejando quando defrontado com
qualquer uma dessas coisas.''

Luang Pó tomava doenças como excelentes oportunidades para aprofundar a
prática do Dhamma, para contemplar a natureza impermanente do corpo
humano e seu curto tempo de vida, para ver que este corpo é sofrimento e
não é um ``eu'' -- é apenas um aspecto da natureza. E, nesse respeito,
ele mesmo sempre foi um bom exemplo -- um discípulo contou sobre uma
ocasião em que Ajahn Chah ficou doente:

``Certa vez ele ficou muito doente, e os discípulos se revezavam para
cuidar dele. Ficavam dois monges do lado de fora (da cabana), e após um
período eles trocavam. Quando entravam, perguntavam como ele estava e
tocavam o corpo dele. Ele não dava permissão para massageá-lo, pois
dizia não querer se apegar a massagens; ele não deixava. Saíamos e ele
ficava deitado no quarto com sua febre. Ficávamos por perto, do lado de
fora, e sentávamos em meditação de costas um para o outro, não
conversávamos. Ele permanecia deitado e enfermo daquele jeito. Quando
dava o horário, entrávamos e tocávamos o corpo dele -- às vezes a febre
tinha aumentado e tentávamos ajudar. Fazíamos desse jeito.

Certo dia a febre aumentou durante a tarde, quando veio um visitante do
exército procurá-lo. O visitante estava esperando debaixo de uma árvore
onde havia um local para sentar. Luang Pó vestiu o manto e foi
recebê-lo. Mesmo com dor de cabeça e vomitando, ele ainda assim suprimiu
a dor e veio receber o visitante\ldots{}

\ldots{}Ele tinha muita resiliência. Não havia remédio, não havia
médico, ninguém havia se oferecido para pagar despesas médicas, por isso
ele dizia: `Se não morrer, transforme em algo bom. Se não ficar bom,
deixe morrer.'

Mas quando seus discípulos ficavam doentes, Luang Pó fazia um esforço
extra para lhes ajudar. Se houvesse alguém doente, ele perguntava os
sintomas, olhava o penico -- se estivesse cheio, ele lavava. Terminado
isso, ele pegava uma vassoura e varria a área ao redor da cabana do
discípulo; ele também discursava sobre o Dhamma para que o enfermo
ganhasse ânimo. Pode-se dizer que fazia tudo de acordo com o que está
especificado no Vinaya.\footnote{Na regra monástica há uma seção que
  especifica os deveres de um professor para com seus discípulos quando
  estes estão doentes.} No que diz respeito a remédios, ele não tinha
muito a oferecer, mas Luang Pó oferecia encorajamento. Alguns monges
ficavam doentes por anos e Luang Pó tentava procurar remédios de ervas,
que era um conhecimento que ele havia adquirido de seus professores. Ele
procurava plantas medicinais para cuidar dos demais e pela manhã também
separava comida para dar ao enfermo.''

Durante aquele período houve uma epidemia de malária e os monges, monjas
e noviços frequentemente ficavam doentes. Por isso, às monjas foi
delegada uma tarefa extra: à tarde elas tinham que cortar \emph{borapet}
para utilizar como remédio contra febre. A epidemia durou três anos, mas
felizmente ninguém em Wat Nong Pah Pong faleceu dela. Uma monja
discípula de Ajahn Chah contou sobre a época em que ele também contraiu
malária:

``Luang Pó havia adoecido com malária mesmo antes dos demais residentes
do monastério. Os sintomas eram muito fortes. Ele pediu que o
carregassem para debaixo de uma árvore, que colocassem uma esteira (para
ele deitar), porque queria ficar ao ar livre, sob uma árvore. Os leigos
vinham visitar, mas naquela época ainda não havia remédio (para
malária). Com relação a hospital, Luang Pó não deixava os monges irem,
pois dizia que era uma confusão desnecessária. Ele não deixava ir, não
deixava que sequer mencionassem a palavra `hospital'. Nos tratávamos de
acordo com o que tínhamos. Às vezes traziam remédios de ervas do campo
para triturar e ferver, e esse foi o tratamento dele.

Luang Pó ficou mais e mais doente, a ponto de a pele dele começar a
ficar verde escura. Quando é assim, significa que a pessoa está em seus
últimos momentos. Naquele dia ele estava muito doente. Estava deitado e
de repente se ergueu, tentou levantar, tentou sentar e deitou novamente.
Ele se sentou e se deitou como se não estivesse ciente do que fazia.
Naquele momento todos os monges, monjas e noviços permaneceram em
completo silêncio, todos os olhos fixos nele. Ele se levantou mais uma
vez, e desta vez sentou balançando o corpo para frente e para trás, como
se não conseguisse se manter erguido. Então olhou ao redor e viu o jarro
com remédio de ervas que estava ao lado. Ele ergueu o jarro e despejou
por completo sobre sua cabeça, ficando completamente encharcado. O monge
que estava perto não foi rápido o suficiente para impedi-lo. Em seguida
largou o jarro e de repente se sentou em meditação, completamente
imóvel\ldots{} em silêncio\ldots{} Quando se sentou em meditação, todos
os residentes do monastério e leigos ficaram espantados e surpresos.

Na manhã seguinte ele ainda estava doente, mas após alguns dias os
sintomas melhoraram. Não sei que remédio ele tomou para poder se curar.
Mas, assim que se curou, todos os demais ficaram doentes, tanto os
monges como as monjas. Quase todos ficaram muito doentes. Luang Pó
mandava colher \emph{borapet} e pilar em forma de pasta. Usávamos uma
seção de \emph{borapet} do tamanho do braço do enfermo, cortávamos em
pequenas rodelas e pilávamos. Misturávamos com um copo d'água e
filtrávamos com um pano para obter o líquido espesso, e então nos
forçávamos a beber tudo aquilo.''

Durante aquela epidemia de malária, um grupo de médicos veio ao
monastério e disse a Luang Pó: ``Esta floresta é fechada demais; você
deveria cortar os galhos das árvores para que fique mais arejada,
facilitando a passagem do vento.'' Luang Pó respondeu: ``Que as pessoas
morram. Se pelo menos preservarmos a floresta, já é suficiente.'' Os
médicos tentaram explicar novamente e convencê-lo com toda forma de
argumento, mas ele permanecia impassível. Sua resposta final foi: ``Se
um dos monges, monjas ou mesmo eu vier a falecer, não há importância. É
melhor preservar a floresta.'' Os médicos não sabiam o que dizer diante
daquilo e não tiveram outra opção a não ser ir embora.
