\chapter{Treinamento nos primeiros anos}

Durante aqueles primeiros anos, Luang Pó ainda estava trabalhando em seu
desenvolvimento pessoal. Além disso, havia apenas um pequeno número de
monges e noviços vivendo no monastério, o que fazia com que o modo de
prática fosse bastante duro, muitas vezes sendo levado ao limite entre
vida e morte. Luang Pó frequentemente elogiava seus discípulos daquela
época dizendo que eram todos realmente resolutos na prática do Dhamma, e
cada um levava sua prática ao máximo de suas habilidades:

``Antigamente os monges não se juntavam em grupos para ficar batendo
papo; se lhes era dito que não conversassem, davam ouvidos. Ficava cada
um em sua cabana; ao cair da tarde, cada um voltava para sua cabana.
Mesmo os cães não conseguiam viver aqui, pois quando chegava a manhã
eles não tinham onde dormir. Cada cabana era distante uma da outra; ao
cair da tarde, terminados seus deveres, cada monge entrava em sua
cabana, um ia lá longe na floresta, o outro ia para o outro lado e os
cães ficavam com medo, não tinham onde morar. Quando seguiam um monge,
este entrava em sua cabana; seguiam outro monge, mas ele fazia o mesmo;
seguiam outro monge, mas era o mesmo. No final, não tinham para onde ir.
Os cães não conseguiam morar aqui, mas as pessoas conseguiam -- que
coisa triste!\footnote{Está se referindo ao fato de que o modo de vida
  era tão austero que as pessoas se encontravam numa situação que nem
  cães conseguiam suportar.}

\ldots{} No começo, Ajahn Jan e Ajahn Tieng ainda não sabiam muito, não
entendiam nada; mas (ganharam sabedoria) graças à resiliência deles,
graças a serem obedientes e darem ouvidos aos ensinamentos do mestre.
Quando algo lhes era dito ou ensinado, não discutiam com o mestre:
aceitavam, contemplavam (sobre o que lhes havia sido dito). Cada um
permaneceu por seis ou sete anos. Ambos estavam sempre aqui, nunca foram
para outro lugar, nunca foram para o norte ou para o sul (do país).
Nunca iam para cá e para lá, caminhando sem rumo -- que é uma perda de
tempo e não vale o esforço. Eles buscavam o Dhamma no mestre, praticavam
tudo que fosse o caminho para progredir e por isso tinham muita força,
tanto corporal como mental. Seguiam o mestre que os levavam a praticar e
ter coragem, a não ter medo de nada. Após seis ou sete anos, mandei que
fossem beneficiar os leigos em suas terras natais, onde permaneceram por
muitos anos.''

Um discípulo falou sobre a atmosfera dentro do monastério naqueles dias:
``Durante o período em que morei com Luang Pó, ele enfatizava muito
Vinaya e \emph{sīla}. Naquela época ele ainda tinha bastante energia e
ainda estava desenvolvendo a si mesmo. Ele era energético em tudo que
fazia. Para ele, tudo fazia parte de \emph{sīla}, ou seja, ele
enfatizava muito \emph{sīla}.
.
Nossa rotina diária era acordar às três da madrugada e nos reunir para a
\emph{pūja} matutina. Na maioria das vezes, ele chegava lá antes dos
discípulos. Assim que chegava, ele se sentava em meditação e então
recitava a \emph{pūja}. O grupo fazia tudo junto e em harmonia. Às vezes
o nome de alguém era chamado -- se algum monge ou noviço estava sendo
relapso em seguir a rotina diária, se esquivando frequentemente, ele
chamava pelo nome e dava quinze minutos para que aquela pessoa chegasse
ao salão. Após quinze minutos, ele chamava o nome e então começava a
\emph{pūja} e a meditação. Ele explicava e ensinava o método inicial
para vencer os \emph{nīvaranas}. Todos tinham que ter resolução em
praticar, e ele estava sempre vigiando -- se os \emph{nīvaranas}
invadissem a mente de alguém, ele chamava o discípulo em voz alta para
que voltasse a si. Quando chegava a hora de encerrar a sessão, ele
repetia vários alertas sobre o corpo, o comportamento e a mente -- ele
dizia para estarmos sempre cientes.

Quando ele falava havia silêncio total; todos tinham medo, não tinham
coragem de se mexer, não diziam nada. Se tivessem que falar, falavam o
mais baixo possível. Em seguida nos preparávamos para ir em
\emph{pindapāta}. Aqueles que seguiriam as rotas mais distantes saíam
antes; quem fosse mais perto esperava um pouco. Luang Pó ia numa rota
próxima. Ele não ia a Ban Kó. Ele dizia que o vírus ainda estava
presente, o vírus do deleite no local e nas pessoas: se fosse até lá,
veria sua casa e lugares que frequentou, veria suas irmãs, seus
sobrinhos e sua mente iria se apegar ali. Se não fosse necessário, não
ia a Ban Kó; ao invés disso, ia em \emph{pindapāta} em Ban Glang.

Enquanto os monges saíam para as rotas mais distantes, ele pegava uma
vassoura e varria o chão antes de sair em \emph{pindapāta}. Ao terminar,
também recolhia o lixo. Às vezes não varria: ao invés disso, pegava um
cesto e recolhia as folhas à beira dos caminhos ao redor do salão
pequeno. Os discípulos ajudavam e, quando dava o horário, saíam em
\emph{pindapāta}. É assim que fazíamos.

Quando dava três da tarde, fechávamos a porta e as janelas de nossas
cabanas; se tivéssemos posto o manto no varal, recolhíamos para que,
caso algo acontecesse, como vento e chuva, não tivéssemos que voltar
correndo para guardar. Tínhamos que prestar reverência o tempo todo.
Antes de sair da cabana prestávamos reverência; quem não o fizesse, se
chegasse ao salão e lembrasse (que havia se esquecido de prestar
reverência), tinha que voltar à cabana e prestar reverência. Também
trazíamos uma vassoura e o bule d'água, trazíamos um sarongue debaixo do
braço ou dobrado sobre a cabeça. Quando chegávamos ao salão, primeiro
colocávamos nossas coisas de lado e íamos prestar reverência perante a
estátua do Buddha no salão grande. Deixávamos o bule sobre o assento
para monges; já o sarongue levávamos junto para colocar sobre a cabeça e
proteger do sol caso necessário, mas não enrolávamos na cabeça: apenas
colocávamos sobre a cabeça de forma organizada; então varríamos o
monastério. Luang Pó também varria, os monges e noviços varriam em fila
e não se ouvia o som de nenhuma conversa. Se não fosse realmente
necessário, ninguém conversava -- só se ouvia o som de gente
trabalhando.

Naquela época não havia muitas bebidas doces: tínhamos só chá amargo de
\emph{borapet} e outras plantas medicinais como \emph{samó} e
\emph{amla}, mas não havia muito açúcar. Havia bastante \emph{samó} que
comíamos com pimenta ou fermentado, mas não comíamos todo dia; quando
comíamos, o fazíamos às duas e meia da tarde. Quando havia, Luang Pó
mandava tocar o sino. Se um monge, ao comer, jogasse fora a semente de
forma descuidada, Luang Pó dizia que ele havia quebrado \emph{sīla}. Não
podia jogar de forma descuidada: tinha que reunir todas e, ao terminar
de comer, recolhê-las e jogar fora num local apropriado. Quando
terminávamos, ele dava um pequeno ensinamento e às três da tarde íamos
(varrer o monastério).

Quando terminávamos de varrer, cuidávamos da água de beber e de lavar.
Antigamente tínhamos que tirar água do poço e carregar. Era Luang Pó
quem segurava a corda e puxava o balde para cima; eu e os demais monges
recebíamos o balde. Naquela época não havia muitos monges: começamos com
seis, oito, dez monges e o número foi aumentando continuamente. Tendo
tirado a água, carregávamos; tanto os velhos como os jovens ajudavam; só
se ouvia o som de gente trabalhando, não havia som de conversa. Ele
vigiava o tempo todo e, por isso, os monges tinham cuidado, respeito e
fé de verdade.

Terminado isso, íamos ao salão varrer e preparar o assento dos monges.
Começávamos (a varrer) do lado de fora e íamos avançando para que a
poeira não entrasse no salão. Tendo varrido, esfregávamos um pano úmido
no chão e preparávamos o assento. Em seguida íamos tomar banho e depois
cada um ia para sua cabana praticar meditação andando. Praticávamos até
as seis da tarde e, quando ouvíamos o som do sino, guardávamos nossas
coisas, fechávamos a cabana e nos apressávamos em nos reunir no salão.

Ao ir ao salão no fim da tarde, Luang Pó sempre alertava para que
chegássemos antes, não obrigando o mestre a ter que esperar. Ao fim da
\emph{pūja} ele subia no \emph{thammat}\footnote{Um acento elevado
  utilizado quando se está ensinando o Dhamma.} e ensinava o
Pubbasikkhā, traduzindo sua linguagem floreada\footnote{O Pubbasikkhā é
  um livro escrito usando uma linguagem muito técnica e floreada, o que
  torna a compreensão difícil para uma pessoa comum.} para entendermos.
Depois disso os monges se dispersavam e voltavam para suas cabanas por
volta das onze da noite. Às vezes continuávamos até a meia-noite ou uma
hora.

Ele possuía um método de prática: ensinava a praticar de pé, sentado,
andando, e de várias outras maneiras. No que diz respeito a utensílios,
ele era muito frugal. Se um monge jogasse no mato algo que ainda podia
ser utilizado, ele via e guardava -- cuias, cestos de lixo, todo tipo de
coisa. Ele dizia que não saber ser frugal é quebrar \emph{sīla}. Quando
eu era recém-chegado, não entendia porque \emph{sīla} se quebrava rápido
e com tanta facilidade. É assim porque se trata de uma ação anormal do
corpo e da fala; falar mal, fazer malfeito, guardar mal as coisas é não
ter \emph{sati}; estar defeituoso em \emph{sati} é quebrar \emph{sīla}.
Ele se esforçava em guardar as coisas para que seus discípulos tivessem
o que utilizar, mas algumas pessoas não entendiam isso e achavam que ele
estava reclamando, que ele era ranzinza, avarento. Ele ensinava os
discípulos a se esforçar a praticar dessa maneira.

Todos os \emph{uposathas} praticávamos \emph{nesajjika},\footnote{Prática
  ascética que consiste em renunciar à postura deitada, ou seja, varavam
  a noite praticando meditação sentados e andando (pāli).} tanto
os monges como os leigos. Nesses dias era muito difícil sair do salão.
Conforme íamos sentando, o corpo começava a doer, mas se fôssemos sair,
tínhamos que ter o máximo de cuidado. Ninguém tinha coragem de sair
primeiro, porque não queríamos que os demais nos olhassem com desdém;
tínhamos esse tipo de pensamento no coração. Quando alguém saía
primeiro, os demais então saíam em seguida; não saíamos para ir sentar
ou deitar, só saíamos para mudar de postura e caminhar um pouco. Às
vezes alguém saía, mas não aguentava e acabava deitando; mas se ouvisse
o som de alguém ou o som de passos se aproximando, tinha que se apressar
em levantar, porque tinha medo que o vissem. Luang Pó alertava para sair
para mudar de postura: não era para sair e bater papo. Tínhamos que ter
cuidado; se realmente tivéssemos que sair, esperávamos alguém sair antes
e então íamos em seguida. Luang Pó permanecia sentado, perfeitamente
imóvel.

Às vezes ele mandava os leigos ensinarem o Dhamma. Naquela época havia
leigos que sabiam discursar, chamavam-se Pó Yai Di, de Ban Dong Kén, e
Pó Yai Mun. Eles conseguiam falar o dia ou a noite inteira. Luang Pó
dizia `OK, monges e noviços, ouçam um leigo discursar sobre a vida
laica, sobre como é difícil ganhar a vida.' Ele dava oportunidade para
que os monges ouvissem e ia apontando as desvantagens, dizendo: `A vida
laica é sofrida e difícil desse jeito.' Era uma meio de também ganharmos
sabedoria graças aos leigos.''

Luang Pó se esforçava em fazer com que seus discípulos vissem o
sofrimento como ele realmente é: ``Se quando o sofrimento surgir
quisermos que ele desapareça, não vamos vê-lo, não vamos conhecê-lo, não
vamos conseguir removê-lo. A verdade é que o sofrimento é o que nos faz
ficar mais inteligentes; faz surgir sabedoria, nos faz contemplar
\emph{dukkha}. Quem sofre deveria contemplar o sofrimento, e não fugir
porque não quer sofrer. Sofrimento é uma ferramenta que nos mostra que
bem aqui estamos errados, bem aqui não está bom. Somos assim: o
sofrimento vai nos levar a buscar um mestre e, no final, alcançar paz.''
Uma vez que olhar o sofrimento é tão importante, Luang Pó tomava medidas
para garantir que seus discípulos sofressem diariamente; ele sempre
enfatizava a importância de desenvolver \emph{khanti pāramī} como ponto
inicial da prática do Dhamma e dizia que era a melhor arma para destruir
\emph{kilesas}.

Durante o verão, após a refeição, ele mandava todos os monges sentarem
em meditação no antigo salão de \emph{uposatha}, dava ordens para que
todas as portas e janelas fossem fechadas e que os monges vestissem o
manto completo. Não demorava muito para que eles se sentissem como que
assando dentro de um forno e que seus mantos ficassem encharcados de
suor. Se alguém se atrevesse a reclamar, a resposta que ouvia era:
``Você conseguiu viver na barriga da sua mãe por nove meses. Isso não é
nada comparado com aquilo!'' Já durante o inverno, ele dava ordens para
que os monges deixassem suas cabanas e vivessem ao ar livre na floresta
do monastério, para treiná-los a não se apegarem a confortos, a serem
capazes de enfrentar o frio e a desolação de viver a céu aberto, a
enfrentar o medo de animais como cobras, escorpiões e lacraias, além da
irritação causada por formigas, cupins e outros animais.

``Eu investiguei muito em minha prática. Coloquei minha vida em risco,
porque acreditei quando o Buddha disse que o caminho e o fruto de
\emph{nibbāna} de fato existem, tal como ele ensinou. Mas todas essas
coisas nascem da prática, nascem de ir contra suas vontades, de ter
coragem: coragem em treinar, coragem em pensar, coragem em mudar, em
fazer. E como é esse `fazer'? O Buddha disse para ir contra sua própria
mente. Nossa mente pensa nessa direção, mas vamos naquela; a mente pensa
em ir naquela direção, mas ele dizia para irmos nessa outra. Por que ele
dizia para ir contra nosso coração? Porque o coração está completamente
envolto em \emph{kilesas}: ainda não foi treinado, corrigido. Ainda não
é \emph{sīla}, ainda não é Dhamma. Se o coração ainda não estiver
brilhante e branco, como podemos acreditar nele?

\ldots{} Se só ficarmos dizendo que não conseguimos cortar, se só
falarmos nisso, todos nós neste monastério não seremos mais que um bando
de cafajestes. Justamente por não conseguirmos cortar é que temos que
nos esforçar! Se não consegue cortar, então rale, rale as
\emph{kilesas}, lixe. Se não consegue remover cortando, remova cavando.
É difícil, não vai de acordo com nossos desejos, tem que ter cuidado por
todos os lados.

\ldots{} Se não opusermos nossos sentimentos agora, hoje, ser uma pessoa
ordinária, um delinquente, continuará algo fixo dentro de nossa
personalidade. O Buddha tornou-se um monge por enxergar dessa maneira.
Ele foi se transformando, se opondo continuamente. Quem não contemplar
com minúcia, não vai conseguir enxergar seus próprios sentimentos.''

Outro discípulo falou sobre os ensinamentos de Luang Pó naquela época:
``O que Luang Pó mais enfatizava era o Vinaya e o modo de prática. Não
deveríamos negligenciar o modo de prática, não deveríamos parar de
praticar. Se não fosse realmente necessário, não deveríamos deixar de
realizar \emph{pūja} diariamente -- se o fizéssemos, que não fosse por
mais de quinze dias ou um mês; é assim que fazíamos. Seus ensinamentos
sobre a prática do Dhamma enfatizavam a meditação: tínhamos que praticar
meditação sentados e andando com frequência -- pela manhã, ao meio-dia,
à tarde e à noite. Se tivéssemos algum dever, o cumpríamos e então
voltávamos à prática de meditação. Os monges nunca batiam papo ou faziam
brincadeiras: todos permaneciam em silêncio. Com os leigos era a mesma
coisa: eles praticavam como os monges, praticavam com seriedade,
regularmente. Escutavam os ensinamentos e estudavam o Dhamma.

Luang Pó não tinha o hábito de bater papo ou brincar; ele nunca falava
sobre assuntos mundanos, sobre assuntos laicos, não fazia piadas, não
falava sobre prazeres dos sentidos -- imagens ou sons agradáveis. Para
evitar encorajar maus hábitos nos monges, ele não falava sobre essas
coisas. Ele proibia que falassem ou conversassem sobre esses assuntos,
ele proibia juntar-se em grupos para evitar que houvesse desarmonia na
comunidade. Ele focava em desenvolver o tríplice treinamento
(\emph{sīla, samādhi, paññā}) e o estudo do Dhamma-Vinaya. Naqueles dias
ele frequentemente ensinava usando o Pubbasikkhā, pela manhã e à noite.
Após a \emph{pūja} ouvíamos sobre Vinaya, estudávamos Vinaya. Fazíamos
tudo de acordo com as regras do Vinaya, mesmo a \emph{kathna},\footnote{Uma
  cerimônia realizada ao fim do vassa (pāli).} o costurar dos
mantos e a fabricação de escovas de dente.\footnote{Ao contrário do que
  muitos pensam, o Buddha não ensinava só doutrinas e assuntos abstratos
  e filosóficos a seus discípulos. Ele, como líder da ordem monástica,
  também era um administrador e, como tal, via a necessidade de ensinar
  aos monges muitas coisas que hoje em dia talvez sejam entendidas como
  óbvias, mas que não o eram há 2.600 anos: como cuidar da saúde, como
  cuidar de bens pessoais, como manter harmonia na comunidade, etc.
  Sobre higiene bucal, ele ensinava a confeccionar ``escovas de dentes''
  utilizando madeiras cuja seiva possuía propriedades bactericidas. O
  processo consistia em pilar uma ponta da madeira para quebrar as
  fibras até que estas ficassem macias e pudessem ser utilizadas como
  escova para limpar os dentes. A escova é de uso único e, portanto, os
  monges tinham que fabricá-las em quantidade suficiente para vários
  dias. Mesmo hoje ainda é comum encontrar pessoas utilizando esse
  produto natural na Índia, Tailândia e outros países da Ásia.}
Usávamos tudo como um treinamento; todas essas atividades ajudam os
monges a se livrarem da sonolência e da preguiça. Ele ensinava assim:
`Se estiver com sono, não vá simplesmente se deitar -- você deve
primeiro encontrar um meio de consertar aquilo: pelo menos vá procurar
madeira para fazer escovas de dente, faça pelo menos nove ou dez todos
os dias.'

Às vezes íamos praticando até as onze horas ou meia-noite antes que ele
anunciasse o fim da reunião e nos mandasse ir descansar por duas horas.
Como poderíamos descansar? Apenas duas horas e já tínhamos que estar de
volta, sentados novamente. Ninguém conseguia bater o sino porque ele
sempre chegava lá antes dos demais e ele mesmo batia o sino para nos
acordar; nunca tínhamos a chance de bater o sino, porque só dormíamos.
Tentávamos ao máximo, mas não conseguíamos fazer como ele. Ele conseguia
aguentar muita dor.

Mas os resultados surgiam. Pessoas importantes da região, tanto leigos
como monásticos, vinham nos pedir poções mágicas, porque pensavam que
ele nos ensinava essas coisas.\footnote{Na Tailândia há a crença de que
  aqueles que desenvolvem samādhi podem adquirir poderes paranormais (o
  que de fato está correto, mas que não são os reais objetivos dos
  ensinamentos do Buddha).} Quando seus discípulos iam estabelecer um
novo monastério, ele prosperava mas na verdade isso não tinha relação
alguma com poderes mágicos. Esse resultado surgia graças à resiliência e
\emph{mettā}\footnote{Bem-querer, amizade, boa vontade (pāli).} de Luang
Pó, graças ao seu Dhamma-Vinaya, e é por isso que ele dizia que aqueles
que protegem o Dhamma são protegidos pelo Dhamma e não caem no caminho
da maldade. Com certeza será assim, se conseguirmos praticar como Luang
Pó fez.

Tudo que ele ensinava, tínhamos que fazer. Se era para caminhar,
tínhamos que caminhar; se era para sentar, tínhamos que sentar e não
levantar. Não era brincadeira, ele não brincava: nós tínhamos que fazer.
Se visse que não estávamos fazendo, ele nos chamava e dava uma bronca.
Ele dava bronca nos monges e noviços. Ele convocava uma reunião, mas não
dizia muito, não podíamos nos mexer. Se ficássemos sentados à toa aqui e
ali, se entrássemos e saíssemos da reunião, ele perguntava
imediatamente: `Por que saiu?' Se fôssemos urinar e levássemos mais de
uma hora para voltar, estávamos em apuros. Uma vez um monge foi urinar e
levou mais de uma hora. Quando voltou, Luang Pó disse: `Da próxima vez
me avise antes de ir; quero ir junto para ver (o que está acontecendo).'

Por causa disso, era impossível não ter medo dele. Você não podia não
querer fazer algo porque ele não brincava, ele não largava. Não era
permitido não ter um espaço para praticar meditação andando na área de
sua cabana, não era permitido não ajudar a varrer o monastério; se você
não viesse um ou dois dias, estava em apuros: `Qual o problema? Está
doente? Se está doente, por que não avisou? Por que não avisou? Se não
me avisar, tem que avisar a um outro monge. Como você pôde simplesmente
fazer o que lhe deu vontade? Por acaso você mora aqui sozinho?' Ele não
hesitava em dar bronca, e você não podia se mexer, não podia brincar. Se
ele encerrasse uma reunião e dissesse a todos para irem praticar, mas
visse alguns andando à toa para cima e para baixo, chamava: `Ei, ei! Por
que está batendo perna à toa desse jeito?'

Seus olhos eram ágeis, ele não deixava passar nada, nem mesmo a coisa
mais ínfima. Se dissesse para caminhar, você não podia não caminhar; se
dissesse para ir, você não podia permanecer, ou levaria mais uma bronca.
Quando era hora de ir, todos tinham que sair imediatamente; se você não
o fizesse, não conseguia escapar. Se você tivesse um motivo, então tinha
que explicar qual era, e por causa disso há muitas histórias engraçadas
daquela época. Ele não fazia isso apenas se impondo aos outros, mas
praticando junto e dando o exemplo. Essa era a forma dele de
administrar; o problema era que nós ainda não éramos Dhamma e Vinaya.''

As diversas regras do Vinaya são muito refinadas, e às vezes Luang Pó
usava alguns truques para testar o quão hábeis seus discípulos estavam
em seguir essas regras. Por exemplo, um discípulo deve estar sempre no
mesmo nível ou num nível mais baixo que o mestre; portanto, quando este
está sentado em uma cadeira, o discípulo deve se sentar mais baixo que
ele, até mesmo no chão, se necessário. Outro exemplo é quando o
professor não estiver calçado, o discípulo então também deve ficar
descalço; se o professor não está cobrindo sua cabeça, o discípulo
também deve descobrir a sua, abaixando o guarda-chuva ou removendo o
chapéu, se for o caso. Às vezes Luang Pó caminhava sob o sol com um pano
protegendo sua cabeça; quando removia o pano, ele olhava para trás -- se
algum dos discípulos estivesse distraído e demorasse para descobrir sua
cabeça, levava uma bronca.

Sobre Vinaya, ele uma vez disse: ``Não obedecer ao Vinaya é o mesmo que
não respeitar o Buddha, porque foi ele quem estabeleceu o Vinaya. Por
isso, não obedecer, não respeitar o Vinaya, é o mesmo que não respeitar
o Buddha. Se respeitamos o Buddha, temos que respeitar o Vinaya e
obedecer estritamente ao Vinaya. Se não fizermos isso, não sei para que
nos ordenar e qual benefício essa ordenação nos traria.

\ldots{} Onde fica o Vinaya? Fica em nós. Se o respeitamos, ele está
presente; se não o respeitamos, ou se o fazemos com desleixo, não somos
nada além de bandidos que destroem o \emph{Buddha Sāsanā}.''

Certa vez um monge veio pedir para passar um tempo no monastério
estudando o modo de prática de Wat Nong Pah Pong. Luang Pó não o
proibiu, mas avisou: ``Se vai ficar comigo, você não pode possuir
dinheiro ou acumular bens. Aqui nós respeitamos o Vinaya.''

O monge respondeu: ``E se eu ficar com meu dinheiro, mas não me apegar a
ele, tudo bem?''

Luang Pó disse: ``Claro, mas só se você conseguir comer uma colher de
sal e dizer que não está salgado -- só que então vai ter que comer o
saco inteiro!''

Um de seus discípulos resumiu os aspectos do Dhamma que eram o coração
do ensinamento de Ajahn Chah durante aquele período:

\textbf{Sempre ter \emph{sati}:} ``Primeiro, ele dizia para estabelecer
\emph{sati} firme e contínua em sua mente. Não se deixe levar, não seja
distraído, não deixe faltar \emph{sati}. Ele dizia que esse Dhamma não possui
abaixo ou acima, curto ou longo; ele dizia que era como um coco -- nossa prática
deveria ser redonda como um coco. Ele nos dizia para firmar \emph{sati}, para
ter fé na Joia Tríplice -- o Buddha, o Dhamma e a Sangha. Ele tinha muito
respeito por isso e nos ensinava a ter o mesmo respeito e a ter \emph{sati}. Ao
prestar reverência, tínhamos que ter \emph{sati}; se não prestássemos
reverência, ainda tínhamos que ter \emph{sati}. Ele dizia que, se estávamos
distraídos, nos deixássemos levar em pensamentos, é porque \emph{sati} não
estava boa, ainda estava defeituosa de uma maneira ou de outra. Se nossa mente
não for pacífica, se for confusa, ela não terá frescor, estará anormal -- deve
ter algo de errado conosco ou nos está faltando \emph{sati}.

Em geral ele enfatizava \emph{sati} em seus ensinamentos: sempre ter
\emph{sati} de forma contínua ao praticar, fazer com que ela fosse
contínua, sem quebras. Ensinava a ter \emph{sati} quando de pé, andando,
sentado, comendo, etc. Não ter \emph{sati}, é como estar morto -- é
assim que ele falava. Se reclamássemos que não tínhamos tempo para
praticar, ele perguntava: `Quando comemos, estamos respirando? Quando
dormimos, paramos de respirar?' -- ele comparava dessa forma. `Como pode
estar tão cansado a ponto de não conseguir meditar? A respiração está
bem ali, praticamos e ela está bem ali; se temos \emph{sati}, ela está
bem ali.' Ele ensinava esse tanto.

Em geral ele ensinava a ter \emph{sati}. `Se não tiver \emph{sati}, você
não vai desenvolver prática alguma, \emph{samādhi} algum ou alcançar
purificação. \emph{Sati} é o mais importante, frescor vem de
\emph{sati}, paz vem de \emph{sati}, bem-estar interno e externo estão
em \emph{sati}; Dhamma-Vinaya, todos os aspectos do caminho de prática
estão em \emph{sati}. Se não tivermos \emph{sati}, o que vamos saber?'
Ele ensinava a verdade dessa forma.''

\textbf{Abrir mão da vaidade:} ```Se você é um monge, não seja arrogante, não
pense que você é melhor que os demais graças à sua nacionalidade, família ou
algo do tipo. Isso não é \emph{sīla} ou Dhamma. Se pensarmos daquela maneira,
nossa prática não dará frutos. Temos que realmente abrir mão de tudo isso, ter a
intenção firme de abandonar a vaidade.' Ele ensinava a não ser arrogante: `Somos
monges vivendo juntos e não deveríamos ter problemas uns com os outros. Somos
todos iguais, todos similares. Não fiquem pensando sobre quem está certo ou quem
está errado. É normal que apontemos algo uns aos outros; se estamos errados, é
só isso, jogue fora, não fique apegado a certo ou errado. Estar apegado não está
de acordo com o Dhamma-Vinaya. Se estivéssemos certos, não agiríamos dessa
forma. Assim é que se é tanto Dhamma como Vinaya. Se nos apegamos e saímos por
aí ensinando aos outros `o Vinaya tem que ser praticado assim, o Dhamma é
assim', não será nem Dhamma, nem Vinaya. Você talvez esteja apegado a algo que
não é Dhamma nem Vinaya -- tudo errado! Se está de acordo com o Vinaya, também
está de acordo com o Dhamma. Praticar Vinaya é praticar o Dhamma'.

Ele falava como se fossem dois caminhos conectados; ele não separava
Dhamma e Vinaya. Como quando ele ensinava sobre a prática de
\emph{sīla-dhamma}: se não estiver de acordo com \emph{sīla}, como
poderia estar de acordo com o Dhamma? Ele os ensinava conectados dessa
forma, e é por isso que ele dizia: `Meu Dhamma não é curto ou longo: tem
a forma de um círculo, como um pomelo ou um coco.'''

Monges da segunda geração de discípulos de Luang Pó Chah também tiveram
a oportunidade de experimentar o gosto apimentado desse estilo intenso
de prática. Um deles falou a respeito: ``Se alguém não fizesse como ele
estipulou, levava bronca imediatamente. Ele subia no \emph{thammat} e
não parava de falar até que chegasse a hora da próxima atividade do
monastério. Não voltávamos às nossas cabanas: só voltávamos após a
atividade ter se encerrado, apenas para continuar praticando lá e
esperar o horário da próxima atividade. Por exemplo, se um dia ele
ouvisse um som alto porque alguém estava tirando água do poço fora do
horário estipulado, assim que chegasse a noite ela dava a bronca, ia
falando continuamente até tarde da noite, e às vezes ainda não descia do
\emph{thammat} -- continuava até as três da madrugada. Assim que descia
do \emph{thammat,} começávamos a \emph{pūja} matinal. Se ele ouvisse o
som de alguém lascando madeira de jaca para ferver e tingir um manto
fora do horário especificado, era a mesma coisa.

Os monges tinham que chamar a atenção e avisar uns aos outros sempre que
viam alguém fazendo algo fora do horário. Tinham medo que fosse motivo
para mais confusão. Mesmo que só uma pessoa agisse errado, Luang Pó não
deixava os demais voltarem para suas cabanas, fazia todos sentarem e
ouvirem juntos. Era um ensinamento: aquela pessoa estava desperdiçando o
tempo daqueles que queriam praticar o Dhamma, desperdiçava o tempo de
prática ou de realizar outras atividades.

Então os monges estavam sempre com medo, medo de ter que ouvir um longo
sermão. Era comum chamarem a atenção dos demais membros do grupo. Se
Luang Pó soubesse que alguém fez algo errado, ele não esperava até o dia
seguinte: logo chamava a pessoa e dava bronca ou então fazia todos
ouvirem o sermão. Ele era sempre rigoroso e controlador. Não importa
quem fosse fazer o que ou o quão necessário fosse, tinha que ir
informá-lo antes de fazer. Não era possível fazer nada sem autorização
dele, porque se ele descobrisse, a pessoa certamente ouviria um longo
sermão.

Ninguém podia faltar à \emph{pūja} ou chegar atrasado nem cinco ou dez
minutos. Se estivesse andando perto do salão, tinha que pisar o mais
leve possível, sem fazer barulho. Ninguém podia andar fazendo barulho ou
levaria uma bronca por estar perturbando a meditação daqueles que já
estavam sentados, e isso seria demérito e mau \emph{kamma} para si
mesmo, além de ofender as boas maneiras dos praticantes do Dhamma. Além
disso, é falta de \emph{sati} e \emph{sampajañña},\footnote{Compreensão
  clara do momento presente (pāli).} falta de compostura, é algo muito
prejudicial a um praticante -- era o que ele dizia.

Luang Pó era muito rigoroso com organização, disciplina, limpeza,
cuidado e atenção. Ele dizia que só ser cuidadoso, mas não ter
atenção,\footnote{Esse é um jogo de palavras com os sinônimos ``\thai{สำรวม}'' e
  ``\thai{ระวัง}''.} não era suficiente. Ele dava o exemplo sobre ter cuidado,
mas não ter atenção: uma vez ele saiu em \emph{pindapāta} com outro
monge que era muito cuidadoso: ele não olhava para direita ou esquerda e
não levantava o rosto, andava olhando para baixo. Quando ia em
\emph{pindapāta} no vilarejo, esse monge levava ao pé da letra a prática
de restringir a visão. Na regra monástica é recomendado olhar somente um
metro à frente de si, e ele andava olhando para baixo, olhando somente
um metro à frente o tempo todo e não levantava o rosto para olhar para
onde estava andando ou o que havia mais à frente.

Ele foi andando e acabou entrando no chiqueiro de um aldeão. Quando
entrou, deu de cara com o esterco de porco e ficou curioso, levantou os
olhos e então descobriu que estava entrando no chiqueiro porque o
caminho que estava seguindo levava ao chiqueiro. Luang Pó não disse nada
porque achou que ele sabia o que fazia, já que era ele que andava à
frente. Desse ponto em diante, Luang Pó tinha que avisar o tempo todo --
se era para virar à esquerda, ele dizia `esquerda, esquerda'; se era
para virar à direita, ele dizia `direita, direita'. Como um professor
ensinando os alunos a desfilarem numa parada -- ele tem que dizer
`esquerda, direita!' Por ter cuidado mas não prestar atenção, esse monge
criava dificuldades desnecessárias aos outros. Luang Pó contava isso
como exemplo de como ter cuidado mas faltar em atenção trazia
problemas.''

Outro discípulo contou sua experiência: ``Se você fosse urinar, levava
bronca. Se ainda não havia chegado a hora, você tinha que aguentar --
era o que ele dizia. `Os outros conseguem, por que você não? Você é uma
pessoa completa como eles, por que não consegue aguentar?' Era desse
jeito. Eu suava por inteiro até meu manto ficar encharcado. Eu suava
tanto que no final não precisava mais ir urinar. Eu tinha medo dele,
medo de verdade. Ele ensinava assim: `Se você comer muito, vai defecar
muito; se beber muito, vai urinar muito.' Eu acho que ouvi tanto isso
que ficou gravado na minha cabeça. Por isso, quando comia ou fazia
qualquer coisa, eu pegava só um pouco. Antigamente eu tomava muito
cuidado.''

Um discípulo contou sobre como foi quando chegou a Wat Nong Pah Pong
pela primeira vez: ``Quando cheguei, em 1960, Luang Pó não me aceitou.
Argumentou que eu não havia entrado em contato antes, portanto não havia
cabanas vazias. Os leigos que vieram comigo insistiram por um longo
tempo e se comprometeram a pagar o custo de construção de uma cabana; só
então ele deu permissão. Antes de vir morar com ele, eu era um monge da
cidade, agia como um monge da cidade. Quando chegou a hora de construir
minha cabana, Luang Pó ficava de pé supervisionando, mas eu era
arrogante e pensei que os monges de Wat Nong Pah Pong, além de não serem
hospitaleiros, não sabiam fazer nada. Eu não conhecia o Vinaya, então
fui ajudar a cavar e enterrar os pilares e cortar cipó para fazer
amarras.\footnote{A regra monástica proíbe monges de cavar o solo e
  cortar plantas. O monge achava que os demais não ajudavam na
  construção porque não sabiam trabalhar, mas o verdadeiro motivo é que
  não queriam quebrar as regras de conduta.} Luang Pó não disse nada.

No começo eu pensava que não deveria haver muita diferença entre viver
num monastério de floresta e num monastério da cidade. Preparei um
estoque enorme de alimentos e trouxe comigo, como leite condensado e
achocolatado. Trouxe várias latas, pensava que poderia beber antes de
sair em \emph{pindapāta}.\footnote{A regra monástica proíbe os monges de
  acumularem alimentos. Eles devem comer só o que tiver sido dado como
  esmola naquele mesmo dia.} Trouxe vários mantos comigo.\footnote{Em
  monastérios de floresta, os monges possuem apenas uma muda de mantos,
  composta de um sarongue, um manto simples e um manto duplo para o
  frio.}

Quando faltavam cerca de quatro ou cinco dias para o início do
\emph{vassa}, ele convocou uma reunião da sangha antes da \emph{pūja}.
Ele determinou que um monge fosse o responsável por verificar os
pertences dos monges novos para ver se havia algo proibido\footnote{O
  Buddha não permitia aos monges possuir ou sequer utilizar dinheiro.
  Caso algum objeto tivesse sido comprado pelo próprio monge, ele tinha
  que renunciar à posse daquele objeto.} e garantir sua pureza perante a
sangha. No final, naquele dia não me sobrou nada além de um único
sarongue: todo o resto era objeto de transgressão do Vinaya. Ele
verificava de forma muito detalhada. Alguns objetos, apesar de eu não
ter comprado, tinham sido oferecido por leigos, mas eu havia lavado com
sabão que havia comprado, e por isso não era considerado puro, não
estava de acordo com o Dhamma-Vinaya, e me mandaram devolvê-los ao meu
monastério anterior. Luang Pó disse que na época dele era ainda pior --
o mestre dizia para jogar tudo fora.

Depois disso, o monge foi arrumar um novo manto para que eu vestisse.
Era um manto \emph{kammatthāna} de cor escura, com marcas de remendo num
local onde tinha rasgado. Luang Pó dizia que não tínhamos que nos
preocupar com utensílios, porque, se nossa \emph{sīla} estivesse pura,
as coisas viriam sozinhas. Então ele anunciou à sangha (que meus
pertences já haviam sido trocados) e mandou que eu realizasse a
cerimônia de confessar transgressões. Eu não sabia fazer como os monges
da floresta e tive que repetir as palavras que ele dizia. Após a sangha
ter aceitado minha confissão, ele me disse para ajudar a aconselhar os
demais monges novatos. Eu fiquei muito feliz por ter me tornado um monge
da floresta; às vezes acordava à noite e me sentava admirando minha
tigela nova.''

Outro monge também contou sobre seus sentimentos no dia em que trocou
seus pertences para que pudesse ser aceito como membro da sangha de Wat
Nong Pah Pong: ``Quando eles trocaram meus mantos e pertences, eu tinha
cerca de quinhentos bahts e os joguei fora por sobre os ombros,\footnote{O
  Vinaya especifica que caso um monge possua dinheiro, ele deve
  renunciá-lo jogando-o fora sem olhar onde irá cair.} para que os
anagārikas pegassem. Mas eu ainda estava com pena de jogar fora meu
relógio. Era um relógio da marca Technos que eu usava para olhar a hora
quando ensinava.\footnote{Esse monge possuía um nível elevado de erudição
  sobre escrituras budistas e costumava dar aulas sobre o assunto.}
Quando peguei para jogar fora e abrir mão, pensei: `Se eu abrir mão
desse relógio, onde vou conseguir um novo para usar quando voltar a dar
aulas?' Luang Pó então falou, como se soubesse o que se passava na minha
cabeça: `Hum\ldots{} você tem um relógio? Jogue fora, não é bom. Fique
com o meu.' E ele pegou um relógio da sacola dele e me deu. Eu ainda
tenho aquele relógio comigo e o guardo com todo cuidado.''

Certas cerimônias, como a recitação do \emph{uposatha} ou a ordenação de
novos monges, não devem ter outros participantes presentes a não ser
monges devidamente ordenados. Por essa razão, é comum um monastério
proibir monges desconhecidos de participarem dessas cerimônias para
evitar que elas sejam invalidadas pela presença deles. Apesar da sangha
de Wat Nong Pah Pong ser muito rigorosa na prática do Vinaya, quando
monges de Wat Nong Pah Pong iam para monastérios Dhammayuta, não eram
autorizados a tomar parte em cerimônias oficiais da sangha, pois Wat
Nong Pah Pong é um monastério do secto Mahā Nikaya.

Na ocasião em que um monge Dhammayuta veio passar o \emph{vassa} em Wat
Nong Pah Pong, Luang Pó perguntou à sangha se eles deveriam deixar o
monge visitante tomar parte na recitação do \emph{uposatha}. A maioria
dos monges era da opinião de que eles não deveriam autorizar, uma vez
que os monastérios Dhammayuta também lhes negavam esse direito. Mas
Luang Pó discordou e disse: ``Podemos fazer como vocês dizem, mas ainda
não é Dhamma ou Vinaya, ainda é arrogância, é muito orgulho e apego a si
mesmo, não gera bem-estar. Que tal fazermos do modo do Buddha? Isto é,
não nos importamos com Dhammayuta ou Mahā Nikaya, mas apenas com o
Dhamma-Vinaya. Se ele se comportar de forma correta, não importando se
ele é Dhammayuta ou Mahā Nikaya, deixamos ele participar. Se não se
comportar, se não tiver vergonha em realizar más ações, mesmo que seja
Dhammayuta, não deixamos ele participar; se for Mahā Nikaya, a mesma
coisa. Se fizermos dessa forma estará de acordo com o que foi
estabelecido pelo Buddha.''

Luang Pó aconselhava seus discípulos, caso estivessem em peregrinação e
lhe perguntassem a qual secto pertenciam, a dizer: ``Não tenho interesse
por sectos ou escolas, meu interesse é pelo Dhamma ensinado pelo
Buddha.'' Se perguntassem se eram Mahā Nikaya ou Dhammayuta, eles
deveriam responder: ``Meu \emph{upajjhāya} é Mahā Nikaya, mas eu sou
apenas um monge que se comporta bem e respeita o Vinaya.''

Em 1963, um evento ocorreu que demonstrou como em tempos de doença Luang
Pó cuidava tanto de seus discípulos leigos como dos monásticos. Quando
Pó Puang, um leigo muito próximo ao monastério, já tinha alcançado idade
avançada, fez um pedido a Luang Pó: ``Quando eu morrer, quero oferecer
meu esqueleto ao monastério\footnote{É comum encontrar esqueletos
  humanos preservados em monastérios da floresta na Tailândia para serem
  utilizados na prática da contemplação da morte.} e peço que Tahn Ajahn
tenha a bondade de buscar meu corpo\footnote{A tradição no nordeste da
  Tailândia é de que, quando alguém morre, monges vão até a casa do
  falecido e lideram a procissão que leva o corpo até o local do
  funeral.} e organizar o funeral aqui no monastério.'' Luang Pó aceitou
o pedido. Não muito tempo depois, Pó Puang começou a adoecer e foi
mandado ao hospital várias vezes, até que seus sintomas pioraram muito.
Seus filhos então o trouxeram de volta para casa, para que lá passasse
seus últimos dias, e vez por outra Ajahn Chah ia visitá-lo. A certa
altura os sintomas ficaram preocupantes: ele já não podia mais falar e
não abria os olhos; tudo que conseguia fazer era ficar deitado e gemer.
Sua esposa e filhos cuidavam dele, mas não tinham como ajudá-lo.

No dia seguinte, a condição de Pó Puang piorou ainda mais. Ajahn Chah
estava participando de um almoço num quartel do exército e, após a
refeição, disse ao oficial de comando que queria ir visitar Pó Puang e
pediu que ele lhe arrumasse um carro grande para isso. O comandante
disse que não era necessário um veículo grande para uma visita como
essa, que um carro pequeno seria suficiente, mas Luang Pó repetiu sua
ordem e o carro foi providenciado. Pó Nu (o mesmo que no passado havia
discutido com Ajahn Chah a cerca de um poço d'água) o acompanhava e
achou aquilo suspeito. Ele perguntou se estava realmente indo visitar Pó
Puang ou recolher seu corpo. Luang Pó respondeu:

``Estou indo recolher.''

Pó Nu respondeu: ``O que o senhor vai recolher? Ele ainda não morreu! O
filho dele pediu que você fosse? Pó Puang não disse que era para o
senhor recolher o corpo dele após a morte dele?''

Luang Pó permaneceu quieto por um instante e então falou: ``Morto ou
não, vou recolher o corpo dele hoje.'' e disse ao monge que o
acompanhava: ``Vamos, Pó Puang está esperando.''

Quando chegaram, Ajahn Chah sentou-se à beira da cama de Pó Puang e
olhou para ele por um instante. Pôs sua mão sobre seu rosto com
suavidade e chamou:

``Pó Puang, Pó Puang\ldots{}'' Após um momento, Pó Puang abriu os olhos
e olhou para ele. Ajahn Chah perguntou:

``Pó Puang, você se lembra de mim?'' O enfermo fez um sinal positivo com
a cabeça, olhou para o rosto de Ajahn Chah e lágrimas rolaram de seus
olhos, mas não parou de gemer.

``Pó Puang, você é um praticante do Dhamma, você vem lutando há muito
tempo. Chegou a hora de eles virem recolher -- deixem que eles levem,
pois pertence a eles, para que se apegar? Devolva o que pertence a eles.
Mantenha sua voz dentro de si, por que está deixando que ela venha para
fora?'' Quando Luang Pó disse isso, o gemido cessou imediatamente. Luang
Pó continuou:

``Esse corpo é de fato impermanente. Esse corpo está feio porque já está
velho, foi utilizado por muito tempo. Vá procurar um corpo novo naquele
lugar que você uma vez viu (durante sua prática de meditação).''

Ajahn Chah continuou acariciando o rosto de Pó Puang, então virou-se e
perguntou ao oficial do exército: ``Que horas são?'' Eram 12h55. Luang
Pó então disse: ``Em cinco minutos Pó Puang partirá.'' Enquanto Ajahn
Chah acariciava seu rosto, os olhos de Pó Puang começaram a fechar pouco
a pouco e ele faleceu exatamente às 13h00. O corpo foi levado a Wat Nong
Pah Pong e o funeral foi realizado de acordo com os desejos do falecido.
