\chapter{Luang Pu Kinari}

Quando Luang Pó chegou a Wat Pah Metta Vivek, seu manto estava em
péssima condição e necessitando renovação urgente:

``Vinha usando o mesmo sarongue por dois anos até ele começar a rasgar
por completo. Toda vez que me sentava, tinha que levantar o manto um
pouco porque ele estava tão velho que grudava no corpo, não deslizava
como um pano novo. Naquela época estava em Ban Pah Tao e estava suado,
varrendo o chão. Me distraí e sentei de supetão, sem erguer o sarongue,
e ele rasgou bem no meio do traseiro. Tive que procurar um pano branco
para vestir, mas não consegui achar um pano para remendar meu sarongue.
Tive que lavar um pano de chão e usá-lo para costurar por dentro.

Então sentei e pensei: `Eh! Por que o Buddha faz as pessoas sofrerem
desse jeito? Não posso pedir a ninguém,\footnote{A regra monástica
  estipula limites sobre o que um monge pode pedir aos leigos e quando
  pode fazê-lo.} não posso fazer nada.' Fiquei desanimando porque meu
manto e sarongue estavam rasgados. Me sentei em meditação e mudei de
atitude: `Que seja! Não importa o que aconteça, não vou desistir, se não
tiver o que vestir, então não visto, fico pelado de vez.' Meu coração
era resoluto a esse ponto, pensava em levar tudo até os extremos e ver o
que acontecia. Daí em diante comecei a usar o manto todo remendado,
aonde quer que fosse me vestia daquele jeito.

Voltei para prestar reverência a Ajahn Kinari e fiquei por um tempo, mas
morar com ele não era como morar com qualquer outra pessoa, porque o
comportamento dele não era igual ao de ninguém. Ele apenas olhava e eu
não pedia nada. Se o manto rasgava de novo, eu procurava mais um retalho
e remendava. Ele não me convidou a permanecer no monastério e eu também
não pedi permissão, mas fiquei lá com ele. Ia praticando, um não falava
com o outro, era como se competíssemos para ver quem ia falar primeiro.

Quando chegou perto do começo do \emph{vassa}, ele deve ter dito aos
familiares dele que havia um monge com um manto em trapos e pedido que
costurassem um manto para oferecer. Digo isso porque trouxeram pano para
oferecer, era um pano de confecção caseira, muito grosso, tingido com
jaca,\footnote{Os monges da floresta na Tailândia costumam tingir seus
  mantos com pigmento obtido da madeira do pé de jaca.} e usaram um fio
grosso, como aqueles utilizados em cerimônias de funeral, para costurar.
Costuraram tudo manualmente, e as monjas ajudaram. Fiquei muito feliz.
Usei por quatro ou cinco anos sem que rasgasse. Na primeira vez que
usei, fiquei parecendo um jarro de barro, porque o pano ainda estava
rígido (por causa do tingimento), ainda não havia amaciado, e quando eu
andava, fazia um barulho: `flop-flop'. Pior ainda se vestisse o
\emph{sanghātī}\footnote{Manto de duas camadas utilizado para proteger
  os monges do frio (pāli).} -- ficava ainda mais gordo. Mas nunca
reclamei, levou uns dois anos para que o pano amaciasse. Usava sempre
aquele manto e sempre pensava com gratidão a Luang Pu porque me deu sem
que eu precisasse pedir, foi muita sorte minha. Desde que recebi aquele
manto me senti feliz e tranquilo.

Olhei para minha conduta desde o passado, no presente e futuro, e isso
me fez pensar que qualquer \emph{kamma} que realizamos e não está
errado, não traz sofrimento, traz apenas bem-estar, é um bom
\emph{kamma}. Eu pensava dessa forma, via nesses termos e gostei do que
vi, então acelerei minha prática com força total. Com relação àquele
manto, se o vestisse, subisse uma montanha e me encontrasse com um
tigre, eu digo que o tigre não ia ter coragem de atacar -- assim que ele
chegasse perto e visse o manto, ia fugir tremendo!''

Mas provavelmente o maior problema para Luang Pó durante esse período
ainda era desejo sexual. Quando chegou em Wat Ban Tóng, em Nakhon Panom,
foi posto frente a frente com seu velho inimigo e quase foi derrotado.
Ele teve que fugir no meio da noite por causa de uma bela e abastada
viúva que lhe oferecia comida todo dia. Não demorou muito para Luang Pó
perceber que ela tinha segundas intenções com relação a ele, e ele mesmo
sentia sua mente fraquejar. \emph{Kilesas} e Dhamma lutavam ferozmente
dentro da sua mente, até que uma noite, quando estava pensando tanto
naquela mulher a ponto de sentir que já não podia mais confiar em si
mesmo, decidiu recolher seus pertences e partir imediatamente. Ele
apressou-se para acordar Anagārika Kéu:

``Não podemos ir amanhã?'', Kéu perguntou.

``Não! Temos que ir agora mesmo!''

Muitos anos mais tarde, quando morava em Wat Nong Pah Pong e já havia
conquistado essa \emph{kilesa}, Luang Pó foi ensinar Dhamma em Wat Ban
Tóng e falou sobre os velhos tempos. Contou em tom de brincadeira aos
moradores sobre as dificuldades que encontrou em sua prática: ``Oh! Eu
tinha muitos desejos, mas minha maior tentação foi essa mulher aqui.''
(apontando para aquela viúva, que então estava sentada junto aos demais,
ouvindo os ensinamentos).

E foi realmente difícil como ele disse. Quando retornou ao monastério de
Luang Pu Kinari para passar o \emph{vassa}, o desejo sexual voltou a
desafiá-lo com ainda mais intensidade do que em ocasiões anteriores.
Durante um período em que estava praticando de forma intensiva, essa
kilesa começou a atormentá-lo com violência. Quer estivesse andando,
sentado ou em qualquer postura, Luang Pó via imagens do órgão sexual
feminino por toda parte. A sensação de desejo era tão forte que ele
quase não conseguia praticar meditação; tinha que lutar com aquela
sensação e com aquelas imagens com muito esforço. Luang Pó contou que o
desejo sexual durante aquele período o oprimia tanto quanto a sensação
de medo o oprimiu durante aquela noite que passou no cemitério. Ele não
podia praticar meditação andando porque seu órgão sexual roçava contra o
sarongue e ficava estimulado. Ele então pediu para um leigo abrir uma
área no meio da floresta para que ele pudesse praticar meditação andando
longe das vistas das demais pessoas e assim poder levantar seu sarongue
e amarrá-lo à cintura. Ele teve que aguentar aquele embate com desejo
sexual por dez dias antes que a sensação e as visões começassem a
retroceder.

Luang Pó contava essa história a seus discípulos porque sabia que ela
seria uma boa fonte de Dhamma, especialmente para monges mais jovens,
pois mostra que mesmo um grande mestre como ele teve que enfrentar esse
desafio, além de provar que, por mais forte que seja o desejo, uma
pessoa realmente resoluta será capaz de \mbox{vencê-lo}. Em 1968, quando um
discípulo estava coletando informações e histórias para escrever uma
biografia de Ajahn Chah, Luang Pó contou-lhe sobre esse evento, mas o
discípulo ficou em dúvida se seria apropriado incluir essa história no
livro ou não, por achar que causaria embaraço a Luang Pó, mas ele
insistiu: ``Você tem que incluir. Se não incluir essa história no livro,
é melhor não escrever!''

Durante aquele ano (seu oitavo como monge), Luang Pó começou a procurar
um método para controlar seu desejo sexual, mas ainda não havia
encontrado nenhum que desse resultados satisfatórios:

``Eu não olhei para o rosto de mulheres durante todo o \emph{vassa}. Eu
ainda conversava com elas, mas não olhava no rosto. Mas tinha que
segurar meus olhos -- eu queria olhar desesperadamente. Ao final do
\emph{vassa} pensei: `Vou dar uma olhada, pois não vi o rosto de uma
mulher durante três meses. A essas alturas as \emph{kilesas} já devem
estar enfraquecidas'. Assim que pensei e olhei -- Oh! Ela estava usando
um vestido vermelho extravagante, e só uma olhadela foi suficiente: meus
braços e pernas ficaram dormentes. Pensei comigo mesmo: `Puxa, quando
essas \emph{kilesas} vão desaparecer?' e fiquei desanimado. Mas não é
assim, esse tipo de coisa tem que surgir da prática para que possamos
conhecer as coisas como realmente são. Mas no começo é preciso se
afastar do sexo oposto.''

Eventualmente ele venceu essa guerra e, graças a ter passado por tantas
experiências difíceis, mais tarde pôde ajudar seus discípulos com muitos
conselhos sobre como lidar com esse inimigo. Por exemplo, numa ocasião
em que um jovem monge queria abandonar a vida monástica para se casar,
ouvindo o conselho que Luang Pó dá ao rapaz podemos ter uma ideia melhor
de quais métodos ele utilizou para derrotar o desejo sexual:

``Transforme seu amor atual em um amor universal, um amor por todos os
seres, um amor como o da mãe e do pai pelo filho. Eu mesmo moro com
vocês e os amo como se fossem meus filhos e netos. Lave seu amor,
removendo dele o desejo, da mesma forma que fazemos com o inhame
silvestre: temos que deixar de molho para remover o veneno antes de
comer. Com o amor é a mesma coisa: temos que contemplar, temos que olhar
até enxergarmos o sofrimento e aos poucos ir removendo o micróbio da
obsessão para que sobre somente amor puro, como o amor de um mestre pelo
discípulo.

Eu já passei por esse amor de juventude, entendo como você se sente. Se
não remover o desejo de dentro desse amor, mesmo quando for velho o
veneno ainda estará ativo. Pegue a mácula do desejo sexual e reflita
sobre ela até conseguir abandoná-la. Se não conseguir resolver isso
usando sabedoria, se não conseguir diminuir a força do desejo, então
fuja e vá recuperar seu equilíbrio e depois volte novamente. Isso se
chama `cair e saber levantar-se', `deixar escapar, mas não deixar cair'.
Não se deve cair e ficar deitado como um defunto, afundado em desejo
sexual. Ter família é estar preso, não se pode ir a lugar algum, não se
pode sair em peregrinação, não se pode viver junto a um mestre. O filho
chora, a esposa reclama, o sogro critica, a sogra xinga. As louças e
talheres vão aprisioná-lo por completo, pense bem!

Tem um ditado que diz: `Cinco coisas impossíveis de impedir: a chuva que
está por cair, a diarreia que está por explodir, uma pessoa que está
para morrer, uma mulher que vai dar a luz e um monge que quer largar o
manto.' Eu acredito nos quatro primeiros, mas não acredito no último
item. Eu digo que é possível impedir um monge que queira largar o manto.
Eu mesmo já pensei em largar o manto, mas consegui mudar de ideia. Após
largar o manto teria que assumir responsabilidades, construir uma
reputação, um ego; não é simples e livre como a vida monástica. As
pessoas dizem: `Viver sozinho é tranquilo, mas não é divertido. Viver
com alguém é divertido, mas não é tranquilo, traz sofrimento.' Eu digo
que é divertido, mas só um pouco, como o gosto de uma comida que é
agradável só na ponta da língua, mas assim que engolimos o sabor
desaparece. Já a prática do Dhamma, se praticarmos até a mente se
pacificar e enxergar o Dhamma, é muito tranquila. Às vezes há tanta
felicidade que não sente necessidade de comer, chega-se a esse ponto,
não é agradável só na ponta da língua.

\ldots{}Ajahn Tongrat, um dos meus professores de \emph{kammatthāna},
uma vez pensou em largar o manto. Não dava ouvidos a ninguém, queria
apenas largar o manto e nada mais. Ele então pediu um machado emprestado
a um leigo e começou a rachar lenha. Rachou por três dias e três noites
até que suas mãos começaram a sangrar e ele ficou exausto. Então
perguntou a si mesmo: `E então, quem é o chefe aqui?' -- ele perguntou
às suas próprias \emph{kilesas}. Amor é assim mesmo, mesmo os nossos
mestres já passaram por isso. Tahn Ajahn Ngai, por exemplo: ele se
apaixonou por uma moça que oferecia \emph{pindapāta} para ele. Os amigos
dele o pegaram e trancaram dentro de um templo e o mandaram praticar
meditação. Ele ficou trancado sem comer por cinco dias e então sua mente
mudou: ele viu \emph{asubha},\footnote{O aspecto repugnante do corpo
  humano (pāli).} e isso fez sua mente se unificar e enxergar o
Dhamma; e assim ele conseguiu escapar.

Desejo sexual é o ponto fraco de todos nós; temos que usar a
contemplação de \emph{asubha} para resolver esse problema. Teste suas
forças para poder conhecê-las, mas não deixe as \emph{kilesas} atacarem
seu ponto fraco até você ser nocauteado. A prática tem que ser flexível:
se as \emph{kilesas} atacarem por cima, temos que nos esquivar para
baixo; se não conseguir devolver os golpes, pule para fora do ringue e
fuja -- não fique trocando socos se não tem força suficiente, ou vai
acabar sendo nocauteado.''

No entanto, o período que Luang Pó passou em Wat Pah Nong Hi não rendeu
apenas histórias ruins, muito pelo contrário. Certo dia, após terminar
sua prática de meditação, Luang Pó resolveu descansar em sua cabana e
deitou-se sem perder \emph{sati}. Assim que começou a cair no sono,
surgiu uma visão de Luang Pu Man se aproximando segurando nas mãos uma
joia que, ao chegar, entregou a Ajahn Chah, dizendo: ``Chah, eu lhe dou
esta joia. Veja como brilha!''. Ajahn Chah sentou-se e levantou as mãos
para receber a joia. Quando voltou a seu estado mental desperto, estava
sentado e ainda tinha as mãos estendidas como na visão que acabara de
ter. Esse evento lhe trouxe muita energia, e desse ponto em diante
\emph{sati} estava sempre presente. Ele se sentia muito feliz e estava
resoluto em investigar o Dhamma e ganhar maior conhecimento sobre o
caminho de \emph{kammatthāna}.

Durante aquele período, Luang Pó estudou com Luang Pu Kinari e tornou-se
próximo a ele. Pôde observar seu modo de prática e desenvolveu muito
respeito e admiração por ele. Luang Pu Kinari era discípulo de Luang Pu
Sao, mas muito poucos o conheciam porque ele era uma pessoa distante,
que gostava de viver em reclusão. Era um mestre que tinha um modo de
prática muito simples e admirável: vivia em reclusão, era firme em sua
prática, era frugal e todos seus pertences eram velhos e gastos, a
maioria tendo sido fabricada por ele mesmo. Uma característica marcante
de Luang Pu Kinari era ser muito diligente em todo tipo de trabalho que
um monge pode realizar. Ele nunca ficava parado, exceto quando praticava
meditação, e manteve esse comportamento mesmo em idade avançada.

Ajahn Chah conta que, durante aquele \emph{vassa} que passou com Luang
Pu Kinari, ele mesmo estava praticando com força total. Praticava
meditação andando até sob a chuva, praticava tanto que o espaço usado
para caminhar virou uma vala. Mas quando olhava para Luang Pu Kinari,
via que ele quase nunca fazia nada disso. Às vezes praticava só um
pouco, parava e ia remendar um pedaço de pano; se não fosse isso, ia
fazer algum outro trabalho.

``Eu olhava para ele com desdém, pensando: `Onde esse ajahn pensa que
vai chegar desse jeito? Não pratica meditação andando, não senta em
meditação por longos períodos, fica só ocupado fazendo isso e aquilo o
dia inteiro. Mas eu pratico sem parar e ainda assim não consegui
alcançar conhecimento algum. O que é que Luang Pu espera alcançar
praticando tão pouco?'

Mas era eu quem pensava errado: Luang Pu já tinha muito mais
conhecimento do que eu. Os ensinamentos dele eram curtos e muito raros,
mas eram profundos e munidos de sabedoria afiada. Seu pensamento era
muito mais amplo do que minha sabedoria jamais poderia ser. A verdadeira
essência da prática é o esforço para remover as impurezas de dentro do
coração; não devemos tomar apenas as posturas externas do mestre como
ponto de referência.''

Certo dia, Ajahn Chah estava com pressa, fazendo um trabalho de costura.
Não queria descansar porque pretendia terminar rápido para voltar a
praticar. Luang Pu Kinari passou por perto e o alertou com um
ensinamento curto: ``Prática é ter \emph{sati} o tempo todo, não importa
o que estiver fazendo. Praticar movido por ganância já é estar errado
desde o começo.''

Ajahn Chah viveu com Luang Pu Kinari apenas por um \emph{vassa}, mas
desde então, sempre que passava pela área, parava para visitar, prestar
reverência a Luang Pu e ouvir seus curtos ensinamentos e conselhos.
Sempre que Ajahn Chah falava sobre o hábito de Luang Pu Kinari de dar
ensinamentos curtos, ele fazia piada de si mesmo, porque naquela época
sua prática ainda não havia progredido muito e ele nem sempre conseguia
entender para onde Luang Pu estava apontando:

```Buddho, buddho! Se buddho estiver a três metros de distância, traga-o
de forma que fique a apenas um metro, traga-o até a distância de um
braço, traga-o para que fique ao alcance da mão, traga-o a um palmo de
distância. Traga-o ainda mais perto -- para dentro do seu coração, bem
aqui!', Luang Pu só dizia isso e eu não sabia do que ele estava falando:
`Puxe para perto como quem conduz um búfalo pelo nariz!'''

Em outra ocasião, Luang Pu Kinari ensinou sobre encontrar equilíbrio na
prática do Dhamma com uma frase curta, cujo significado Ajahn Chah levou
muito tempo para compreender: ``Maddi\footnote{A esposa do bodhisatta
  Vessantara. Vessantara foi a vida passada do Buddha na qual ele
  aperfeiçoou a qualidade da caridade (dāna pāramī).} não era alta ou
baixa, não era escura ou clara, não era gorda ou magra. Tudo nela era
ideal para que fosse bela.''

Por essas e outras razões, Ajahn Chah sempre teve muita gratidão por
Luang Pu Kinari e sempre o mencionava como um de seus mestres. Após
Ajahn Chah estar estabelecido em Wat Nong Pah Pong e Luang Pu Kinari ter
alcançado idade avançada, Ajahn Chah costumava enviar seus monges a Wat
Pah Nong Hi para cuidar dele, além de enviar medicamentos e recursos
materiais sempre que necessário. Após a morte de Luang Pu Kinari, foi
Ajahn Chah quem se responsabilizou por realizar todos os arranjos para o
funeral.

Ajahn Chah ainda permaneceu com Luang Pu Kinari durante a estação seca
de 1948, e então se despediu para continuar sua vida na estrada. Antes
que partisse, Luang Pu lhe deu um curto aviso, como era seu costume:
``Tahn Chah, tudo na sua prática está bom, mas peço que tenha cuidado
com dar ensinamentos.''

