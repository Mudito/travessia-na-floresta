\chapter{Nota do tradutor}

Com a data de comemoração do centenário de nascimento de Ajahn Chah se
aproximando, meu professor, Luang Pó Piak, notou que ainda hoje não
havia nenhum livro em língua ocidental que relatasse de forma concisa a
história de vida de Ajahn Chah. A princípio ele me pediu que preparasse
um texto em inglês, mas assim que terminei o primeiro manuscrito,
deixei-o sob responsabilidade da sangha de Wat Pah Nanachat para que se
encarregassem de produzir aquela versão do livro e então comecei a
trabalhar na versão em português que você agora tem em mãos.

Apesar de o primeiro rascunho ter sido composto em inglês, tomei o
cuidado de retraduzir todas as citações do tailandês para garantir a
maior fidelidade possível ao original. Também adaptei a grafia de nomes
tailandeses à fonética portuguesa (nomes como Ajahn Mun e Luang Pu
Kinaree foram escritos como Ajahn Man e Luang Pu Kinari). Um outro
detalhe que vale a pena mencionar é a escolha de traduzir o termo
\emph{tudong} (\thai{ธุดงค์}) como ``peregrinação''. É necessário explicar que,
diferente do que conhecemos no ocidente, no caso tailandês essa
``peregrinação'' nem sempre almeja ir a um local sagrado ou consiste em
``pagar uma promessa''. Em geral se trata de um modo de vida: os monges
simplesmente vivem como andarilhos, treinando a si mesmos a enfrentar
todo tipo de dificuldades e buscando locais isolados e conducentes à
prática de meditação.

O livro não foi composto com o objetivo de ser um documento completo
sobre a vida e obra de Ajahn Chah -- a quantidade de histórias e
ensinamentos disponíveis fariam a tarefa de compilar tal livro um
trabalho verdadeiramente hercúleo -- mas apenas dar uma boa noção de
quem era Ajahn Chah e de como ele chegou a ser o grande mestre que todos
conhecem hoje. A quem tem interesse em se aprofundar mais nos
ensinamentos de Ajahn Chah, recomendo a leitura de ``An Introduction to
the Life and Teachings of Ajahn Chah'', escrita por Ajahn Amaro e também
a coleção ``The Collected Teachings of Ajahn Chah'', ambos disponíveis
gratuitamente em \emph{forestsanghapublications.org}. Até o presente
momento, só existe uma compilação de ensinamentos de Ajahn Chah em
português que tenha sido traduzida direta do tailandês, chamada ``Darma
da Floresta'', que também está disponível através da Forest Sangha
Publications. Além disso, o site \emph{dhammadafloresta.org} oferece uma
quantidade grande de ensinamentos de Ajahn Chah, seus discípulos e
outros mestres da tradição da floresta tailandesa, assim como links para
mais fontes de informações.

O público-alvo desse trabalho são pessoas que já têm algum conhecimento
sobre budismo e mais especificamente a tradição Theravada da qual fazia
parte Ajahn Chah, mas, ainda assim, incluí inúmeras notas explicativas e
um glossário ao final do texto para auxiliar a compreensão dos termos
mais difíceis.

O texto aqui contido é uma compilação de várias fontes, mas
principalmente do livro \thai{อุปลมณี}, escrito por Ajahn Jayasaro; outras
fontes foram \thai{สุภทททานุสสรณ์} e \thai{ใต้ร่มโพธิญาณ}, ambos escritos por Ajahn
Mahā Amón Kemacitto; \thai{มิขันติคือให้พรแก่ตัวเอง}, por Mitsuo Shibaiashi;
Still Forest Pool, por Jack Kornfield e Paul Breiter, No Worries, por
Aruna Publications e diversas gravações de áudio e vídeo onde discípulos
de Ajahn Chah contavam suas histórias com o mestre.

Gostaria de agradecer a Ven. Gambhīro pela composição das edições
impressa e eletrônica do livro e Carla Pedro Schiavetto, Luciano Tadeu e
Ângelo de Vita por ajudarem na revisão do texto. Peço que quaisquer
omissões ou erros contidos neste trabalho sejam de minha inteira
responsabilidade, e que os méritos gerados aqui sejam de benefício aos
que ajudaram na composição deste livro assim como a todos que têm
respeito e reverência pela Aryia Sangha, cuja presença entre nós, ainda
na época atual, é o verdadeiro testemunho de que os ensinamentos do
Buddha jamais perderam sua validade ou eficácia para aqueles que de fato
estejam dispostos a praticá-los com humildade, sinceridade e dedicação.

Que todos possam estar livres de aflição e angústia. Que todos estejam
sempre acompanhados de bons amigos. Que todos encontrem o caminho da paz
e libertação.

\bigskip

{\raggedleft
Anumodanā.\\
Mudito Bhikkhu
\par}
