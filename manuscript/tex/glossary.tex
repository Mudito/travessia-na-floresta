\chapter{Glossário}

\begin{longtable}[c]{@{}ll@{}}
\toprule
Abhidhamma

(pāli) & Tratado técnico sobre aspectos psicológicos e metafísicos dos
ensinamentos do Buddha.\tabularnewline
Ajahn

(tailandês) & Professor. Esta palavra é a transcrição tailandesa do
termo em pāli ``acārya''.\tabularnewline
Anagārika

(pāli) & ``Sem-lar'', uma pessoa que renunciou à vida mundana, mas que
não fez votos monásticos. Em geral vestem uma roupa branca e seguem oito
preceitos: não matar, não roubar, celibato, fazer uso correto da fala,
não fazer uso de intoxicantes, alimentar-se somente no período entre o
nascer do sol e o meio-dia, não ouvir música, dançar, cantar, assistir a
espetáculos, enfeitar ou perfumar o corpo, não dormir em camas grandes e
luxuosas.\tabularnewline
Ānāpānasati

(pāli) & Técnica de meditação que utiliza a respiração como
foco.\tabularnewline
Añjali

(pāli) & Gesto com as mãos postas palma contra palma, como o gesto de
oração conhecido no ocidente.\tabularnewline
Arahant

(pāli) & Pessoa que alcançou nibbāna.\tabularnewline
Baht

(tailandês) & Denominação da moeda tailandesa.\tabularnewline
Bodhisatta

(pāli) & Aquele que almeja se tornar um Buddha.\tabularnewline
Buddha

(pāli) & Uma pessoa que desenvolveu ao máximo suas qualidades
espirituais (ver pāramī) com o objetivo tanto de alcançar a iluminação
como também de estabelecer um Buddha Sāsanā para ajudar os demais seres
a alcançar nibbāna.\tabularnewline
Buddha Sāsanā

(pāli) & O conjunto de recursos e suportes que um Buddha cria para
auxiliar as demais pessoas a alcançar a iluminação, ou seja, os textos,
os exercícios espirituais, os valores, os costumes, as tradições, etc.
De maneira superficial, poderia ser traduzido como ``o
budismo''.\tabularnewline
Dhamma

(pāli) & A doutrina do Buddha. Outro uso comum desta palavra é para se
referir a algo como ``a verdade transcendental'', a verdadeira natureza
de todas as coisas.\tabularnewline
Dukkha

(pāli) & Sofrimento, desconforto, insatisfação.\tabularnewline
Glot

(tailandês) & Uma espécie de guarda-chuva grande e robusto que os monges
utilizam quando acampam na floresta. Por cima dele é posta uma tela
contra mosquitos para proteger o monge durante a prática de
meditação.\tabularnewline
Jhāna

(pāli) & Estados profundos de samādhi. No ensinamento do Buddha são
catalogados oito estágios de jhāna.\tabularnewline
Kamma

(pāli) & Resultado de ações boas e ruins.\tabularnewline
Kammatthāna

(pāli/tailandês) & Em pāli significa ``fundações para prática''; na
Tailândia, o termo é usado para se referir às pessoas, principalmente
monges e monjas, que se dedicam à prática do Dhamma.\tabularnewline
Khanti

(pāli) & Resiliência, a capacidade de aguentar situações e sensações
desagradáveis.\tabularnewline
Kilesas

(pāli) & Impurezas mentais.\tabularnewline
Luang Pó

(tailandês) & Venerável pai, respeitado pai. Uma forma respeitosa de se
referir a monges mais velhos.\tabularnewline
Luang Ta

(tailandês) & Venerável vovô. Uma forma informal e às vezes ofensiva de
se referir a monges mais velhos.\tabularnewline
Nimitta

(pāli) & Imagens (ou sons, ou sensações) que às vezes surgem durante a
prática de meditação.\tabularnewline
Nibbāna

(pāli) & O objetivo final do caminho budista, a iluminação completa, o
fim do ciclo de nascimento e morte.\tabularnewline
Nīvaranas

(pāli) & As cinco obstruções à obtenção dos jhānas: desejo sensual, má
vontade, torpor e preguiça, inquietação e ansiedade,
dúvida.\tabularnewline
Pāli & Idioma usado na região norte da Índia na época do Buddha. É
derivado do magadi, tem semelhança com o sânscrito e os únicos registros
encontrados nesse idioma são os ensinamentos do Buddha.\tabularnewline
Pāramī

(pāli) & Boas qualidades mentais/espirituais, necessárias no caminho
para a iluminação.\tabularnewline
Pavāranā

(pāli) & Convite aberto para que o monge peça algo em caso de
necessidade. Por exemplo, um leigo pode colocar um valor à parte e
avisar ao monge que peça caso um dia esteja precisando de algo dentro
daquele valor. O leigo então se encarrega de providenciar o artigo
especificado.\tabularnewline
Pindapāta

(pāli) & O ato de os monges saírem pelas ruas recolhendo alimentos como
esmola.\tabularnewline
Pūja

(pāli) & Serviço devocional composto de gestos de louvor como fazer
prostrações, recitar cânticos, oferecer flores, velas, incenso,
etc.\tabularnewline
Sāmanera

(pāli) & Um monge noviço.\tabularnewline
Samsāra

(pāli) & O ciclo de nascimento e morte onde estão presos todos aqueles
que ainda não alcançaram nirvana.\tabularnewline
Sangha

(pāli) & Nome normalmente utilizado para se referir à comunidade de
monges budistas.\tabularnewline
Sāsanā

(pāli) & Ver ``Buddha Sāsanā''.\tabularnewline
Sati

(pāli) & A capacidade de aplicar a mente ao momento presente, estar
ciente; presença mental; faculdade de memória ativa, adepta em trazer e
manter em mente instruções e intenções que serão úteis em agir
habilmente no momento presente.\tabularnewline
Sīla

(pāli) & O aspecto do treinamento que diz respeito à disciplina e
moralidade dos atos corporais e verbais. Existem vários níveis de
prática de sīla: cinco preceitos praticados por discípulos leigos, oito
preceitos praticados por discípulos leigos e anagārikas, dez preceitos
praticados por noviços e noviças e as mais de duzentas regras do Vinaya
praticadas pelos monges e monjas.\tabularnewline
Sotāpanna

(pāli) & O primeiro estágio de iluminação. A partir desse ponto o
praticante nasce no máximo mais sete vezes entre seres humanos antes de
se tornar um arahant.\tabularnewline
Tahn

(pāli) & Senhor/senhora. Um título respeitoso em geral utilizado antes
do nome da pessoa.\tabularnewline
Theravada

(pāli) & A linhagem budista mais antiga ainda presente nos dias
atuais.\tabularnewline
Tipitaka

(pāli) & Conjunto de escrituras que contém os ensinamentos originais do
Buddha.\tabularnewline
Upajjhāya

(pāli) & Monge que preside a cerimônia de ordenação
monástica.\tabularnewline
Uposatha

(pāli) & Dia em que os monges se encontram para confessar suas
transgressões e recitar a regra monástica. Também é o nome dado ao
templo onde essa cerimônia é realizada.\tabularnewline
Vassa

(pāli) & \emph{Uma vez por ano, durante as monções, todos os monges
devem permanecer numa única residência por três meses. Em pāli esse
período se chama `vassa', mas no ocidente criou-se o hábito de chamá-lo
de `retiro das chuvas' ou `retiro das monções'.}\tabularnewline
Vinaya

(pāli) & \emph{A regra monástica criada pelo Buddha.}\tabularnewline
Vipassanā

(pāli) & O aspecto da prática de meditação que diz respeito a investigar
a realidade e enxergar com clareza.\tabularnewline
\bottomrule
\end{longtable}

