\chapter{Glossário}

\begin{glossarydescription}

\item[{{Abhidhamma}{(pāli)}}] Tratado técnico sobre aspectos psicológicos e metafísicos dos ensinamentos do Buddha.

\item[{{Ajahn}{(tailandês)}}] Professor. Esta palavra é a transcrição tailandesa do termo em pāli ``acārya''.

\item[{{Anagārika}{(pāli)}}] ``Sem-lar'', uma pessoa que renunciou à vida mundana, mas que não fez votos monásticos. Em geral vestem uma roupa branca e seguem oito preceitos: não matar, não roubar, celibato, fazer uso correto da fala, não fazer uso de intoxicantes, alimentar-se somente no período entre o nascer do sol e o meio-dia, não ouvir música, dançar, cantar, assistir a espetáculos, enfeitar ou perfumar o corpo, não dormir em camas grandes e luxuosas.

\item[{{Ānāpānasati}{(pāli)}}] Técnica de meditação que utiliza a respiração como foco.

\item[{{Añjali}{(pāli)}}] Gesto com as mãos postas palma contra palma, como o gesto de oração conhecido no ocidente.

\item[{{Arahant}{(pāli)}}] Pessoa que alcançou nibbāna.

\item[{{Baht}{(tailandês)}}] Denominação da moeda tailandesa.

\item[{{Bodhisatta}{(pāli)}}] Aquele que almeja se tornar um Buddha.

\item[{{Buddha}{(pāli)}}] Uma pessoa que desenvolveu ao máximo suas qualidades espirituais (ver pāramī) com o objetivo tanto de alcançar a iluminação como também de estabelecer um Buddha Sāsanā para ajudar os demais seres a alcançar nibbāna.

\item[{{Buddha Sāsanā}{(pāli)}}] O conjunto de recursos e suportes que um Buddha cria para auxiliar as demais pessoas a alcançar a iluminação, ou seja, os textos, os exercícios espirituais, os valores, os costumes, as tradições, etc. De maneira superficial, poderia ser traduzido como ``o budismo''.

\item[{{Dhamma}{(pāli)}}] A doutrina do Buddha. Outro uso comum desta palavra é para se referir a algo como ``a verdade transcendental'', a verdadeira natureza de todas as coisas.

\item[{{Dukkha}{(pāli)}}] Sofrimento, desconforto, insatisfação.

\item[{{Glot}{(tailandês)}}] Uma espécie de guarda-chuva grande e robusto que os monges utilizam quando acampam na floresta. Por cima dele é posta uma tela contra mosquitos para proteger o monge durante a prática de meditação.

\item[{{Jhāna}{(pāli)}}] Estados profundos de samādhi. No ensinamento do Buddha são catalogados oito estágios de jhāna.

\item[{{Kamma}{(pāli)}}] Resultado de ações boas e ruins.

\item[{{Kammatthāna}{(pāli/tailandês)}}] Em pāli significa ``fundações para prática''; na Tailândia, o termo é usado para se referir às pessoas, principalmente monges e monjas, que se dedicam à prática do Dhamma.

\item[{{Khanti}{(pāli)}}] Resiliência, a capacidade de aguentar situações e sensações desagradáveis.

\item[{{Kilesas}{(pāli)}}] Impurezas mentais.

\item[{{Luang Pó}{(tailandês)}}] Venerável pai, respeitado pai. Uma forma respeitosa de se referir a monges mais velhos.

\item[{{Luang Ta}{(tailandês)}}] Venerável vovô. Uma forma informal e às vezes ofensiva de se referir a monges mais velhos.

\item[{{Nimitta}{(pāli)}}] Imagens (ou sons, ou sensações) que às vezes surgem durante a prática de meditação.

\item[{{Nibbāna}{(pāli)}}] O objetivo final do caminho budista, a iluminação completa, o fim do ciclo de nascimento e morte.

\item[{{Nīvaranas}{(pāli)}}] As cinco obstruções à obtenção dos jhānas: desejo sensual, má vontade, torpor e preguiça, inquietação e ansiedade, dúvida.

\item[{{Pāli}{(pāli)}}] Idioma usado na região norte da Índia na época do Buddha. É derivado do magadi, tem semelhança com o sânscrito e os únicos registros encontrados nesse idioma são os ensinamentos do Buddha.

\item[{{Pāramī}{(pāli)}}] Boas qualidades mentais/espirituais, necessárias no caminho para a iluminação.

\item[{{Pavāranā}{(pāli)}}] Convite aberto para que o monge peça algo em caso de necessidade. Por exemplo, um leigo pode colocar um valor à parte e avisar ao monge que peça caso um dia esteja precisando de algo dentro daquele valor. O leigo então se encarrega de providenciar o artigo especificado.

\item[{{Pindapāta}{(pāli)}}] O ato de os monges saírem pelas ruas recolhendo alimentos como esmola.

\item[{{Pūja}{(pāli)}}] Serviço devocional composto de gestos de louvor como fazer prostrações, recitar cânticos, oferecer flores, velas, incenso, etc.

\item[{{Sāmanera}{(pāli)}}] Um monge noviço.

\item[{{Samsāra}{(pāli)}}] O ciclo de nascimento e morte onde estão presos todos aqueles que ainda não alcançaram Nibbāna.

\item[{{Sangha}{(pāli)}}] Nome normalmente utilizado para se referir à comunidade de monges budistas.

\item[{{Sāsanā}{(pāli)}}] Ver ``Buddha Sāsanā''.

\item[{{Sati}{(pāli)}}] A capacidade de aplicar a mente ao momento presente, estar ciente; presença mental; faculdade de memória ativa, adepta em trazer e manter em mente instruções e intenções que serão úteis em agir habilmente no momento presente.

\item[{{Sīla}{(pāli)}}] O aspecto do treinamento que diz respeito à disciplina e moralidade dos atos corporais e verbais. Existem vários níveis de prática de sīla: cinco preceitos praticados por discípulos leigos, oito preceitos praticados por discípulos leigos e anagārikas, dez preceitos praticados por noviços e noviças e as mais de duzentas regras do Vinaya praticadas pelos monges e monjas.

\item[{{Sotāpanna}{(pāli)}}] O primeiro estágio de iluminação. A partir desse ponto o praticante nasce no máximo mais sete vezes entre seres humanos antes de se tornar um arahant.

\item[{{Tahn}{(pāli)}}] Senhor/senhora. Um título respeitoso em geral utilizado antes do nome da pessoa.

\item[{{Theravada}{(pāli)}}] A linhagem budista mais antiga ainda presente nos dias atuais.

\item[{{Tipitaka}{(pāli)}}] Conjunto de escrituras que contém os ensinamentos originais do Buddha.

\item[{{Upajjhāya}{(pāli)}}] Monge que preside a cerimônia de ordenação monástica.

\item[{{Uposatha}{(pāli)}}] Dia em que os monges se encontram para confessar suas transgressões e recitar a regra monástica. Também é o nome dado ao templo onde essa cerimônia é realizada.

\item[{{Vassa}{(pāli)}}] Uma vez por ano, durante as monções, todos os monges devem permanecer numa única residência por três meses. Em pāli esse período se chama `vassa', mas no ocidente criou-se o hábito de chamá-lo de `retiro das chuvas' ou `retiro das monções'.

\item[{{Vinaya}{(pāli)}}] A regra monástica criada pelo Buddha.

\item[{{Vipassanā}{(pāli)}}] O aspecto da prática de meditação que diz respeito a investigar a realidade e enxergar com clareza.

\end{glossarydescription}

