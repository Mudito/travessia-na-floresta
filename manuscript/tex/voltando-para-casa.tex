\chapter{Voltando para casa}

De Wat Pah Nong Hi, Luang Pó caminhou até Ubon Ratchathani, parando em
Pah Tao para ensinar. Quando chegou a Ban Kó, seu vilarejo natal,
acampou no cemitério de floresta em Ban Kó Nók e lá visitou sua mãe,
seus parentes e começou a instruir dois rapazes que estavam interessados
em seguir a vida monástica. Quando ambos já haviam recebido instrução
suficiente para se ordenar, Luang Pó os levou a Wat Warintaram para que
pudessem receber ordenação como noviços.

Sabendo de sua estadia em Ban Kó Nók, parentes e velhos amigos de
infância ficaram felizes e vieram lhe prestar reverência e conversar com
Luang Pó, porque desde que ele havia partido em peregrinação, nunca mais
haviam tido chance de vê-lo. Mas ao encontrá-lo naquela ocasião, notaram
que o amado amigo de infância que havia acabado de retornar estava muito
mudado. Antes uma pessoa falante e alegre que gostava de dar risada,
agora era uma pessoa quieta e séria, que falava pouco e não ria tanto
quanto antigamente.

Luang Pó permaneceu naquele cemitério de floresta por 15 dias. Instruiu
sua mãe, parentes e amigos e deu conselhos a eles, e em seguida partiu
para Kantarat, em Sri Saket, levando consigo um dos recém-ordenados
noviços; eles acamparam em uma floresta perto do vilarejo de Ban Suan
Gluei. Luang Pó viu que aquele lugar era apropriado para a prática de
meditação, porque era pacífico, recluso e habitado por muitos animais
selvagens como esquilos, galinhas selvagens, veados, ocelotes\footnote{Espécie
  de predador felino de pequeno porte.} e tigres. Todas essas
características faziam com que o local fosse muito apropriado para o
treinamento do Dhamma e ele então decidiu passar o \emph{vassa} de 1949
naquela floresta.

Durante sua estadia, Luang Pó teve três sonhos que considerou serem
muito importantes. Infelizmente ele nunca explicou claramente qual o
significado desses sonhos, e seus discípulos têm diferentes opiniões
sobre o assunto. Alguns dizem, por exemplo, que o significado do
primeiro sonho é de que, para atingir o estágio mais elevado do Dhamma,
ele teria que morrer para o mundo, e só então não haveria mais nada
dentro de si a não ser o Dhamma puro. Outros clamam que o segundo sonho
prediz que Luang Pó daria luz à sangha de monges ocidentais, mas, por
causa de seu \emph{kamma}, não teria a oportunidade de cuidar deles
(como veremos mais adiante, Luang Pó adoeceu e ficou incapacitado de
ensinar relativamente cedo). Já o terceiro previa que ele atingiria a
realização mais elevada do Dhamma. De qualquer forma, o real significado
desses sonhos é algo que só mesmo Ajahn Chah poderia saber com certeza,
mas, mesmo sem uma explicação clara, julgamos apropriado incluí-los
aqui, uma vez que ele os mencionou como eventos relevantes em sua vida.
Ele os descreveu da seguinte maneira:

\begin{enumerate}
\def\labelenumi{\arabic{enumi}.}
\item
  Certa noite, tendo terminado sua prática de meditação, deitou-se para
  dormir e sonhou com uma pessoa lhe oferecendo um ovo. Luang Pó recebeu
  o ovo e o jogou no chão, em frente a si. Quando o ovo quebrou, dois
  pintinhos saíram de dentro e correram em direção a ele. Luang Pó
  estendeu suas mãos para recebê-los, um em cada mão, e naquele instante
  ambos tornaram-se garotos. Assim que isso ocorreu, ouviu uma voz que
  dizia que o garoto em sua mão direita se chamava ``Mérito do Dhamma'',
  e o garoto em sua mão esquerda, ``Mérito do Ouro''. Luang Pó cuidou de
  ambos até que crescessem e pudessem correr e brincar sozinhos. Então
  Mérito do Ouro contraiu uma disenteria severa e, por mais que Luang Pó
  se esforçasse em cuidar dele, não melhorou e eventualmente morreu em
  seus braços. Quando isso ocorreu, Luang Pó ouviu uma voz dizer:
  ``Mérito do Ouro está morto, agora só lhe resta Mérito do Dhamma.''
  Nesse momento, Luang Pó despertou e perguntou a si mesmo: ``Que foi
  isso?'' A resposta veio logo em seguida: ``Isso é uma manifestação
  espontânea do Dhamma.''
\item
  Na noite seguinte o mesmo ocorreu. Assim que começou a dormir, sonhou
  que estava grávido e movendo-se com dificuldade, como fazem as pessoas
  grávidas; porém, no sonho ele ainda era um monge. Quando sua barriga
  estava grande e ele estava a ponto de dar à luz, uma pessoa apareceu e
  o convidou a receber comida como esmola. Ele olhou ao redor e viu que
  estava um campo alagado e que lá havia uma cabana de bambu numa área
  seca em meio ao campo. Ele também viu três monges chegando numa canoa.
  Os leigos ofereceram a comida e os três monges sentaram-se na parte de
  cima da cabana para comer, mas pediram que Luang Pó sentasse na parte
  de baixo, pois achavam que ele já ia dar à luz. Assim que terminou de
  comer, deu à luz a um garoto que possuía pelos delicados na palma das
  mãos e na sola dos pés; sua fisionomia era feliz e clara. A barriga de
  Luang Pó ficou murcha e ele achava que realmente havia dado à luz, por
  isso tocou seu corpo, mas viu que não estava sujo de sangue ou
  qualquer outro líquido corporal. Então chegou a hora de mais uma
  refeição e os leigos discutiam sobre o que deveriam oferecer, tendo em
  vista que ele havia acabado de dar à luz. Por fim lhe trouxeram três
  peixes assados. Luang Pó se sentia fraco e cansado e não queria comer,
  mas fez um esforço em comer por consideração aos leigos que haviam
  oferecido o alimento. Antes de começar a comer, pediu que um dos
  leigos segurasse a criança e, quando Luang Pó terminou sua refeição, o
  leigo ergueu a criança para devolvê-la, mas ela escapou e caiu no chão
  -- e nesse exato momento Luang Pó despertou. Ele se perguntou ``Que
  foi isso?'', e a resposta surgiu: ``É um fenômeno que ocorre
  sozinho.'', e assim a questão se encerrou.
\item
  Na terceira noite ele teve mais um sonho. Desta vez sonhou que havia
  recebido um convite para viver no topo de uma montanha junto com um
  noviço. O caminho até o topo era uma espiral, como a casca de um
  caramujo. Era noite de lua cheia e a montanha era muito alta. Quando
  ele finalmente atingiu o topo, sentiu uma atmosfera fresca e
  agradável, e lá havia um abrigo com o chão coberto de tecido e um
  telhado muito bonito, como jamais havia visto antes. Quando chegou a
  hora da refeição, o convidaram a uma caverna e lá estavam sua mãe e
  sua tia Mi, junto com muitos outros leigos que aguardavam a
  oportunidade de lhe oferecer comida. A refeição era composta de melões
  e outras frutas oferecidas por sua mãe, além de galinha e pato assado,
  oferecidos por sua tia. Ao ver aquela comida, Luang Pó disse: ``Me
  parece que Mi gosta de viver no mercado. Trouxe galinha e pato assado
  para oferecer aos monges.'' Ouvindo isso, Mi ficou feliz e sorriu. Ao
  fim da refeição ele ensinou Dhamma aos presentes e, tendo terminado,
  despertou.
\end{enumerate}
