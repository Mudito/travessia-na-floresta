\thistitleoffsettrue
\chapter{Infância}

Ajahn Chah, carinhosamente chamado de ``Luang Pó'' (venerável pai) por
seus discípulos, nasceu numa sexta-feira, dia 17 de junho de 1918, num
pequeno vilarejo no distrito de Warin Chamrap chamado Ban Kó, localizado
na província de Ubon Ratchatani -- um lugar famoso por ter sido lar de
muitas sábias personalidades no passado. Filho de Sr. Maa e Sra. Pim
Chuang Choti, tinha um total de dez irmãos e irmãs. Luang Pó foi criado
em uma família em que havia muito carinho e harmonia, sendo ela uma das
mais abastadas do vilarejo e possuidora do hábito de ajudar as famílias
mais pobres quando os tempos eram difíceis.

Quando criança, Luang Pó era rechonchudo e barrigudo. Um de seus amigos
de infância, Put Tumakon, conta que seu comportamento, já quando criança
pequena, era de ser bastante extrovertido e um líder natural. Sempre que
estava com seus amigos, quer brincando ou fazendo qualquer outra coisa,
era ele quem fazia os planos e delegava tarefas. Na maior parte do
tempo, era bem-humorado e alegre e, quando não estava por perto, as
demais crianças se sentiam entediadas e sem ânimo -- conversar e brincar
já não era tão divertido como antes. Crianças, especialmente aquelas que
vivem no campo, gostam de jogos de ação e competição, de brincadeiras
energéticas, como brincar de soldados, polícia-e-ladrão e assim por
diante, mas o modo de brincar do jovem Chah era um pouco incomum. Em
certa ocasião Luang Pó falou a respeito:

``Quando eu era criança, queria brincar de ser monge, então me declarava
abade e vestia um pano branco como se fosse um manto monástico. Quando
chegava o horário do almoço, eu batia um sino e meus amigos que na
brincadeira eram os leigos traziam água para me oferecer, recebiam os
cinco preceitos\footnote{É comum as pessoas realizarem uma cerimônia que
  simboliza o ``receber'' dos cinco preceitos. Para saber mais sobre os
  cinco preceitos consulte ``sīla'' no glossário.} e uma benção.''

Outra característica aparente desde sua infância era seu amor pela paz
-- ninguém nunca o presenciou tendo problemas com ninguém, muito menos
brigando com alguém mais fraco que ele. Na verdade era o oposto: quando
as demais crianças estavam brigando, era ele quem apaziguava ambos os
lados e encerrava a briga de forma pacífica com uma habilidade que só
ele possuía. Por sua boa índole, generosa e justa, seus amigos sempre
respeitavam seu julgamento em tais ocasiões.

Luang Pó era uma criança energética, vigorosa e ágil que gostava de
comer bem. Mas também era trabalhador e não ficava à toa, e por isso
começou a ajudar com tarefas domésticas desde muito cedo em sua vida. As
duas principais responsabilidades que possuía eram: alimentar os búfalos
e cuidar da plantação de tabaco. Logo cedo, após o café da manhã, o
pequeno Chah preparava uma sacola de comida para si, tocava os búfalos
para fora do curral e os levava a uma área do campo onde havia grama.
Após deixá-los comendo grama, ele saía à procura de comestíveis nas
redondezas, como sapos, peixes, cogumelos e brotos de bambu para servir
de alimento para a família à noite. Mas o trabalho mais pesado era nas
quatro ou cinco plantações de tabaco pertencentes à família. Ele tinha
que ajudar a regar e tomar conta das plantas até que as folhas fossem
colhidas e levadas ao mercado para serem trocadas por outros bens,
comida e utensílios.

Já aos nove anos, enquanto ajudava em tarefas domésticas com dedicação
total, Chah começou a sentir interesse pela vida no monastério. Após
terminar seus estudos primários na escola do vilarejo, ele decidiu que
queria se tornar um atendente do monastério, o que lhe daria a
oportunidade de morar lá e estudar o modo de vida dos monges. Muitos
anos mais tarde, quando Luang Pó já tinha idade avançada, um grupo de
ocidentais veio visitar Wat Nong Pah Pong e lhe perguntou qual foi a
motivação que o levou a querer viver no monastério ainda quando criança.
Luang Pó respondeu:

``Antes de me tornar monge? Fazia parte da minha personalidade ter medo
de más ações, eu era sempre honesto e sincero, nunca mentia a ninguém,
sempre tive um comportamento limpo. Mesmo quando partilhava algo, eu
sempre gostava de ganhar menos que as outras pessoas -- eu tinha
consideração por elas. Sempre fui assim. Quando essa natureza começou a
amadurecer, o pensamento de morar no monastério surgiu -- era assim que
eu pensava. Eu perguntei aos meus amigos se eles também pensavam nisso,
mas eles disseram que nunca o haviam feito. Essas coisas acontecem por
si mesmas, são resultados de minhas ações, são resultados que surgem
naturalmente. Eu frequentemente pensava nisso e a ideia crescia
continuamente, me fazia agir daquela forma, me fazia pensar daquela
forma.''

Em outra ocasião ele contou aos leigos -- meio sério, meio brincando --
que foi morar no monastério porque estava cansado de regar os pés de
tabaco e se sentia oprimido pelo trabalho em casa, tão repetitivo e que
lhe parecia sem fim: ``Eu era um garoto pequeno, nunca fumei com os
adultos, mas toda manhã eles me tocavam para fora de casa para regar
centenas de pés de tabaco -- era de amargar\ldots{}''

Houve um evento que aparentemente foi o que fez a sensação no coração de
Luang Pó explodir e o levou a decidir-se de uma vez por todas a ir morar
no monastério. Sua irmã contou a história: ``Ele ter ido morar no
monastério não ocorreu porque as pessoas lá de casa organizaram, foi ele
mesmo quem tomou a iniciativa. Um dia estava ajudando os irmãos a
descascar o arroz colhido, porém o fazia com má vontade. Aconteceu de a
ferramenta usada para bater o arroz estar frouxa e precisar ser calçada
com um pedaço de madeira, mas ele se recusou a fazê-lo. Outra pessoa
então foi bater o calço de madeira e ele escapou, voou no ar e o acertou
em cheio. Deve ter doído porque ele gritou: `Chega! Vou virar monge!'''

Algum tempo mais tarde, Chah pediu para que seu pai e sua mãe o levassem
ao monastério e o deixassem tornar-se um discípulo. Os pais não se
opuseram e o deixaram sob a tutela de Ajahn Si, em Wat Ban Kó Nók. Lá
ele teve a oportunidade de aprender pela primeira vez as regras de
comportamento e atividades diárias de um monastério budista. Seu amigo
Put foi viver no mesmo monastério sob tutela de Ajahn Pon, e desta forma
Chah tinha alguém para lhe fazer companhia.

Após tornar-se um atendente no monastério, quando Chah já havia recebido
instrução suficiente e já tinha idade para se tornar um noviço, o abade,
vendo que ele era bem comportado, diligente e capaz de cumprir suas
responsabilidades, organizou uma cerimônia de ordenação de noviços para
ele e vários de seus amigos em Wat Ban Kó, com Prah Kru Wichit
Dhammapani (Puang), abade de Wat Moniwanaram em Ubon Ratchatani, atuando
como \emph{upajjhāya}.\footnote{Monge que preside a cerimônia de
  ordenação monástica (pāli).} O evento ocorreu em março de
1931, quando Luang Pó tinha 13 anos. Após ordenar-se, além de aprender
todos os cânticos normalmente recitados pelos monges, Sāmanera Chah
estudou o primeiro nível do curso oficial de Dhamma (Nak Thamm
Tri).

Durante esse período, Sāmanera Chah frequentemente servia um monge
chamado Ajahn Lang, que acabou desenvolvendo muita afeição por ele.
Ajahn Lang passou a se responsabilizar não só por ensiná-lo, mas também
por supervisionar todos seus estudos e, como consequência, acabou
criando amizade com a família de seu discípulo. Sempre que havia tempo
livre, Ajahn Lang encorajava Sāmanera Chah a ir visitar sua família e
usava isso como pretexto para ir junto. Após algum tempo, essas visitas
começaram a ficar mais e mais frequentes e às vezes já era noite antes
que eles voltassem para o monastério.

Em seguida, Ajahn Lang começou a passar mais e mais tempo conversando
sobre assuntos mundanos com seu discípulo, até que um dia finalmente
declarou que estava planejando largar a vida monástica e encorajou
Sāmanera Chah a fazer o mesmo. O coração do jovem noviço vacilou diante
da proposta, pois sua fé no \emph{Buddha Sāsanā} ainda não era tão firme
a ponto de ser capaz de continuar sozinho quando até mesmo seu professor
desistia. Quando Ajahn Lang começou a repetir esse pedido com
frequência, Sāmanera Chah concordou e deixou a vida monástica aos 16
anos de idade. Pouco tempo após deixar a ordem monástica, Lang pediu Sa
Chuang Choti, irmã mais velha de Cha, em casamento, mas o matrimônio não
durou muito e logo o casal se separou.

Após deixar o manto monástico e voltar para casa, mais uma vez Chah
tornou-se uma importante força em todas as tarefas domésticas da
família, especialmente na colheita, que era a principal fonte de
recursos da casa. Seu pai e sua mãe estavam muito felizes com seu
retorno, mas Chah ainda sentia que faltava substância à vida laica. Anos
mais tarde ele contou a seus discípulos sobre seus sentimentos naquela
época:

``Cansado de tudo. Não queria viver com meu pai e minha mãe,
frequentemente pensava que queria ir morar sozinho mas não sabia para
onde ir. Isso continuou por vários anos. Eu gostava de pensar comigo
mesmo: `Cansado!', mas não sabia do que estava cansado. Queria ir para
algum lugar para poder ficar sozinho. Isso durou um bom tempo até que me
ordenei. Era algo que já fazia parte da minha personalidade, mas eu
ainda não sabia. Minha sensação era sempre aquela.''

Uma vez que não conseguia encontrar uma saída para sua situação, Chah
procurava se manter ocupado com diversões como forma de aliviar essa
sensação de desânimo. Um desses divertimentos era sair com seus amigos,
e entre eles estava sempre seu velho amigo Put. Eles saíam em busca de
diversão como os demais rapazes faziam na época. Às vezes iam flertar
com garotas dos vilarejos vizinhos e às vezes dos mais distantes também.
Nessas ocasiões, os amigos de Chah podiam testemunhar sua grande
resistência física: quando uma festa ocorria numa localidade distante,
às vezes tinham que caminhar mais de trinta quilômetros para o trajeto
de ida e volta; seus amigos queriam parar e descansar, mas Chah não dava
ouvidos: só parava quando alcançava seu destino.

Luang Pó vivia em Ban Kó Nók e Put, em Ban Kó Nai. Os dois vilarejos
ficavam cerca de um quilômetro de distância um do outro. Para visitar um
ao outro eles tinham que atravessar uma floresta que os habitantes da
região temiam, pois acreditavam ser mal-assombrada. Por essa razão,
sempre que saíam juntos e voltavam tarde, tinham que dormir na casa de
um ou do outro porque ambos tinham muito medo de fantasmas. Nenhum deles
tinha coragem de caminhar sozinho de volta para casa, e esse medo de
fantasmas era algo forte e constante na vida de Chah, até que conseguiu
vencê-lo muitos anos após se ordenar monge, como veremos mais adiante.

Apesar de Chah e Put frequentemente irem flertar com garotas dos
vilarejos próximos, Chah acabou se apaixonando por Jai, uma filha
adotiva da mãe de Put. Put era mais próximo a seu pai do que à sua mãe,
e a casa onde vivia ficava longe da casa onde Jai morava. O romance
entre Chah e Jai era conhecido de todos e ninguém se opunha a ele, menos
ainda seus pais adotivos, que estimavam Chah como um filho, sendo ele um
rapaz de boa família e cheio de boas qualidades. Talvez por isso, os
pais da moça se opunham a qualquer outro pretendente que viesse
visitá-la, não os deixando sequer entrar em casa.

Cha e Jai prometeram um ao outro que esperariam até que ele tivesse sido
dispensado do serviço militar obrigatório e se ordenado por um período
curto para pagar seu débito de gratidão a seus pais, como era costumeiro
rapazes fazerem na Tailândia ao completarem 20 anos de idade. Só então,
quando tudo isso estivesse terminado e todos os preparativos feitos,
eles se casariam. Na época, Chah tinha dezenove anos de idade e Jai,
dezessete.

Enquanto a estação chuvosa daquele ano se aproximava e as famílias da
região se preparavam para começar o plantio da lavoura, na casa de Put
os pais discutiam o trabalho no campo, que passava por dificuldades
devidas à falta de trabalhadores. Ambos concordavam que o melhor seria
que Jai se casasse para que seu marido viesse ajudar com o trabalho, mas
eles não conseguiam decidir quem poderia ser um bom candidato. Chah, que
era seu atual pretendente, ainda não estava pronto, e eles teriam que
esperar vários anos antes que estivesse. Ao final o pai disse: ``Que ela
se case com nosso filho, Put!'' A razão por trás disso era o fato de que
ambos os jovens se conheciam bem, como se fossem irmãos, mas não eram de
fato irmãos biológicos. Além disso, havia um fator econômico, como dizia
um ditado da região: ``Se o barco afundar na lagoa, o tesouro não vai se
perder no oceano!'', que significa que com esse arranjo eles não teriam
que dividir os bens da família com um estranho.

Por respeito a seus pais, Put e Jai -- mesmo se sentindo
desconfortáveis, uma vez que desde sempre se enxergaram como irmão e
irmã -- não se atreveram a opor o desejo deles. Muitos anos mais tarde,
Luang Pó falou a seus discípulos sobre o sentimento em seu peito no
momento em que recebeu tal notícia tão inesperada:

``Quando tinha dezoito anos eu gostava de uma garota e acho que ela
também gostava de mim. Gostávamos um do outro à maneira das pessoas do
campo. Estava tão apaixonado a ponto de ficar completamente amarrado.
Como as pessoas dizem, queria que ela fosse minha mulher. Sonhava em
tê-la ao meu lado, me ajudando na lavoura a ganhar vida de acordo com os
modos do mundo. Um dia eu estava voltando do campo e passei em frente à
casa do meu melhor amigo. Ele me disse: `Chah\ldots{} Vou me casar com a
garota.' Eu escutei e meu corpo inteiro ficou dormente, fiquei em choque
por várias horas. Pensei na profecia de um vidente que um dia havia dito
que eu não teria esposa, mas vários filhos\footnote{Aqui há um jogo de
  palavras: o vocábulo tailandês para filho é ``luk'' (\thai{ลูก}) e a palavra
  para discípulo é ``luksit'' (\thai{ลูกศิษย์}). Além disso, na Tailândia é
  comum se referir ao Ajahn como ``venerável pai'', o que por
  consequência implica que os discípulos seriam como seus filhos e
  filhas.}. Na época eu não conseguia entender como isso poderia ser
possível.''

No final, Chah conseguiu aceitar a notícia e não sentiu raiva de seu
amigo, pois sabia que ele não agia por má fé, muito pelo contrário: ele
teve que se forçar a obedecer à ordem de seus pais. Mas essa decepção
foi uma importante lição sobre como a vida é incerta -- algo que no
futuro se tornaria um dos ensinamentos que Luang Pó mais ressaltava a
seus discípulos.

O jovem Chah preservou sua amizade por Put como se nada tivesse
acontecido, mas com relação a Jai foi o oposto. Luang Pó disse que ele
tinha que ser muito cuidadoso -- mesmo após ter se ordenado monge, toda
vez que via Jai se aproximando ele corria e se escondia, por medo de não
conseguir controlar suas emoções. Luang Pó afirmava que durante os
primeiros sete anos de sua vida monástica ainda não conseguia superar
seu pesar pela perda de Jai. Somente após ter começado sua vida como
monge andarilho e passado a dedicar plenamente sua vida à prática do
Dhamma é que a sensação desapareceu.

Muitos anos mais tarde, quando já era abade de Wat Nong Pah Pong e
estava ensinando monges e noviços sobre as desvantagens dos prazeres
sensuais, Luang Pó frequentemente dava o exemplo de Put como uma pessoa
para a qual ele tinha uma dívida de gratidão imensurável: ``Se ele não
tivesse se casado com Jai, eu provavelmente não teria me ordenado.'' Era
natural para ele se expressar dessa forma, uma vez que era uma pessoa de
grande humildade, mas se olharmos em maior detalhe a grande quantidade
de \emph{pāramī}\footnote{Boas qualidades mentais/espirituais,
  necessárias no caminho para a iluminação (pāli).} que possuía, é
difícil não ter a sensação de que se não fosse por esse incidente
específico, alguma outra coisa certamente o levaria à vida monástica,
uma vez que é fácil ver que sua vida fluía continuamente em direção ao
Dhamma, como se nada pudesse impedi-lo.

Em sua vida de praticante do Dhamma, o pior inimigo que por muitos anos
o obrigou a lutar continuamente até finalmente conseguir derrotar foi
\emph{kāmma-rāga} -- desejo sexual. Mesmo quando ainda era leigo, já
começou a ter problemas com esse inimigo. Naquela ocasião, um evento
ocorreu que o forçou a enfrentar \emph{kāmma-rāga} face a face e serviu
como um prelúdio para os desafios que o aguardavam mais à frente, quando
se ordenasse monge e se deparasse com desafios ainda maiores. Eis o
ocorrido:

No monastério em que viveu durante sua primeira ordenação como sāmanera,
Luang Pó fez amizade com um rapaz mais velho que na época estava
ordenado como monge, e mesmo após ambos abandonarem a vida monástica
eles ainda visitavam um ao outro e davam continuidade à sua amizade.
Após algum tempo aquele amigo adoeceu e veio a falecer e Chah, já de
volta à vida laica, esteve presente do primeiro ao último dia do funeral
e ajudou com todos os arranjos necessários. Quando o corpo foi cremado e
tudo estava encerrado, os amigos do falecido foram embora e voltaram às
suas casas, deixando Chah para trás porque ele, sendo um amigo íntimo da
família, decidiu fazer-lhes companhia por mais alguns dias, para que não
tivessem medo de ficarem sozinhos.

Quando chegou a hora de dormir, a viúva e as crianças foram para dentro
do quarto e Chah dormiu na varanda coberta, em frente à casa, e a
primeira noite passou sem nenhum evento. Na segunda noite a viúva saiu
do quarto e deitou-se ao lado dele, pegou sua mão e o fez tocar várias
partes de seu corpo. Chah, por sua vez, continuou fingindo que estava
dormindo como se nada estivesse acontecendo. Quando a viúva percebeu que
suas intenções não eram recíprocas, ela se levantou e voltou para dentro
do quarto para dormir com as crianças.

Naquela noite o jovem Chah provavelmente sentiu-se muito confuso e
agitado mas, ainda assim, foi a primeira vitória de sua vida contra
\emph{kāmma-rāga}. Ele a derrotou controlando a si mesmo por respeito a
seu amigo que acabara de falecer, por ter noção do que é correto e
apropriado, e também por ser mais forte que as \emph{kilesas} que
habitavam seu coração. Naquele instante, o pensamento de certo e errado,
bem e mal, estavam firmemente estabelecidos em sua mente. Esse relato já
nos mostra a presença das qualidades de autocontrole e vergonha de
cometer más ações que, mais tarde, quando tomasse o caminho monástico,
se tornariam as características mais fortes de sua personalidade.

A sensação de desencanto resultante desse primeiro contato com a
realidade do modo de ser das pessoas no mundo e suas desonestidades
começou a fomentar um certo tipo de pensamento, profundo em sua mente,
que aos poucos se transformou na firme determinação de tornar-se monge e
buscar o caminho da libertação.

