\chapter{A chegada dos ocidentais}

Talvez o que mais ocupasse Luang Pó nos seus últimos anos de vida era a
presença de homens e mulheres ocidentais que vinham até Wat Nong Pah
Pong interessados em estudar o Dhamma. Apesar de que aprender a ensinar
esses novos visitantes e lidar com eles tenha sido um verdadeiro desafio
para Ajahn Chah, por outro lado também serviu para fazer com que toda a
sua destreza em transmitir o Dhamma viesse à tona e brilhasse ainda mais
do que antes.

O primeiro a chegar foi um americano chamado Robert Jackman, que mais
tarde se tornaria monge e receberia o nome em \emph{pāli} ``Sumedho''.
Robert nutria um interesse pela cultura oriental desde seus dias de
universidade e, em parte por isso, voluntariou-se para fazer parte do
``U.S. Peace Corps'', o que lhe deu a oportunidade de viajar para muitos
países asiáticos e, finalmente, ter contato com as culturas que vinha
estudando havia tanto tempo. Ele eventualmente chegou à Tailândia para
trabalhar como professor de inglês e começou a estudar Budismo
Theravada. Em 1966 se ordenou noviço em Wat Sri Saket, na província de
Nong Khai, e um ano mais tarde recebeu ordenação como monge. Através de
um encontro casual com um monge tailandês, discípulo de Ajahn Chah,
ficou sabendo da existência de Wat Nong Pah Pong e sentiu interesse em
visitar o local. Tahn Sumedho então pediu permissão a seu
\emph{upajjhāya} para partir e, após chegar a Wat Nong Pah Pong,
tornou-se discípulo de Ajahn Chah. Luang Pó o aceitou como discípulo,
mas com uma condição, como ele mais tarde relatou:

``Naquela época, veio Tahn Sumedho, um ocidental. Eu nunca o tinha
visto, não o conhecia; apenas tinha ouvido dizer que o país dele era
muito divertido, mas nunca tinha visto. Pensei `agora ele veio ficar
conosco' e fiquei preocupado, apreensivo. Por que apreensivo? Porque ele
sempre havia vivido no conforto, como iria conseguir aguentar as
dificuldades daqui? Ele respondeu: `Eu consigo.' Eu então estabeleci uma
condição: `Não vou fazer esforço algum para te agradar. Na sua terra
você comia só do bom e do melhor; agora que veio morar na floresta, não
vou te mimar a esse ponto. Por quê? Porque eu já renunciei à vida leiga,
meu manto, minha comida, minha moradia dependem das doações feitas por
outras pessoas. Vivo de acordo com o que oferecerem. Eu mesmo não
consigo as coisas que desejo, e por isso não vou lhe fazer agrados. Você
pode vir viver comigo, vai ser um pouco difícil, mas não vou fazer
esforço em lhe agradar. Por quê? Por medo de que você fique burro. Você
veio para a Tailândia para quê? Veio estudar o \emph{Buddha Sāsanā},
estudar a cultura tailandesa. Na Tailândia, como eles moram? Como eles
comem? Como eles fazem as coisas? Você deveria conhecer tudo isso. Se eu
ficar lhe agradando de todas as formas, você só vai ficar mais burro,
vai ficar comendo pão\footnote{Originalmente na Tailândia não havia
  cultura de trigo e pão, na época, era considerado comida de gente
  rica.} para o resto da vida. O que vai ganhar com isso? Não vai saber
como fazem as coisas na Tailândia e isso será causa para que seja
ignorante dessa forma.' No final ele conseguiu ficar, mas passou por
muitas dificuldades.''

Após a chegada de Ajahn Sumedho, mais e mais estrangeiros começaram a
vir -- alguns querendo ordenar-se como monges, outros apenas querendo
encontrar Luang Pó e ouvir seus ensinamentos. Uma pergunta frequente que
Luang Pó ouvia era: ``Como você consegue ensinar a ocidentais, se não
sabe falar inglês e eles não falam tailandês?'', à qual Luang Pó
respondia com uma comparação: ``Água quente, \emph{nam rón} e \emph{nam
hón}\footnote{``Água quente'' em tailandês e laociano, respectivamente.}
são apenas nomes. Se você colocar a mão dentro, não vai precisar de
língua alguma para saber (que queima). Não importa de que país venha,
você vai saber.'' Ou então ele devolvia a pergunta: ``Na sua casa você
cria búfalos? Você tem gado, cães ou galinhas? Diga-me, você fala
`bufalês' ou `cachorrês'? Bom, então como consegue se comunicar com
eles?''

Mas a verdade é que ele era muito hábil em encontrar maneiras para
ensinar seus discípulos, às vezes utilizando poucas palavras ou mesmo
sem palavra alguma. Um monge britânico uma vez falou a respeito: ``Luang
Pó não tinha que ensinar muito, ele falava apenas duas ou três palavras
para que fôssemos refletir. Por exemplo, uma vez eu estava varrendo o
monastério, ele passou por perto e perguntou:

`Tudo bem?'

`Sim, tubo bem, Luang Pó.'

`Tudo bem?'

`Tudo bem.'

`Tudo bem não é bom.' Ele disse só isso e foi embora. Eu fiquei de pé,
confuso, nunca havia ouvido alguém dizer que `tudo bem não é bom'.
Pratiquei meditação andando por um bom tempo até que consegui entender:
naquela época eu havia acabado de me ordenar e ainda havia muitas
\emph{kilesas} em mim. Se estava `tudo bem', deveria ser porque eu
estava obedecendo às \emph{kilesas}, o que é algo ruim. Se estivesse
praticando de verdade, as \emph{kilesas} estariam criando oposição e com
certeza não estaria `tudo bem'\thinspace ''.

Ajahn Sumedho também tem várias histórias semelhantes: ``Uma manhã eu
estava varrendo as folhas no monastério e estava de mau humor, irritado
e com raiva. Estava reclamando comigo mesmo que vivendo em Wat Nong Pah
Pong eu não obtinha nada a não ser sofrimento. Bem nessa hora, Luang Pó
passou por perto, sorriu e me disse: `Wat Nong Pah Pong é muito
sofrimento!' e foi embora. Eu fiquei tentando adivinhar porque ele havia
dito aquilo. Quando voltei para minha cabana e pensei a respeito,
compreendi que esse sofrimento não vinha de Wat Nong Pah Pong, mas sim
de minha própria mente. Perceber isso me ajudou a parar de ficar achando
defeitos nas coisas e pessoas ao meu redor. Luang Pó falou apenas
algumas palavras, mas elas tinham o poder de nos fazer olhar de volta
para nós mesmos.

\ldots{} Luang Pó impunha uma regra de que ao ensinar o Dhamma, seus
discípulos estavam proibidos de fazê-lo tendo antes pensado num assunto
para falar ou memorizado um discurso. Você deveria falar sobre o que
estava em seu coração naquele momento. Ele ensinava a não esperar nada
ao dar ensinamentos, não falar com intenção de ganhar o respeito alheio
ou agradar a alguém. Tudo isso tinha o propósito de evitar que vaidade
se infiltrasse no ato de ensinar.

Uma vez eu o desobedeci e me preparei anteriormente em detalhes,
pensando em todo tipo de coisas boas para falar a respeito. Quando
terminei meu discurso, estava me sentindo muito satisfeito comigo mesmo
por ter falado tão bem, mas assim que desci do \emph{thammat} e prestei
reverência a Luang Pó, ele estava olhando para mim com uma expressão
séria e disse apenas: `Ruim. Não faça isso novamente.'

\ldots{} Todo dia, quando Luang Pó voltava de \emph{pindapāta}, havia
vários monges esperando por ele em frente ao salão principal para
lavar-lhe os pés. Quando eu era recém-chegado em Wat Nong Pah Pong e vi
aquela atividade, critiquei os monges em minha mente porque pensava que
apenas um ou dois monges seriam suficientes para realizar aquela tarefa,
não havia necessidade de tantos monges.\footnote{O motivo para haver
  tantos monges é porque eles entendiam que realizar essa tarefa era um
  ato meritório e, por isso, todos queriam tomar parte.} Conforme o
tempo passou, comecei a mudar e certa manhã, antes que me desse conta,
lá estava eu (esperando para lavar os pés de Luang Pó) antes de todos os
outros monges. Enquanto estava curvado ao chão, lavando-lhe os pés, ouvi
o som suave e refrescante da voz de Luang Pó rindo e dizendo sobre minha
cabeça: `Se rende, Sumedho?'\thinspace ''\footnote{Tradução alternativa: `Aceita (as
  coisas como são), Sumedho?'}

Um monge australiano também tem uma história interessante para contar:
``Certo dia fui em \emph{pindapāta} e na volta caminhei acompanhado de
outro monge. Ele era recém-ordenado e começou a achar defeito nos demais
monges do monastério -- um não fazia o que ele queria, o outro não dizia
o que ele queria ouvir, e assim por diante -- ele estava reclamando. Me
lembro de pensar que não queria ouvir reclamações sobre bons monges e
passei mais à frente deixando-o para trás, mas continuei pensando: `Ele
não deveria criticar bons monges. Por que está criticando bons monges?'
-- no final eu mesmo estava reclamando dele.

Enquanto pensava `ele não deveria fazer isso, não deveria fazer aquilo',
passei pelo portão de entrada do monastério. Na minha cabeça eu ainda
estava carregando aquela conversa que já havia acabado dez minutos
antes, quando ouvi alguém dizer em inglês claro: `Good Morning!'
Levantei a cabeça e vi que Ajahn Chah estava de pé a um metro de mim com
um grande sorriso no rosto. Era a primeira vez que eu o ouvia falar
inglês, então sorrindo pus minhas mãos em \emph{añjali} e respondi `Good
morning Luang Pó!', e ele sorriu.

É claro, meu humor mudou, esqueci por completo o que aquele monge havia
dito e fiquei feliz pelo resto do dia. Voltei, tomei a refeição, passei
o dia praticando meditação sentado e andando em minha cabana. Quando
chegou a noite, decidi visitar Luang Pó, porque quando era leigo eu
havia aprendido shiatsu e de vez em quando ia à cabana dele e lhe
oferecia uma massagem nos pés, porque ele sentia que era útil. Naquele
dia, por Ajahn Chah ter me dito `good morning', eu decidi ir lhe
oferecer uma massagem nos~pés.

Fui à sua cabana e havia cerca de dez ou quinze monges lá, conversando
sobre Dhamma e ouvindo ele explicar diferentes aspectos. Ele estava
sentado numa cadeira de bambu e eu estava sentado ao chão em frente a
ele e comecei a fazer a massagem enquanto ele conversava com os demais
monges. Quando o sino para a \emph{pūja} vespertina soou, ele disse aos
monges que fossem se juntar à recitação, mas me disse: `Você fica.'

Era a primeira vez que estava sozinho com ele. Não havia velas, mas a
lua estava subindo e era uma dessas noites belas e luminosas, a
temperatura estava perfeita, não havia mosquitos e eu estava sentado com
Ajahn Chah. Ele estava sentado de olhos fechados, como se meditasse,
enquanto eu massageava seus pés. Bem nessa hora os monges começaram a
recitar no salão e eu senti como se estivesse num paraíso celestial. Lá
estava eu, ouvindo monges recitarem elogios ao Buddha, Dhamma e Sangha,
pagando meu débito de gratidão para com meu professor, como o Buddha
disse que deveríamos fazer, e ele estava sentado pacificamente -- era
minha chance de fazer muito mérito!

Nesse exato momento em que sentia que minha mente estava num paraíso
celestial, ele me chutou no peito tão forte que caí de costas e bati
minha cabeça no chão. Ele apontou para mim e disse: `Está vendo? Um
monge diz algo que não lhe agrada, então outro lhe diz `good morning' e
você fica feliz o dia inteiro. Não se deixe levar pelas palavras dos
outros. Vigie sua mente!', e então ele me deu um ensinamento sobre não
tomar deleite ou se deixar levar pelas palavras alheias, não importando
se eram elogios ou críticas, se eu gostasse delas ou não. Ou, como ele
diria: se gostamos, está bom; se não gostamos, está bom. Se é bom, está
bom; se é ruim, está bom. Eu fiquei impressionado e permaneci sentado
com minhas mãos em \emph{añjali} enquanto ele me dava aquele longo
ensinamento. Mas a coisa que mais me impressionou era que ele me
conhecia melhor do que eu conhecia a mim mesmo!''

Luang Pó Chah não se deixava intimidar pelo hábito ocidental de ser
inquisitivo, que possui muitos aspectos positivos, mas que também pode
acabar levando a pessoa a se perder no mundo das teorias e complicações
filosóficas, que por si só podem se tornar um obstáculo ao progresso no
Dhamma. Isto é o que ele dizia a respeito: ``Eu vejo esses praticantes
como crianças de seis anos que não sabem nada. Quando veem uma galinha
eles perguntam: `Pai, o que é aquilo?'; veem um pato e: `Pai, o que é
aquilo?'; veem um porco e de novo: `Pai, o que é aquilo?' No final o pai
se cansa de responder. Se ele não parar de responder, a criança vai
continuar perguntando, porque ela nunca viu aquelas coisas antes. Após
um período o pai só resmunga `Hum\ldots{} Hum\ldots{}' Ele vai morrer de
exaustão se for responder a todas as perguntas. Mas a criança nunca se
cansa; não importa o que veja, logo pergunta: `O que é isso, o que é
aquilo?' Nunca acaba. Mas quando ela cresce esse problema desaparece,
porque ela cresceu. Treinamos através de contemplação até sabermos o que
é o que, então conseguimos responder nossas próprias questões.
É~assim.'' Devido à sua maneira simples e direta de ensinar, muitas
pessoas sugeriam que Ajahn Chah era como um mestre Zen, mas ele não
aceitava tal comparação: ``Não, eu não sou. Eu sou como Ajahn~Chah.''

Um exemplo disso era o monge ocidental que tinha dificuldades em
escolher um objeto de meditação apropriado para si porque já havia
experimentado recitar ``Buddho'' e a prática de \emph{ānāpānasati} por
um longo período e sua mente ainda não havia alcançado \emph{samādhi}.
Ele tentou a contemplação da morte, mas não obteve resultados. Ele
tentou contemplar os cinco \emph{khandhas} sem sucesso e não sabia o que
fazer em seguida. Luang Pó respondeu simplesmente: ``Largue, se você não
sabe mais o que fazer, apenas largue!''

Uma vez um monge coreano veio visitar Wat Nong Pah Pong por alguns dias
e ele gostava de propor \emph{koans}\footnote{Uma frase enigmática
  utilizada como objeto de meditação na tradição Zen/Chan.} a Ajahn
Chah. Luang Pó, por sua vez, não conseguia enxergar o propósito -- ele
pensava que eram piadas. Era fácil perceber que era necessário conhecer
as regras do jogo para poder dar as respostas corretas. Um dia esse
monge contou para Ajahn Chah a história Zen sobre a bandeira e o
vento\footnote{Ao ver uma bandeira balançando ao vento, dois monges
  brigavam, pois um dizia que era o vento que se movia enquanto o outro
  dizia que era a bandeira. O mestre Zen passa perto e diz `É a mente
  que se move.'} e perguntou `É a bandeira que sopra ou o vento?' Ajahn
Chah respondeu: `Nenhum dos dois, é a mente.' O monge coreano ficou
muito impressionado e imediatamente se prostrou perante Ajahn Chah, mas
Luang Pó achou melhor trazê-lo de volta ao mundo da realidade e informou
que sabia a resposta porque havia acabado de ler essa mesma história
numa tradução tailandesa dos ensinamentos de Hui Neng.

Além de lidar com monges estrangeiros, Luang Pó também tinha muito
contato com leigos ocidentais. Naquela época, uma mulher dava palestras e escrevia livros bastante populares sobre o
Abhidhamma. Uma vez ela foi visitar Wat Nong Pah Pong e queria conhecer
Ajahn Chah. Aparentemente achou que seria uma boa ideia dar a ele uma
longa palestra sobre quão importante era o estudo do Abhidhamma e por
que ele era uma ferramenta excelente para ensinar o Dhamma. Ao final
perguntou se ele concordava com a importância desse estudo, e a resposta
foi simples:

``Sim, muito importante.''

Feliz em ouvir isso, ela então perguntou se ele também encorajava seus
estudantes a aprender o Abhidhamma, e ele respondeu:

``Sim, claro.''

E onde, ela perguntou, ele recomendava que eles começassem, quais livros
e quais estudos eram os mais apropriados?

``Bem aqui,'' ele respondeu apontando para o centro de seu peito, ``bem
aqui.''

Na Tailândia, ocidentais são famosos por seu hábito de reclamar e achar
defeito em tudo. Um monge ocidental tinha o hábito de criticar os demais
monges do monastério, seus professores e o local em si, dizendo que não
era suficientemente conducente à prática de meditação. Sua mente era tão
obcecada com negatividade que aquilo impregnava tudo que via. Luang Pó
lhe repreendeu com força, mas também com bom humor, dizendo: ``Você é
estranho, gosta de colocar cocô dentro da sua sacola e o carrega consigo
aonde quer que vá; depois reclama que o lugar está fedendo. Aqui fede,
ali fede, aonde quer que vá só há fedor de fezes. Você só reclama, mas
por que não experimenta olhar na sua sacola e ver o que há dentro?''

Outro ponto de ``choque cultural'' com discípulos ocidentais era o
hábito muito difundido na região de fumar e mascar noz de bétel. Naquela
época, cigarros industrializados eram um fenômeno recente na Tailândia,
o que significava que uma pessoa comum fumava muito menos que nos dias
atuais. Uma pessoa normalmente fumava três ou quatro cigarros por dia,
enquanto hoje em dia um fumante comum facilmente chega a consumir um
maço inteiro (20 cigarros) ou mais diariamente. Por essa razão, e também
graças à falta de informações sobre o assunto, na Tailândia rural o
hábito de fumar não era ainda associado a problemas de saúde ou visto
como uma válvula de escape para o estresse. No que diz respeito à noz de
bétel, além de ser um estimulante, é também um ótimo bactericida. Como o
tratamento dental moderno ainda não estava disponível, a percepção geral
das pessoas era de que mascar a noz regularmente era bom para os dentes,
pois matava bactérias e diminuía a incidência de cáries (mas ela também
possui efeitos indesejáveis como tingir os dentes de vermelho e, ao
final de muitos anos de uso, deixá-los completamente negros).

Conforme mais e mais ocidentais chegavam a Wat Nong Pah Pong, eles
traziam consigo a percepção ocidental de cigarros associados a vício,
fraqueza emocional e danos à saúde. Eles provavelmente pensavam: ``Como
pode uma pessoa ensinar a abrir mão das coisas, se ela mesma continua
apegada a algo tão nocivo como tabaco?'' Esse novo ponto de vista sobre
o hábito de fumar também já começava a emergir entre os tailandeses,
especialmente entre aqueles que tinham acesso à cultura ocidental ou
viviam em centros urbanos. Luang Pó também refletiu e concluiu que fumar
não era uma necessidade real para monges e servia apenas como fardo
extra para os leigos, que se viam obrigados a gastar dinheiro comprando
cigarros para oferecer aos monges.

Com tudo isso em mente, uma nova regra foi estabelecida em Wat Nong Pah
Pong e todas suas filiais proibindo os monges de mascar bétel e fumar
tabaco. Na reunião da sangha em que a regra foi criada, os membros da
comunidade convidaram Luang Pó a continuar fumando enquanto o resto do
grupo abandonaria o hábito, mas ele prontamente recusou a oferta e
também passou a obedecer à nova regra. Atualmente, apesar de haver uma
lei imposta pela Autoridade Budista da Tailândia proibindo o uso de
tabaco em todos os monastérios e templos do país, é seguro dizer que só
mesmo em Wat Nong Pah Pong e filiais essa regra é de fato respeitada à
risca.

Mais um ponto de conflito era o vegetarianismo. Desde a época em que o
Buddha era vivo, comer ou não comida vegetariana era um debate que
frequentemente levava a brigas entre os leigos e também entre os
monásticos. Enquanto algumas pessoas defendem que comer carne é um ato
de maldade, que é encorajar a matança de animais, outros argumentam que
o próprio Buddha comia carne e explicitamente recusou-se mais de uma vez
a impor regras obrigando os demais a consumirem somente comida
vegetariana. Mesmo Luang Pó Chah mais de uma vez foi acusado de não ter
compaixão por não ser vegetariano, e isso é o que ele dizia a respeito:

``É como sapo e rã. Qual vocês dizem que é melhor, o sapo ou a rã? Na
verdade o Buddha não comia nada, não era nada, na mente dele não havia
nada, já não era mais nada. Consumir alimentos é apenas uma forma de
manter o corpo para que ele possa funcionar. O Buddha dizia para não se
apegar ao gosto da comida, não se apegar a nada. Ele dizia para ter
moderação ao se alimentar, não comer por ganância. Isso é o que
significa dizer que o Buddha não comia nada, não tinha nada, já não era
mais nada.

Se uma pessoa come carne e se apega ao gosto da carne, isso é ganância.
Se uma pessoa não come carne, mas assim que vê alguém que come, fica com
raiva e ódio, vai acusá-lo e criticá-lo, é como se pegasse a maldade do
outro e colocasse dentro de seu próprio coração. Isso é ser ainda mais
tolo do que o outro. Também é agir sob o poder da ganância. Esse estado
de raiva e ódio da outra pessoa é como um fantasma que estava escondido
no nosso coração. A outra pessoa come carne e isso é maldade, mas então
ficamos com raiva dela e viramos fantasmas -- também é maldade. Ambos
ainda são animais, ainda não são Dhamma. Por isso digo que é como sapo e
rã. O correto é comerem o que quiserem, mas tenham Dhamma dentro de si.

Quem come carne não deve se preocupar apenas com seu estômago; não tome
apenas o gosto da comida como referência na hora de escolher, não mate
para comer. Já quem come comida vegetariana, não se obceque com essa sua
prática, não fique com ódio quando vir alguém comendo carne, tome conta
de si mesmo, não se apegue a práticas exteriorizadas.

Com os monges e noviços em meu monastério é a mesma coisa. Quem quiser
come comida vegetariana; quem quiser come comida normal de acordo com o
que estiver disponível. Mas não briguem! Não olhem uns aos outros com
maus olhos. É assim que ensino e eles conseguem viver juntos, não vejo
surgir problema algum. Entendam que alcançamos o verdadeiro Dhamma
através de sabedoria. O caminho de prática correto é \emph{sīla,
samādhi} e \emph{paññā}. Se guardarmos bem as portas dos sentidos, ou
seja, os olhos, os ouvidos, o nariz, a língua, o corpo e o coração, a
mente se pacifica e a sabedoria, o saber capaz de conhecer todos os
fenômenos condicionados, surge. A mente se desencanta com todos os
objetos de desejo e então a liberdade (\emph{vimutti)} se manifesta.''

Mais uma história interessante desse período ocorreu em 1973, quando
Luang Pó recebeu do rei tailandês o título de ``Chao Khun Prah Bodhiñāna
Thera''. Com isso, era esperado que comparecesse a um almoço
tradicional, oferecido no palácio real, acompanhado dos demais monges
ilustres que receberiam homenagens similares naquela ocasião. Quando
chegou à porta do palácio e desceu do carro, encontrou-se com um
eminente Chao Khun que chegava para participar do mesmo evento. O Chao
Khun notou que Luang Pó havia trazido consigo sua tigela de
esmolas\footnote{Uma das práticas ascéticas (dhutanga) observadas pelos
  monges de Wat Nong Pah Pong é só comer utilizando a tigela de esmolas,
  o que significa que toda comida, incluindo doces e sobremesas, será
  posta e misturada num mesmo recipiente. É uma prática que visa ajudar
  a fomentar a humildade e abandonar o apego ao gosto da comida.} e
perguntou num tom de chacota: ``Tahn Chah, você não tem vergonha de
aparecer perante o rei com sua tigela de esmolas?'' Ao que Luang Pó
respondeu: ``Tahn Chao Khun, você não tem vergonha de aparecer perante o
Buddha sem a sua?''
