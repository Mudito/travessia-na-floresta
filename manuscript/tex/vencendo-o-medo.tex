\chapter{Vencendo o medo de fantasmas}

Na manhã seguinte, Luang Pó e seus companheiros andaram até Wat Prong
Klong, onde vivia Ajahn Kamdi, e pediram permissão para ficar por algum
tempo. Estavam na estação seca e o chão já não estava mais úmido e,
portanto, era uma boa época para viver ao ar livre, ao pé de uma árvore,
por exemplo. Antigamente na Tailândia as pessoas simplesmente levavam os
mortos a uma área afastada na floresta e queimavam ou enterravam o corpo
ali. Não havia cemitérios como os normalmente encontrados no ocidente.
Alguns monges do monastério haviam ido viver num cemitério de floresta
como este em busca de reclusão e Luang Pó sentiu interesse em
experimentar essa prática. Ele queria tentar porque pensou que, se não o
fizesse, jamais saberia como era e quais benefícios trazia para a
prática de meditação. Ao mesmo tempo, havia outra parte dele que não
queria ir, pois desde criança sempre teve pavor constante de fantasmas.
Após muito lutar consigo mesmo, conseguiu controlar-se e enfrentar
aquele desafio. Muitos anos mais tarde ele relatou detalhes sobre o que
ocorreu:

``Chegou a tarde e eu estava com muito medo, não queria ir. Não
conseguia fazer nada, ordenava a mim mesmo a ir, mas não ia. Chamei um
velhinho, Anagārika Kéu, para ir junto. Estava indo para morrer. `Se for
para morrer, então que morra! Se vai ser tão teimoso, tão burro, então é
melhor morrer.' Era o que eu me dizia, mesmo sabendo que no fundo não
queria ir. Mas me obriguei a ir, porque se fosse esperar até que me
sentisse plenamente preparado acabaria não indo nunca, e desse jeito não
seria um treinamento. Tem que se obrigar a ir.

Cheguei ao cemitério e era minha primeira vez, nunca tinha ficado num
cemitério antes. O anagārika veio acampar perto de mim, mas não deixei,
mandei-o ir acampar lá longe. Na verdade eu queria que ele ficasse
perto, mas não queria que minha mente se apoiasse nele, achando que
tinha um amigo por perto e então não sentisse medo. Não deixei, mandei
ir para longe para que eu não pensasse que poderia contar com ele. `Se
sentir muito medo, então morra hoje à noite! Que é que tem?' Mesmo com
medo ainda fui em frente. Não pense que não tinha medo, mas eu também
tinha coragem. `O máximo que pode acontecer é eu morrer, só
isso\ldots{}'

Assim que o sol se pôs, lá veio -- que sorte, trouxeram um defunto!
Tinha que ser assim, não? Nossa! Quando caminhava quase não sentia meus
pés tocando o chão, queria fugir. Eles pediram que eu recitasse os
cânticos para o funeral, mas eu não queria saber de recitar nada para
ninguém e fui para longe deles. Passou um pouco e voltei: eles
enterraram o defunto bem ao lado de onde eu estava acampado, pegaram o
bambu que usaram para carregar o cadáver e construíram uma plataforma
para que me sentasse. E agora, que fazer? A distância entre o cemitério
e o vilarejo era de mais ou menos dois ou três quilômetros, não tinha
outra opção a não ser morrer! E aí, o que fazer? Tem que deixar morrer,
ora!

O anagārika veio pedir para ficar perto mas mandei ele embora. `Que eu
morra! Para que tanto medo? Assim é mais divertido.' Se não tiver
coragem de fazer, nunca vai saber como é. Não era brincadeira, quando
andava quase não sentia o chão. Foi escurecendo\ldots{} `E agora, para
onde vou fugir? Estou no meio de um cemitério numa floresta
afastada\ldots{} Ora essa, morra! Você nasceu para morrer, não foi?'
Lutava comigo desse jeito.

Quando chegou a noite, uma sensação me disse para ficar dentro do
\emph{glot.} Não conseguia mover a perna para caminhar. Minha sensação
era de querer ficar no \emph{glot}, mas eu continuava praticando
meditação andando entre o \emph{glot} e o local onde enterraram o
cadáver. Quando caminhava em direção ao \emph{glot} não era tão ruim,
mas quando dava meia-volta e começava a caminhar\ldots{} não sei
explicar, é como se algo me puxasse para trás, uma sensação fria\ldots{}
É assim que se treina. Quando o medo ficou muito forte, eu não conseguia
mais mexer a perna e então parava. Quando passava o medo, eu continuava
a caminhar. Quando já estava bastante escuro, entrei no \emph{glot} e
senti um grande alívio no peito. Me sentia como que protegido por uma
parede de várias camadas. Olhava minha tigela de esmolas e era como se
fosse minha amiga. É possível uma tigela virar nossa amiga -- se a
pessoa não tem amigos, então pensa na tigela como uma amiga. Olhava para
a tigela e sentia felicidade: a mente não tinha onde se apoiar, então se
apoiava na tigela. É em situações como essa que enxergamos como é nossa
mente.

Fiquei ali sentado dentro do \emph{glot} esperando os fantasmas a noite
inteira, até amanhecer. Não dormi nem um instante por causa do medo.
Tinha tanto medo como coragem de treinar, coragem de fazer. Sentei firme
a noite inteira, não senti sono -- o sono também estava com medo de
fantasmas. Sentei daquele jeito a noite inteira. Quem teria coragem de
fazer isso? Na prática do Dhamma, se for para praticar seguindo suas
vontades, quem iria praticar? Eu estava com medo àquele ponto, mas em
qualquer coisa neste mundo, se não fizermos, não surge benefício já que
deixamos de fazer. Prática é assim.

Quando amanheceu: `Ufa!', estava feliz, não tinha morrido, me senti
muito aliviado. Queria que só houvesse dia, não queria que houvesse
noite. Minha vontade era matar a noite, queria que só houvesse dia. Que
alívio: `Ufa! Dessa vez não morri!' Durante o dia descansei um pouco,
estava feliz por ter cerca de 50\% de vitória. Pensei: `Não tem nada de
mais, é só medo.' Achei que na segunda noite ia conseguir praticar sem
problema, porque já tinha passado o pior: `Hoje à noite não deve ter
nada de mais.'

Eu estava disposto a experienciar até cachorros: quando saí em
\emph{pindapāta}, um cachorro me seguiu e ameaçava me morder. Não o
afugentei, deixei ele morder, deixava tudo ir às últimas consequências.
Ele mordia de novo e de novo, às vezes ele acertava e eu sentia dor.
Pensava que tinha rompido meu tendão. As mulheres do vilarejo não
queriam ajudar a pegar o cachorro, porque achavam que um fantasma devia
ter vindo (do cemitério) junto com o monge e o cachorro estava latindo
para afastar o fantasma; então deixavam, não afugentavam o cachorro. Eu
deixava morder. Na noite anterior quase morri de medo e de manhã um
cachorro vem me morder, pois então que morda, talvez numa vida passada
eu também o tenha mordido. Mas a mordida dele às vezes acertava, às
vezes errava. É assim que se treina a si mesmo.

Voltei de \emph{pindapāta} e comi. Quando terminei me senti feliz -- o
sol saiu e me senti acalentado. Descansei e pratiquei meditação andando.
`Acho que hoje à noite vai ser boa minha prática, já passei no teste
ontem à noite, não deve ter nada de mais.' Foi só cair a tarde e lá vêm
eles de novo trazendo mais um defunto, e agora era de um adulto. Desta
vez foi ainda pior, vieram cremar bem em frente ao local onde eu estava
acampado. Ôe! Ainda pior do que a noite anterior! Bom, eles vieram
cremar e me convidaram para ir contemplar o cadáver,\footnote{As
  cremações eram feitas a céu aberto usando lenha. O cadáver ficava
  exposto durante todo o processo, o que dava a oportunidade aos
  presentes de observar e refletir sobre a morte.} mas não fui, só fui
depois que eles foram embora. Foram embora e deixaram o cadáver
queimando para eu ficar olhando sozinho. Não sabia o que fazer, não sei
com o que poderia comparar aquela sensação de medo -- ainda mais no meio
da noite! O fogo queimava verde e vermelho e estalava, as chamas subiam
e desciam. Não conseguia praticar meditação andando em frente à fogueira
e assim que escureceu entrei no meu \emph{glot}, como antes.

Passei a noite inteira naquele cemitério de floresta abandonado, fedendo
à fumaça de cremação. Foi ainda pior do que na noite anterior. O fogo
estalava, sentei de costas para a fogueira e nem considerei me deitar,
não sabia como seria possível dormir, a ideia de dormir não me ocorria;
passei a noite inteira completamente desperto -- estava com medo. Tinha
medo e não sabia a quem pedir ajuda. Só tinha eu ali, então só podia
contar comigo mesmo. Não tinha para onde fugir. Pensava em ir embora,
mas não tinha para onde fugir, porque a noite estava absolutamente
escura. O jeito era morrer sentado ali mesmo, para onde ir? Se
perguntasse para meu coração se queria praticar daquele jeito\ldots{}
nunca! Se dependesse dele, faria isso para quê? E quem nesse mundo
alguma vez já pensou em atormentar a si mesmo a esse ponto? Só quem tem
fé firme no ensinamento do Buddha, nos resultados da prática do Dhamma.

Me sentei de costas para o fogo sem saber o que iria acontecer. Por
volta das dez da noite ouvi um barulho vindo da fogueira atrás de mim e
achei que o cadáver tinha rolado para fora da fogueira e que alguns cães
estavam brigando para comê-lo. Mas não era isso. Continuei ouvindo
sentado e me pareceu que era um som de algo pesado se arrastando.
Pensei: `Dane-se!' Passados alguns minutos ele veio andando até mim, era
um som como de uma pessoa caminhando atrás de mim, em minha direção. Era
um passo pesado, parecia o passo de um búfalo, mas não era. Estávamos em
março e as árvores estavam trocando folhas; o chão estava coberto de
folhas secas e eu ouvia o som de passos sobre as folhas, um som alto e
pesado. Onde estava acampado tinha um cupinzeiro. Ouvi o som dos passos
dando a volta por trás do cupinzeiro e vindo em minha direção. Pensei:
`Seja lá o que for que ele vai fazer comigo, é problema dele. Já decidi,
estou disposto a morrer.'

Não tinha para onde fugir. Mas não veio mesmo até mim, continuou em
frente com passos pesados em direção a onde estava Anagārika Kéu, lá
longe. O som então cessou por causa da distância. Eu não sabia o que era
porque só havia medo, e esse medo me fez pensar em várias coisas. Após
cerca de meia hora ouvi o som dos passos voltando, vindos da direção do
Anagārika Kéu. Era realmente como o som de uma pessoa andando em minha
direção. Veio diretamente em minha direção, como se fosse me atropelar.
Decidi ficar sentado de olhos fechados: não iria abrir os olhos para ver
o que era. `Se for para morrer, vou morrer deste jeito.' Quando chegou
até onde eu estava, parou de repente e ficou em pé, em silêncio, bem em
frente a meu \emph{glot}. Minha impressão era como se mãos em chamas
estivessem tentando agarrar algo bem à minha frente. `Ôe, desta vez eu
morro\ldots{}' Meu corpo inteiro estava duro como pedra, esqueci
completamente de `buddho', `dhammo', `sangho', só havia medo em mim,
estava tenso como a pele de um tambor. Não tinha como fugir, só havia
medo. Quando penso a respeito, vejo que desde que nasci nunca houve
outra ocasião em que senti tanto medo. Não sabia mais o que era
`buddho', `dhammo', `sangho', estava tenso como a pele de um tambor.
`Bom, se você vai ficar aí, eu vou ficar aqui!' Não conseguia pensar em
nada, não sabia dizer onde estava sentado, não conseguia dizer se estava
flutuando no ar ou sentado numa cadeira -- fixei completamente minha
mente em \emph{sati}.

Estava com muito medo, então foi como quando colocamos água num jarro --
se colocar muita água, ela acaba transbordando e vaza para fora. Deve
ter sido assim porque eu estava com muito medo mesmo e então veio para
fora. Me perguntei:

`Todo esse medo, é medo do quê? Por que tanto medo?' Eu não disse isso,
meu coração sozinho foi quem falou e logo veio a resposta:

`Estou com medo da morte.' -- foi o que ele disse. Então perguntou de
novo:

`Onde mora a morte? Por que está com ainda mais medo do que as pessoas
comuns do vilarejo e da cidade?' Investiguei a morte, fui perguntando,
perguntando e achei uma resposta:

`Morte mora em mim e, sendo assim, para onde seria possível fugir e
escapar disso? Sempre que eu fugir e correr, ela correrá junto; onde me
sentar, ela estará sentada junto; se me levantar e for embora, ela se
levanta e vai junto porque morte mora em mim, não há para onde ir. Não
importa se tenho ou não medo da morte, ainda assim tenho que morrer
porque a morte mora em mim. Não é possível fugir.' -- encerrei a
conversa desse jeito. Quando as perguntas e respostas se encerraram, a
sensação, a percepção antiga mudou, se transformou. Era completamente
diferente, como a palma e as costas da mão. Fiquei maravilhado em ver
como um medo tão forte podia sumir e `não estar com medo' surgir em seu
lugar.

Meu coração subiu ao céu! Assim que consegui vencer o medo, começou a
chover. Não sei se era uma chuva celestial ou o quê, tinha raio, trovão
e vento. O barulho dos trovões era muito alto, mas eu não tinha mais
medo; árvores ao redor caíram, mas eu nem me interessei. Choveu muito
forte e todas minhas roupas e pertences ficaram encharcados. Quanto a
mim, permaneci sentado, imóvel como antes, e em seguida chorei. O choro
foi automático, as lágrimas jorravam e desciam pelo meu rosto. Chorava
porque pensava: `Pareço um órfão sem pai nem mãe. Estou aqui sentado
tremendo e tomando chuva como um indigente.' Então pensei: `Aqueles que
moram em casas confortáveis não devem imaginar que esta noite um monge
irá sentar a noite inteira sob a chuva. Eles provavelmente não pensam
nisso, devem estar dormindo tranquilos enrolados em cobertores em suas
casas, mas eu estou aqui sentado tomando chuva a noite inteira, por que
isso?' Fui pensando assim e senti tristeza pela minha vida e então
chorei, as lágrimas jorrando\ldots{} `Bom, que essas lágrimas jorrem até
acabarem, que não sobre nada!' Prática é assim, ela tem uma dinâmica
própria.

Mas não sei como expressar o que acorreu em seguida, não sei como
expressar. Eu apenas continuei sentado, mas, tendo vencido, coisas
começaram a surgir automaticamente, todo tipo de coisa para que eu visse
e conhecesse. Todo tipo de coisas, tão variadas que é impossível
descrever até o fim. Pensei no que o Buddha disse: \emph{paccattam
veditabbo viññūhi} -- os sábios saberão por si mesmos. É verdade mesmo.
Eu estava sofrendo e tomando chuva daquele jeito, quem saberia disso se
não eu? Só eu sabia, é \emph{paccattam}\footnote{Algo que só pode ser
  visto por si mesmo, experienciado por si mesmo (pāli).} desse jeito.
Senti muito medo e então vi o medo desaparecer. As pessoas comuns não
têm como saber isso, só eu o sei porque é \emph{paccattam}. Para quem
poderia dizer isso? Para quem poderia contar isso? É \emph{paccattam}.
Quanto mais contemplava, mais certeza surgia em mim, meu coração ganhava
ainda mais força, minha fé ficava ainda mais firme. Continuei
contemplando até amanhecer.

Quando amanheceu e abri os olhos, onde quer que olhasse estava banhado
de luz amarela, o mundo inteiro. O perigo havia passado. Durante a
noite, dentro do meu \emph{glot}, senti vontade de urinar, mas por causa
do medo não tive coragem de levantar, então reprimi a vontade. Após um
tempo a vontade passou. Pela manhã, quando levantei, onde quer que
olhasse estava banhado de amarelo com o sol da manhã. Fui tentar urinar
porque estava com vontade desde a noite passada, mas só saiu sangue.
Desconfiei que algo dentro de mim devia ter rasgado. Me assustei e
pensei que certamente tinha algo errado dentro de mim, mas um pensamento
se interpôs imediatamente e disse: `Quem rasgou? Rasgou sozinho.' -- o
pensamento surgiu e dissipou a dúvida com uma tacada só -- `Deixe
rasgar, deixe morrer. Eu só estava sentado, não fiz nada. Se quiser
rasgar, que rasgue!' O pensamento dizia assim, como duas pessoas
disputando algo, um puxava para um lado e outro puxava para o outro. Uma
parte de mim me oprimia dizendo que eu estava em perigo, outra parte
revidava imediatamente. Quando urinei estava cheio de sangue e então
pensei:

`Onde encontrar remédio?'

`Não vá procurar. Vai procurar aonde? Monges não podem cavar raízes. Se
for a hora de morrer, então deixe morrer. Que fazer? Será uma boa morte.
Se for para morrer por ter praticado o Dhamma, estou 100\% disposto a
morrer. Se for morrer por ter feito algo ruim, então não vale a pena.
Morrer desse jeito é apropriado, deixe morrer!' Era assim que meu
pensamento dizia.

Havia chovido a noite inteira e quando amanheceu tive febre. Meu corpo
inteiro tremia. Aí fui até o vilarejo recolher \emph{pindapāta,} mas só
me deram arroz puro. Enquanto caminhava de volta notei um velhinho
carregando duas ou três vagens e uma garrafa de molho de peixe andando
atrás de mim. Pensei comigo mesmo: `Será que esse leigo vai cozinhar as
lentilhas e me oferecer?' Fui pensando assim até que o leigo começou a
cozinhar. Eu mesmo não sabia se ia comer ou não, porque achei que se
comesse a febre ia aumentar, pois sou alérgico. Ele continuou cozinhando
e eu continuava pensando sozinho: `Como ou não como?', porque não tinha
mais nada além de arroz, estava na floresta, tinha só arroz puro, não
tinha mistura\ldots{} Ele continuava cozinhando e eu continuava
pensando. Ainda não tinha certeza se ele iria oferecer ou não, mas
continuava pensando. Quando terminou, ele trouxe a comida para oferecer
e eu recebi.

Recebi e coloquei dentro da tigela, mas ainda não tive coragem de comer,
só fiquei ali, refletindo. Pensei: `Sabendo que vai piorar minha febre,
se comer vai ser por pura ganância. Se não for isso, então não sei o que
é\ldots{} Não seria melhor comer arroz puro?' Eu olhava os pensamentos
competindo e no final veio a decisão: `Que seja ganância, mesmo que
houvessem outras opções eu provavelmente ainda comeria, mas aqui só tem
uma, então vou comer. O que pode acontecer? Se a febre piorar o que vou
fazer? Mesmo que piore não deve chegar a me matar. Ou alguém vai ter que
vir me socorrer ou vou ter que vomitar a comida. Se morrer, não tem
problema. Se morrer, não há como alguém vir ajudar -- morro de vez.'
Após conseguir me decidir, comi. Refleti até não ter mais dúvida e então
comi. Terminado, recitei uma benção para o leigo e ele foi embora.

Por volta do meio-dia me lembrei das vagens. Sentia minha cabeça girar,
meus pelos se arrepiavam, era como se fosse ter febre. Eu realmente sou
alérgico àquelas vagens. `Bom, venha o que vier. Se não tiver quem venha
me ajudar vou ter que vomitar, isso se não morrer\ldots{}' Fui
empurrando, empurrando e por volta da uma da tarde o vômito veio de
verdade. Quando chega a hora, tem que vomitar mesmo, se não fosse assim
ia ter que vir alguém me socorrer. Eu pensava dessa forma e deixava as
coisas acontecerem.''

Depois de passar sete dias praticando naquele cemitério, Luang Pó
ficou muito doente e teve que voltar ao monastério de Ajahn Kamdi para
descansar e recuperar-se. Permaneceu lá por cerca de dez dias até que
cessasse a febre, e então se despediu de Ajahn Kamdi e foi praticar à
beira de um bosque perto de Ban Tóng por vários dias. Em seguida
caminhou em direção de Ban Nong Hi, em Nakhon Panom, com a intenção de
visitar Wat Pah Metta Vivek, onde morava Ajahn Kinari Candyio.
