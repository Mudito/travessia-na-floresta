\chapter{Viagem à região central}

Em 1950 Luang Pó recebeu uma carta de Tahn Mahā Bun Mi, um velho amigo
com quem costumava praticar meditação. Ele mandava notícias de Luang Pó
Sot, abade de Wat Pak Nam Passit Jaroen, em Bangkok\footnote{Luang Pó
  Sot foi quem deu origem à técnica de meditação ``Dhammakaya''. Apesar
  disso, muitos argumentam que ele não é o responsável pelo polêmico
  movimento Dhammakaya existente hoje, pois foi algo que se desenvolveu
  após sua morte.}, e Ajahn Chah resolveu viajar até lá para
experimentar a técnica de meditação Dhammakaya por cerca de sete dias.
Ao final decidiu que não era um bom método para sua prática e partiu em
direção a Ayutthaya, onde se hospedou em Wat Yai Chai Mongkhon. Entre
1950 e 1951, seu décimo segundo ano como monge, Luang Pó residiu em Wat
Yai Chai Mongkhon.

Em 1951, durante sua estadia em Wat Yai Chai Mongkhon, Luang Pó sofreu
de uma doença do sistema digestivo que provocou inchaço no lado esquerdo
do abdômen e muita dor. Além disso, um antigo problema de asma voltou
com força total, causando-lhe muito sofrimento. Ele refletiu que, uma
vez que estava longe de sua família e amigos, seria um fardo para as
pessoas do monastério levá-lo ao hospital e pagar tratamentos, portanto
decidiu curar a si mesmo usando um medicamento do Dhamma: jejum. Ele fez
jejum completo, consumindo apenas água e renunciando a todas suas
preocupações ligadas ao corpo. Ele também parou de dormir e apenas
alternava entre as posturas de pé e sentada, recusando-se a deitar.
Luang Pó tomou refúgio no Dhamma, decidiu morrer praticando o Dhamma.
Essa resolução era firme e mesmo ele ficou surpreso em ver como a mente
de uma pessoa que está resoluta em lutar com a morte se transforma,
ganhando força e poder nunca antes experienciados. A mente dele não
manifestava o mínimo sinal de fraquejo perante a morte.

Durante as manhãs, enquanto os demais monges saíam em \emph{pindapāta},
ele praticava meditação andando; e quando eles retornavam, ele voltava
para sua cabana e continuava praticando meditação sentado. Seu corpo
estava muito enfraquecido, mas sua mente estava firme e forte, não havia
medo da morte ou de nenhuma outra coisa neste mundo. Após oito dias em
jejum, um de seus amigos lhe implorou que encerrasse a prática e
voltasse a se alimentar. Uma vez que os sintomas já haviam regredido,
tanto do estômago como da asma, ele decidiu voltar a se alimentar uma
vez por dia, como normalmente fazia.

Durante sua estadia em Wat Yai Chai Mongkhon, Luang Pó interrompeu todas
suas atividades ensinando o Dhamma e focou exclusivamente em desenvolver
\emph{samādhi.} Ao fim do \emph{vassa} partiu para a ilha de Si Chang em
busca de reclusão. Ele experienciou muita paz em seu coração durante a
estadia de um mês no local, e o seguinte pensamento lhe ocorreu: ``Os
habitantes de Si Chang moram nesta ilha cercada de água. Para que esta
terra possa existir, ela tem que estar bem acima do nível d'água. A ilha
de Si Chang é um apoio exterior para o corpo, mas eu vim até aqui para
buscar uma ilha em meu coração, uma ilha que as \emph{kilesas} não
conseguiam inundar. Mesmo estando na ilha de Si Chang, ainda tenho que
continuar procurando uma ilha dentro do meu coração para que tenha uma
moradia dentro de mim. Essa ilha no coração está cercada pelo mar de
\emph{kilesas} e desejos. Aqueles que ainda não venceram \emph{kilesas,}
ganância\emph{,} apegos e mau \emph{kamma} podem ser comparados com
pessoas jogadas ao mar: é de se esperar que, cedo ou tarde, irão se
afogar, caso contrário, ser vítimas de predadores marítimos como
tubarões, por exemplo.''

Após deixar Si Chang, Luang Pó voltou a Wat Yai Chai Mongkhon e
permaneceu lá por mais um período antes de começar a viagem de volta à
sua terra natal, Ubon Ratchatani. Quando seus parentes e amigos ouviram
a notícia de seu retorno, todos ficaram felizes e vieram visitá-lo e
prestar reverência a ele, pois já havia mais de dois anos desde que o
viram pela última vez. Nesse novo encontro, todos achavam que seu porte
e comportamento estavam ainda mais dignos e nobres. Todos o tratavam com
o mais alto respeito, e sempre que ele dava algum conselho todos ouviam
com interesse e ninguém o contradizia. Seus ensinamentos fizeram com que
eles entendessem de forma ainda mais profunda a importância de boas e
más ações.

Certa noite, um dos garotos que era apenas um noviço na época de seu
encontro anterior com Luang Pó, mas que agora já havia se ordenado
monge, pediu que ele o aceitasse como discípulo. Pediu permissão para
acompanhá-lo em sua prática do Dhamma, mas Luang Pó apenas ouviu em
silêncio, nem aceitando ou recusando o pedido, o que fez o rapaz
sentir-se inseguro e desapontado. Após um momento em silêncio, Luang Pó
perguntou:

``Por que você quer vir?''

``Eu vejo que viver aqui não vai me ajudar a melhorar, por isso quero ir
e praticar como o senhor.''

``Bom, se realmente quer vir, peça que lhe façam um mapa para Ban Pah
Tao e me espere lá.''

Após passar um período instruindo seus familiares e amigos sobre o
Dhamma, Luang Pó partiu em direção a Ban Pah Tao, parando pelo caminho
sempre que encontrava lugares reclusos para praticar meditação. Ao
chegar a Ban Pah Tao, ele se estabeleceu em Lan Hin Ték (que mais tarde
passou a se chamar Wat Tam Hin Ték) e lá passou o \emph{vassa} de 1952,
seu décimo quarto.

Naquele ano, vários monges e noviços se juntaram a ele durante o
\emph{vassa}. Luang Pó determinou uma rotina de prática bastante intensa
para o período: algumas vezes eles praticavam meditação sentados e
andando durante todo o dia e toda a noite. Luang Pó os ensinava:

``Não se apeguem demais a convenções. Dia e noite são apenas convenções
inventadas pelas pessoas no mundo; se olharmos do ponto de vista do
Dhamma mais elevado, não há dia nem noite, não há lua cheia ou
minguante. Portanto, podemos combinar uma nova convenção: que o dia seja
noite e que a noite seja dia. Se pensarmos que não há diferença entre
dia e noite, vamos poder nos dedicar à nossa prática sem nos preocupar
com horários.''

Um dia Luang Pó reparou que um discípulo regularmente tomava um
medicamento, e por isso perguntou:

``Já faz tempo que vem tomando esse remédio?''

``Sim senhor, já faz vários anos.''

``E você melhorou?''

``Ajuda a diminuir os sintomas\ldots{}''

Luang Pó permaneceu em silêncio por um instante e então disse: ``Eh!
Você vem tomando esse remédio há muito tempo e ainda não está curado.
Jogue-o fora e experimente uma nova receita de remédio: coma pouco,
durma pouco, fale pouco e pratique muita meditação sentado e andando.
Experimente, e se não melhorar, deixe morrer.''

A maioria dos moradores de Ban Pah Tao eram pobres e, apesar de terem fé
em fazer doações, não tinham muito a oferecer, uma vez que quase não
tinham suficiente para si mesmos. Por causa disso, ajudavam a Sangha
como podiam, com o pouco que era possível. Em geral ofereciam apenas
arroz, pimentas, sal, alguns vegetais frescos e às vezes bananas. Mas
Luang Pó e seus discípulos continuavam praticando sem sentirem-se
desencorajados pela escassez de alimentos. Pelo contrário, eles
utilizavam essas dificuldades como oportunidades para desenvolver
\emph{khanti pāramī}\footnote{Resiliência, a capacidade de aguentar
  situações e sensações desagradáveis (pāli).} e aprender a não se
preocupar com comida. Mas certo dia algo aconteceu que colocou em teste
a resolução deles:

Perto da cabana onde Luang Pó morava havia uma pequena represa cheia de
peixes. Quando havia muita chuva, a água transbordava e os peixes que
eram arrastados para fora da represa tinham que lutar para retornar.
Alguns eram fortes e conseguiam pular a parede de pedra da represa, mas
outros ficavam exaustos pelo esforço e permaneciam no chão, tentando
respirar na água rasa. Luang Pó via e frequentemente recolhia os peixes
e os jogava de volta para dentro da represa. Certa manhã, como era seu
hábito antes de sair em \emph{pindapāta,} Luang Pó foi verificar se
havia peixes precisando de ajuda, mas ao invés disso encontrou fileiras
de varas de pescar à beira da represa, algumas com peixes já fisgados.
Ele não podia libertá-los, porque as varas pertenciam a alguém e soltar
os peixes significaria destruir propriedade alheia. Portanto ele apenas
olhou sentindo tristeza e pensou: ``Os peixes engolem a isca, mas na
isca há o anzol. É algo triste de ver. Por causa da fome, os peixes
engolem a isca que prepararam para eles e, por mais que se debatam, não
conseguem escapar. Isso é resultado do mau \emph{kamma} dos peixes por
não terem refletido antes de comer. Nós seres humanos somos iguais -- se
comermos por gula, sem discernir ou refletir, seremos como os peixes que
se deixam enganar pela isca e ficam presos no anzol. Facilmente
estaremos em perigo.''

Ao retornar de \emph{pindapāta} naquela manhã, Luang Pó notou que os
aldeões trouxeram um prato especial para oferecer aos monges: cozido de
peixe. Ele imediatamente pensou que com certeza o cozido havia sido
feito com os peixes que viu presos às varas na represa pela manhã;
talvez alguns estivessem entre aqueles que ele ajudou a colocar de volta
n'água. Luang Pó imediatamente sentiu repulsa e perdeu a fome. Quando
ofereceram a panela com o cozido, ele a recebeu, mas não se serviu.
Embora houvesse muita pouca comida -- os únicos outros pratos
disponíveis eram pasta de peixe fermentado e alguns vegetais -- ele se
recusou a comer o cozido por medo de que os aldeões ficassem felizes
pensando terem realizado bom \emph{kamma} e, por causa disso,
continuassem pescando na represa para cozinhar e oferecer aos monges. No
final, não haveria mais peixes. Luang Pó aceitou a panela de cozido, mas
logo a passou adiante ao próximo monge da fila. O monge notou que Luang
Pó não havia se servido e decidiu também não comer. Vendo isso, os
aldeões perguntaram:

``O ajahn não vai comer cozido de peixe?''

``Não. Fico com pena deles.''

Por um momento houve um silêncio incômodo, até que um homem exclamou:
``Se fosse eu, não ia conseguir resistir!''

Desse dia em diante, os aldeões não mais perturbaram os peixes da
represa e, mais do que isso, espalharam a notícia de que os peixes
pertenciam ao monastério e que, portanto, todos deveriam ajudar a
protegê-los.

Em 1953 (seu décimo quinto ano como monge), Luang Pó e seus discípulos
continuaram praticando em Wat Tam Hin Ték, mas, com a chegada da estação
chuvosa, Luang Pó se separou do grupo e foi passar o \emph{vassa}
sozinho no topo da montanha Pu Kói, que ficava a cerca de três
quilômetros do monastério. Ele temporariamente delegou a
responsabilidade sobre os demais monges a Ajahn Úan. Pela manhã, Luang
Pó ia em \emph{pindapāta} e fazia a refeição junto com os demais monges,
mas, terminada a refeição, caminhava de volta para praticar sozinho em
Pu Kói.

Luang Pó estabeleceu uma prática bastante intensa para que os monges
observassem durante aquele \emph{vassa}. Ele determinou que todos
deveriam passar as noites em claro, praticando meditação sentados e
andando. Quando o sol raiava, eles saíam em \emph{pindapāta} -- alguns
vilarejos eram próximos, outros exigiam uma caminhada de até seis
quilômetros e, nesses casos, quando enfim retornavam ao monastério e se
alimentavam, já eram quase nove da manhã. Às dez horas, terminada a
limpeza do refeitório, todos voltavam às suas cabanas para praticar
meditação e então descansavam até às três da tarde. Nesse horário, um
sino era tocado para que os monges se reunissem para limpar as demais
áreas do monastério e realizar outras tarefas conforme fossem
necessárias. Por volta das seis, o sino era tocado novamente para chamar
os monges à \emph{pūja}\footnote{Serviço devocional composto de gestos
  de louvor como fazer prostrações, recitar cânticos, oferecer flores,
  velas, incenso, etc (pāli).} vespertina e, terminada a \emph{pūja},
começavam a novamente praticar meditação até o raiar do sol.

Durante os primeiros dois meses do retiro, os monges estavam autorizados
a escolher como queriam praticar: podiam alternar entre meditação
sentada e andando quando quisessem. Mas, quando o terceiro mês chegou,
Luang Pó mudou a regra, que agora exigia que os monges praticassem a
noite inteira numa só postura, ou seja, quem quisesse praticar sentado,
teria que permanecer sentado até o amanhecer; quem quisesse praticar
meditação andando, teria que permanecer caminhando até o sol raiar. Eles
não podiam mudar de postura durante a noite. Luang Pó também aumentou a
intensidade de sua prática durante aquele período. Quando chegava o dia
do \emph{uposatha}\footnote{Dia em que os monges se encontram para
  confessar suas transgressões e recitar a regra monástica. Também é o
  nome dado ao templo onde essa cerimônia é realizada (pāli).}\emph{,}
ele descia da montanha para dar ensinamentos aos monges, noviços e
leigos; para os demais dias, eles eram instruídos a simplesmente
continuarem praticando como ele havia determinado.

Durante sua estadia em Pu Kói, Luang Pó sofreu com um problema de saúde:
suas gengivas começaram a inchar e causar uma dor indescritível. Ele
curou a si mesmo através da prática do Dhamma, utilizando \emph{khanti
pāramī} como fundação para que a mente se pacificasse e contemplando que
adoecer é natural a todos os seres. Luang Pó aguentava a dor e treinava
a si mesmo para ser capaz de lidar com qualquer obstáculo que surgisse;
ele lutava utilizando a força de \emph{samādhi} e sabedoria até que
conseguisse separar a dor -- que era uma manifestação do corpo -- e sua
mente. Ele não permitia que sua mente sofresse junto com o corpo,
multiplicando a quantidade de sofrimento. A doença nas gengivas
permaneceu por sete dias antes de desaparecer sem uso de medicamentos.

Ao fim do \emph{vassa}, Luang Pó voltou a residir no monastério, mas deu
ordem para que os demais monges fossem praticar na floresta, distantes
um do outro, mas que se reunissem no monastério todo dia de
\emph{uposatha}. Eles continuaram praticando dessa maneira até março de
1954, quando a mãe de Ajahn Chah, Mé Pim, veio visitar-lhe acompanhada
de seu irmão mais velho e outros cinco parentes e amigos de Ban Kó. Eles
pediram que retornasse a sua terra natal para ensinar o Dhamma para as
pessoas de lá. Luang Pó refletiu e decidiu que já era hora de retribuir
seu débito de gratidão à sua mãe lhe ensinando o Dhamma, e por isso
aceitou o convite. Após os leigos partirem, ele convocou uma reunião da
sangha e delegou o dever de cuidar do monastério a dois discípulos.
Quando tudo estava preparado, Luang Pó e um pequeno número de monges se
despediram dos aldeões em Ban Pah Tao e começaram a caminhada de volta a
Ban Kó.
