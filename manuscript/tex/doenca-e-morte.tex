\chapter{Doença e morte}

Durante toda sua vida, Ajahn Chah sempre teve um corpo forte, era
energético e trabalhador. Essa característica também influenciou sua
vida como monge e sua busca por libertação. Ele tocava sua prática com
força total, sem receio por si mesmo e plenamente disposto a sacrificar
sua vida para cumprir a nobre aspiração que fez quando seu pai morreu:
dedicar esta vida à realização de \emph{nibbāna}.

Um exemplo desse comprometimento incondicional era seu hábito de nunca
buscar auxílio médico. Ele enxergava tempos de doença como oportunidades
para desenvolver \emph{khanti pāramī} e testar sua resolução em praticar
o Dhamma. Houve muitas ocasiões em que o bom senso diria que sua
condição física era preocupante e que seria sábio buscar auxílio. Por
exemplo, na ocasião em que enfrentou seu medo de fantasmas, indo morar
num cemitério: na manhã do quarto dia ele ficou tão doente que começou a
urinar sangue, mas ainda assim não se preocupou em visitar um médico.
Outros exemplos foram quando residia em Wat Yai Chai Mongkhon e sofreu
uma severa dor e inchaço em seu estômago, além das várias ocasiões em
contraiu malária e outras doenças.

Toda essa austeridade não veio sem consequências. Apesar de ele ter sido
capaz de aguentar todas essas dificuldades durante sua vida, o preço que
teve a pagar foi um rápido envelhecimento e subsequente perda de saúde
que o assolou logo que chegou aos 60 anos de idade, deixando-o inválido
aos 64 e finalmente culminando em sua morte aos 74. Luang Pó mesmo não
estava surpreso com nada disso, pelo contrário: quando lhe perguntavam a
respeito ele expressava surpresa de ter sobrevivido tanto tempo,
considerando todas as provações e tormentos que impôs a seu corpo
durante sua juventude.

Os primeiros sinais desse envelhecimento e rápida perda de saúde
apareceram em 1977, durante sua visita à Inglaterra. Ele começou a
sentir dificuldade em equilibrar-se quando de pé e caminhando; sentia
seus pés dormentes e teve que começar a usar uma bengala para se apoiar.
Pouco tempo depois, sua memória começou a falhar a ponto de não mais
conseguir se lembrar do nome de pessoas próximas. Esses sintomas ainda
não eram constantes, passavam por períodos de melhora e piora. Luang Pó
tentava manter sua rotina diária ensinando e supervisionando o
monastério tanto quanto lhe fosse possível,
mas em vista dessa nova situação e talvez já pensando no futuro quando
não poderia mais exercer suas funções, decidiu oficializar Tahn Ajahn
Liem como vice-abade de Wat Nong Pah Pong. Ajahn Liem falou um pouco
sobre a ocasião:

``Após retornar de Vientiane, permaneci em Wat Pah Pong o tempo todo.
Luang Pu Chah disse que eu deveria ajudar em Wat Pah Pong, e por isso
nunca mais fui a lugar algum.

Eu não queria fazer mais que isso (ser vice-abade) porque era um monge
novato. Além disso, não sou de Ubon, portanto pensava que não seria
apropriado aceitar a posição de líder de Wat Pah Pong; mas se fosse útil
para os demais, então o faria.

Por respeitar muito Luang Pu Chah, fiquei e trabalhei ao lado dele, ele
queria que eu permanecesse aqui. Quando sugeriu que me tornasse
vice-abade, respondi que não queria essa responsabilidade tão cedo, não
me parecia apropriado, uma que vez haviam muitos monges que já faziam
parte de Wat Pah Pong há mais tempo que eu e eram bastante capazes. Mas
Luang Pu respondeu dizendo que deveríamos cumprir nossos deveres,
ajudando uns aos outros com ensinamentos também beneficiamos o Buddha
Sāsanā como um todo. E assim venho me esforçando de todas as formas, e
nunca com sentimentos ruins.''

Um exemplo de como, mesmo doente, Luang Pó continuava exercendo suas
funções foi um dia em que estava supervisionando o trabalho dos monges
que transportavam uma grande quantidade de terra para o jardim do
recém-construído salão de \emph{uposatha}. Enquanto estava de pé, dando
instruções, um grupo de adolescentes que passeava pelo monastério se
aproximou. Eles escolheram ficar de pé bem perto de Luang Pó de uma
maneira um tanto desrespeitosa; por seu modo de vestir, davam a
impressão de gostar de imitar os adolescentes ocidentais em tudo, e isso
talvez explicasse porque faziam questão de mostrar que não tinham
respeito por ele. O líder do grupo fez algumas perguntas a Luang Pó e
então disse: ``Por que você não manda esses monges irem sentar-se em
meditação? Você só os faz trabalhar!''

Ao que Luang Pó prontamente respondeu: ``Ficar muito tempo sentado causa
prisão de ventre.''\footnote{A frase original é um pouco mais ofensiva
  do que a tradução publicada aqui. Deixamos a cargo do leitor imaginar
  uma tradução mais apropriada.}

Os jovens foram pegos de surpresa pela resposta e não sabiam o que
entender daquilo. Após um instante, Luang Pó apontou sua bengala para o
peito do rapaz e disse: ``O correto não é apenas sentar ou apenas
caminhar. Precisamos sentar, mas também temos que trabalhar e corrigir
nosso saber e pontos de vista a cada instante. O correto é assim. Volte
para seus estudos, você ainda é muito criança. Se não tem real
conhecimento sobre prática do Dhamma, não tente falar a respeito: não
vai conseguir nada a não ser fazer papel de bobo.''

Outro encontro que teve quando supervisionando o trabalho de construção
do novo salão de \emph{uposatha} foi com uma americana que tinha extenso
conhecimento teórico sobre budismo. Ela ouvira falar da reputação de
Luang Pó e decidiu visitar Wat Nong Pah Pong. Quando chegou, Luang Pó
estava de pé no local da construção e ela lhe perguntou como ele
ensinava \emph{vipassanā}, ao que ele respondeu: ``Eu não ensino
\emph{vipassanā}, eu só ensino `atormentação'.''\footnote{Em tailandês:
  \thai{ทรมาน}.} Disse isso e foi embora, seguido do monge que o
acompanhava. Após caminharem um pouco, Luang Pó virou-se e disse para o
monge: ```Atormentar' é uma forma de desenvolver \emph{khanti pāramī}
para que possamos aprender sobre nossa mente. É como se nossa mente
fosse um rádio e tirássemos os circuitos para ver o que há dentro dele;
olhamos em detalhe, olhamos claramente, estudamos o funcionamento
interno até compreendermos e sermos capazes de consertá-lo, assim como
um técnico consegue consertar um rádio quando está quebrado.''

Ele também dedicava tempo exclusivamente para receber visitantes e
periodicamente visitava Wat Pah Nanachat, mas raramente ia a qualquer
outro lugar. Em 1980 sua saúde começou a decair ainda mais, causando-lhe
dificuldades para se alimentar, o que muitas vezes o deixava fraco. Ele
foi levado a um hospital e diagnosticado com vários problemas: falta de
oxigenação cerebral, deficiência cardiovascular, traqueia dilatada,
diabetes, além da asma que o perturbava já havia muito tempo. Em 1981,
quando sua memória e coordenação motora começaram a se degenerar
gravemente, ele foi submetido a uma operação cerebral, mas não houve
melhoria significativa da sua condição.

Alguns médicos disseram que a doença neurológica dele era de um tipo
mais comumente encontrado em pessoas com idade acima de 80 anos, o que
intrigou muitas pessoas em relação à possível causa. Alguns especulavam
se não era algo relacionado às diversas ocasiões em que contraíra
malária, porque é sabido que algumas formas da doença são capazes de
causar danos graves ao sistema nervoso. Ninguém sabia ao certo que tipos
de malária ele havia contraído no passado, uma vez que ele jamais ia ao
médico -- sequer para saber se o que tinha era de fato malária, e muito
menos para de fato tratar a doença. Outros especulavam se não havia
relação à ocasião em que ele removeu todos os dentes de sua boca numa
única operação. Isto foi o que aconteceu:

Em meados de 1967, Luang Pó estava sofrendo novamente do mesmo inchaço e
dor nas gengivas que o afligira durante sua estadia em Wat Tam Hin Ték,
mas agora estava tão forte que ele mal podia se alimentar.
Diferentemente de sua atitude costumeira nesse tipo de ocasião, desta
vez ele decidiu visitar um dentista, mas não para tratamento: ele queria
que o dentista removesse os dezesseis dentes que ainda lhe restavam na
boca de uma só vez. A partir de então ele usaria uma dentadura e o
problema estaria resolvido.

O dentista obviamente se recusou, uma vez que isso colocaria a vida do
paciente em risco. Normalmente ele removia um dente por vez, dando um
período de dois ou três dias de intervalo entre cada extração. Se Luang
Pó estava realmente com pressa, o dentista estava disposto a remover um
dente por dia, mas não mais que isso. Luang Pó não aceitava essa opção e
insistia que todos os dentes fossem removidos de uma só vez. Ele disse:
``Doutor, não se preocupe, se vou morrer ou viver é minha
responsabilidade. Não quero ter que vir vários dias e, de qualquer
forma, esses dentes têm que ser removidos. Por favor, tire os dezesseis
dentes e eu não vou mais ter que me preocupar com esse assunto.'' Mas
não foi só isso que Luang Pó pediu ao médico: ele também insistiu que a
operação inteira fosse realizada sem o uso de anestesia; a intenção dele
era usar \emph{samādhi} para lidar com a dor.

Tendo sido criado na Tailândia onde, naquela época, monges eram
altamente respeitados e venerados, o dentista provavelmente supôs
(corretamente até certo ponto) que Luang Pó estava num nível tão
superior às pessoas comuns que possuiria algum poder sobrenatural, e que
por isso não haveria problema em realizar a operação. Se Luang Pó havia
dito que conseguiria aguentar a extração de dezesseis dentes removidos
de uma só vez sem anestesia, só restava ao médico realizar a operação. O
dentista foi em frente e começou a extrair os dentes.

Mais tarde, Luang Pó contou a seus discípulos que permaneceu
perfeitamente ciente durante toda a operação e pôde observar que aquilo
estava tendo um efeito devastador sobre seu corpo, e que a certo momento
chegou muito perto da morte. Após retornar ao monastério, sua boca
sangrou ininterruptamente por três dias. A razão pela qual muitos
acreditam que esse evento foi uma das principais causas para seus
problemas de saúde nos anos finais é porque as raízes dos dentes estão
vinculadas diretamente ao sistema nervoso, e o choque de uma operação
tão drástica como essa certamente deve ter deixado um rastro de graves
sequelas.

Voltando ao ano de 1981, após a operação no cérebro, Luang Pó descansou
meses na casa de um médico nas cercanias de Bangkok para assim facilitar
seu tratamento. Durante esse período, ele em geral apenas descansava e
escutava fitas cassete com ensinamentos de Ajahn Buddhadasa, ou então
pedia a seus discípulos que lessem um livro com ensinamentos do mestre
chinês Huang Po e, enquanto liam, Luang Pó explicava para eles o
significado mais profundo das passagens.

Quando começou a se sentir um pouco melhor, retornou a Wat Nong Pah
Pong, mas em pouco tempo sua saúde deteriorou muito e ele começou a
manifestar dificuldades neurológicas sérias, como não ser capaz de falar
claramente, dificuldade em coordenar os movimentos do corpo,
incapacidade de sentir dor, sentir frio quando na verdade estava calor e
calor quando estava frio, além de algumas ocasiões em que caiu em
prantos e gargalhadas misturadas sem razão aparente. A partir de então
Ajahn Liem passou a exercer a função de abade do monastério.

Em abril de 1982, ele falou perante a sangha pela última vez,
exortando-os a terem respeito pelos monges mais sêniores e preservarem a
harmonia dentro da comunidade. Sua capacidade de locomoção e fala
continuaram a se deteriorar rapidamente, até que em janeiro de 1983 ele
estava completamente incapaz de falar ou se mover. Em março de 1987, ele
foi diagnosticado com uma doença respiratória e teve que receber uma
traqueotomia para poder continuar respirando; a partir desse ponto em
diante foi necessário alimentá-lo através de tubos. Desse momento em
diante sua condição continuou se deteriorando lentamente até seu
falecimento, em 16 de janeiro de 1992, aos 74 anos de idade.

Luang Pó passou um total de dez anos incapaz de falar ou se mover, tendo
que ser nutrido através de tubos. Muitos respeitados mestres de
meditação vinham visitá-lo regularmente e alguns deles afirmavam que a
situação em que se encontrava era devida a seu próprio \emph{kamma}, mas
também à sua compaixão por seus discípulos. Um venerado mestre da região
sul do país dizia: ``Ajahn Chah se deita doente para o benefício de seus
discípulos.'' Luang Pó Put Tāniyo, outro grande mestre discípulo de
Ajahn Sao, afirmava o mesmo e dizia que Luang Pó não estava em estado
vegetativo; muito pelo contrário, sua mente estava sempre brilhante e
pacífica, cem por cento ciente de tudo que acontecia a seu redor. Luang
Pó Put instruiu a sangha a reunir-se e recitar \emph{suttas}
regularmente para que Luang Pó Chah ouvisse, como oferenda a seu
professor.

É impossível sabermos com certeza se de fato Luang Pó tinha intenção de
permanecer vivo, mesmo em tais condições, para o benefício de seus
discípulos. Porém, aqueles que estiveram presentes durante aqueles anos
são unânimes em afirmar que a presença do professor, mesmo que na forma
de uma pessoa inválida, foi crucial para manter a harmonia da comunidade
e o nível de disciplina entre seus discípulos. Isso porque mesmo doente
ele continuava servindo de foco de atenção para toda a sangha, e sua
presença tinha o poder de fazê-la lembrar e respeitar o modo de prática
que lhes ensinou.

Exemplos do passado mostram que, quando um mestre famoso falece, o
padrão mais comum de eventos é seus discípulos se dispersarem em várias
direções; o monastério do mestre é abandonado ou transformado em museu
dedicado à sua memória, e às vezes ocorre mesmo de surgir inimizade e
disputas entre os antigos discípulos. Esse não foi o caso com Wat Nong
Pah Pong e Luang Pó Chah. Por décadas a sangha conseguiu continuar
vivendo em paz e harmonia antes que qualquer coisa do tipo se
manifestasse. Mesmo após a morte de Luang Pó Chah, Wat Nong Pah Pong,
sob liderança de Luang Pó Liem, continuou sendo um importante centro de
treinamento para monges, monjas e leigos. A comunidade de discípulos e o
número de monastérios filiados continuou crescendo continuamente dentro
e fora da Tailândia. Ainda hoje, a sangha de discípulos de Wat Nong Pah
Pong continua servindo como ponto de referência de praticantes que
protegem o verdadeiro caminho de libertação do Buddha e sua linhagem de
seres iluminados.

