\thistitleoffsettrue
\chapter{De volta à estrada}

Um monge chamado Luam acompanhou Luang Pó durante aquela nova
peregrinação. Certo dia, pararam para descansar num cemitério de
floresta perto de um vilarejo e dois garotos portadores de deficiência
física vieram conversar com eles. Ambos ficaram interessados em
enfrentar os perigos da vida em peregrinação e pediram permissão para
juntar-se a eles na viagem. Luang Pó não se opôs e, após terem recebido
permissão de seus pais, os garotos juntaram seus pertences e passaram a
acompanhar os monges na estrada. Tahn Luam os ensinou o método para
prática de meditação sentada e andando e ambos conseguiam praticar de
forma satisfatória; eram bastante resolutos e esforçados. Certa vez
Luang Pó falou sobre eles:

``Aqueles dois garotos, apesar de aleijados, ainda tinham fé no
\emph{Buddha Sāsanā}. Fizeram o esforço de se juntarem à viagem
enfrentando todas as dificuldades. Observá-los me fez refletir sobre
vários aspectos do Dhamma. Um tinha as pernas boas, boa visão, mas era
surdo. O outro tinha boa audição, boa visão, mas uma perna era torta. Na
hora de caminhar, o da perna torta andava, mas várias vezes a perna
torta se enroscava com a perna boa e o fazia tropeçar e cair. Já o que
era surdo, quando falávamos com ele tínhamos que usar as mãos, mas
quando ele dava as costas não adiantava chamar porque ele não escutava.
Foi real interesse que fez esses dois aleijados quererem juntar-se à
viagem, e a deficiência física não era obstáculo à determinação deles.

Peço que todos aqui tenham determinação de verdade; assim alcançaremos
todos nossos objetivos. Esses dois garotos não desejaram aquela
deficiência física. A mãe e o pai deles tampouco queriam que os filhos
fossem aleijados, mas ainda assim não conseguiram escapar à lei do
\emph{kamma}. É como o Buddha dizia: `Todos os seres são donos de seu
\emph{kamma}, são herdeiros de seu \emph{kamma}, nasceram de seu
\emph{kamma}.' Quando refleti sobre a deficiência dos dois garotos que
se juntaram à viagem, aquilo virou um ensinamento para mim mesmo: `Esses
dois garotos têm deficiência física, mas são capazes de trilhar o
caminho e atravessar selvas e florestas, mas eu sou deficiente em meu
coração, pois ele ainda possui \emph{kilesas} dentro de si. Essas
\emph{kilesas} vão me levar a atravessar as florestas? Esses dois
garotos com deficiência física não oferecem perigo a ninguém, mas se uma
pessoa tiver muitas deficiências em seu coração, ela criará confusão e
dificuldades que farão os demais sofrerem muito.''

Certo dia, a peregrinação os levou a uma grande floresta perto de um
vilarejo em Nakhon Panom. Eles chegaram ao pôr do sol e por isso
decidiram passar a noite ali. Lá havia uma trilha abandonada que
aparentemente cruzava a floresta inteira e levava às montanhas, o que
fez Luang Pó então lembrar-se de um velho ditado que diz: ``Se for à
floresta, não deite em uma trilha antiga.'' e ficar curioso em saber
qual seria a razão para isso. Ele queria descobrir qual o significado do
ditado e por isso mandou Tahn Luam acampar dentro da floresta, enquanto
ele acamparia bem no meio da trilha. Os dois garotos acamparam entre ele
e Tahn Luam, ao lado da trilha, para que pudessem ver ambos. Eles
começaram a noite praticando meditação, mas Luang Pó deixou sua proteção
contra mosquitos erguida para que os meninos pudessem vê-lo e assim não
sentissem medo. Já tarde, ele deitou-se sob seu \emph{glot} na postura
do leão,\footnote{O Buddha recomendava que os monges sempre dormissem
  deitados sobre o lado direito do corpo.} obstruindo a trilha -- de
costas para a floresta e de frente para o vilarejo.

Enquanto estava deitado, ainda praticando meditação, ouviu o som de
passos sobre as folhas secas, passos vagarosos e ritmados chegando cada
vez mais perto e trazendo consigo um cheiro pungente de carne podre que
logo impregnou toda a área. Luang Pó continuou deitado imóvel,
plenamente ciente de que aquele som e cheiro não poderiam ser de outro
animal senão um tigre. Parte dele sentiu medo de morrer a ponto de
começar a tremer, mas a parte guerreira que havia dentro de si mais uma
vez emergiu e raciocinou: ``Não se preocupe com perder esta vida. Mesmo
que o tigre não nos mate, ainda teremos que morrer. Se morrermos
enquanto caminhamos seguindo os passos do Buddha, nossa vida terá valido
a pena. Estou disposto a servir de comida de tigre caso no passado tenha
bebido o sangue ou comido a carne dele, pois assim vai se encerrar minha
dívida cármica; mas se nunca tivemos inimizade no passado, provavelmente
não haverá perigo.'' Tendo pensado isso, Luang Pó trouxe à mente as Três
Joias\footnote{O Buddha, o Dhamma e a Sangha.} e também tomou sua pureza
de \emph{sīla} como refúgio naquele momento difícil. Assim que ele
refletiu dessa maneira, sua mente ficou clara, leve e livre de quaisquer
preocupações.

O som do tigre caminhando cessou, mas Luang Pó ainda conseguia ouvir a
respiração dele a cerca de cinco metros de distância. Após um instante,
o animal deu meia-volta e foi embora, retornando para dentro da
floresta. E assim Luang Pó conseguiu descobrir a razão pela qual os
antigos costumavam dizer: ``Se for à floresta, não deite em uma trilha
antiga.''

Sobre esse assunto, Luang Pó uma vez disse: ``Abandonar o temor por
nossa vida, abrir mão dela sem remorso, sem medo de morrer, faz surgir
verdadeira leveza e bem-estar na mente. \emph{Sati} e sabedoria ficam
afiadas, destemidas e nos acompanham como nossa própria sombra. A mente
fica corajosa, não balança perante nada, é fantástico. É possível usar
esse método de renúncia quando se está doente ou quando estamos sob
perigo. Vai levantar nossa moral, não vai nos deixar perder \emph{sati}
e ficarmos loucos. Tendo \emph{sati} vamos conseguir resolver o
problema, vamos poder agir de forma correta, sem cometer erros.''

Luang Pó, Tahn Luam e os dois garotos continuaram partilhando as
alegrias e sofrimentos da busca pelo Dhamma na estrada, mas tendo vivido
juntos por um longo período, a verdadeira personalidade de cada pessoa
começou a se manifestar e Luang Pó achou que continuar viajando com
pessoas que não estavam no mesmo nível que ele causaria atraso em sua
prática. Ele também se sentia irritado e oprimido pela presença e
comportamento dos seus companheiros e decidiu então que seria melhor
continuar sozinho para poder acelerar ainda mais sua prática. Ele então
se separou do grupo e Tahn Luam se voluntariou a levar os dois garotos
de volta ao seu vilarejo de origem. Luang Pó seguiu adiante sozinho até
chegar a um monastério abandonado perto de Ka Nói, em Nakhon Panom. Ele
refletiu e decidiu que aquele era um bom local para prática de meditação
e ali permaneceu por muitos dias.

No começo teve uma sensação de liberdade por ter se separado de seus
amigos. Não tinha mais preocupações e podia agora acelerar seu ritmo de
prática à vontade, e era fácil manter a mente presente, atenta aos
estímulos sensoriais e mentais o tempo todo. Quando saia para
\emph{pindapāta}, não olhava para ninguém; a única coisa que reparava
era se a pessoa oferecendo comida era um homem ou uma mulher. Quando
terminava sua refeição, lavava e guardava seus pertences e em seguida
começava a praticar meditação andando por várias horas seguidas. Agiu
dessa forma por vários dias até que seus pés começaram a inchar por
causa do esforço excessivo. Luang Pó então teve que mudar para apenas
prática de meditação sentada até que seus pés voltassem ao normal, após
cerca de três dias.

Naquela época, Luang Pó não admitia receber visitantes, pois achava que
atividades sociais atrasariam seu progresso no Dhamma. Porém, um dia as
\emph{kilesas}, que haviam sido suprimidas pela força de \emph{samādhi},
emergiram mais uma vez para perturbar sua mente e fazê-lo pensar:

``Eu estou aqui sozinho deste jeito\ldots{} Ia ser bom se tivesse um
noviço ou um anagārika para me fazer companhia, assim ajudaria com
tarefas pequenas.'' Mas logo em seguida um novo pensamento surgiu:

``Ei, você deve ser muito importante, não? Se estava cansado dos amigos,
por que agora quer ter um amigo?''

``É verdade que estava cansado, mas só de pessoas ruins; o que estou
querendo agora é um bom amigo.''

``E onde você vai encontrar uma boa pessoa? Não enxerga? É possível
encontrar uma boa pessoa? Os amigos que lhe acompanhavam na estrada,
você disse que não eram bons -- você deve pensar que é a única pessoa
boa neste mundo, a ponto de ter fugido para um lugar abandonado como
este.''

Quando pensou assim, achou um fundamento que passou a utilizar em sua
prática daquele dia em diante: ``Onde vou achar uma boa pessoa? As boas
pessoas estão em nós mesmos. Se formos bons, onde quer que formos será
bom. Se nos criticarem, elogiarem ou disserem o que seja, ainda
continuamos bons, mesmo que nos ofendam. Mas se ainda não somos bons,
quando nos criticarem ficaremos com raiva, quando nos elogiarem
ficaremos felizes -- vamos ficar balançando desse jeito.

Quando entendermos onde estão as boas pessoas, teremos um fundamento
para abrir mão de nossos pensamentos. Aonde quer que formos, mesmo se
não gostarem de nós ou nos criticarem, não serão eles os bons ou ruins
porque `bom' e `ruim' está dentro de nós, e~somos nós quem nos
conhecemos melhor do que qualquer outra pessoa\ldots{}''

Após sua estadia nesse local, Luang Pó continuou viajando em busca de
lugares para desenvolver sua prática até que chegou a Ban Kók Yau, em
Nakhon Panom, e lá se abrigou em mais um monastério abandonado que
ficava a cerca de 400 metros do vilarejo. Durante aquele período sua
mente estava bastante pacífica e leve; ele sentia que algo estava por
acontecer, como mais tarde relatou:

``Certa noite, por volta das onze horas da noite, estava praticando
meditação andando. Estava me sentindo estranho e aquela sensação já
vinha desde durante o dia. Não tinha muitos pensamentos e tinha uma
sensação de bem-estar. Havia alguma celebração no vilarejo. Quando me
cansei de praticar meditação andando, me sentei em minha cabana com
telhado de palha. Quando me sentei, quase não tive tempo de cruzar as
pernas: minha mente queria entrar em \emph{samādhi}. Foi algo
automático: assim que sentei, a mente se pacificou de verdade, senti meu
corpo pesado. O som deles cantando no vilarejo\ldots{} não é que não
ouvisse, eu ainda conseguia ouvir, mas conseguia parar de ouvir se
quisesse. Estranho\ldots{} quando não prestava atenção, havia silêncio
-- se quisesse ouvia; se não quisesse, não ouvia, mas não me irritava.
Dentro da minha mente era como se houvesse dois elementos distintos, não
relacionados. Era como se a mente e os objetos mentais existissem
separadamente, assim como este cesto e este bule aqui (são objetos
distintos). Então compreendi que, quando a mente entra em
\emph{samādhi}, mas ele ainda não é firme suficiente, conseguimos ouvir
sons; mas, se a deixarmos vazia, haverá silêncio. Se houver sons, basta
focar naquela ciência -- ambos são separados.

Então pensei: `Se não for isso, vai ser o quê? É assim, não são
conectados!' E fui contemplando dessa forma até entender `Oh! Isso é
importante. Isso é \emph{santati}\footnote{Continuidade do fluxo mental
  (pāli).} que então resulta em \emph{santi}.\footnote{Paz, tranquilidade
  (pāli).}' e continuei praticando. Naquele momento em que
estava sentado em meditação minha mente não se interessava por nenhuma
outra coisa. Se quisesse parar de praticar, conseguia sem problemas.
Caso parasse, seria por preguiça? Seria por cansaço? Seria por estar
irritado? Não, não havia nada disso. Posso dizer com certeza que não
havia, não havia nada disso em minha mente. Só havia equilíbrio e
harmonia dentro dela. Se parasse seria porque parei, só isso.

Eventualmente parei para descansar, mas o que parou foi só a postura
sentada -- a mente continuava como antes, não parava. Assim que minha
cabeça tocou o travesseiro, a mente se inclinou -- eu não sabia para
onde se inclinaria, mas ela se inclinou para dentro. Foi como se fosse
um fio elétrico cujo interruptor foi desligado. Meu corpo explodiu com
um grande estrondo. A ciência que havia ali era muito refinada, e quando
a mente ultrapassou aquele ponto, ela se desprendeu e foi bem fundo para
dentro. Foi para dentro onde não há absolutamente nada e onde nada nesse
mundo é capaz de alcançar. Nenhuma coisa desse mundo é capaz de alcançar
aquele ponto. Ela permaneceu ali dentro por um momento e então voltou
para fora. Dizer `voltou para fora' não significa dizer que eu mandei
ela vir -- eu era apenas o observador, eu era apenas aquele que estava
ciente de tudo aquilo. Ela continuou emergindo até voltar a seu estado
normal.

Tendo voltado a seu estado normal, surgiu a pergunta: `Que foi isso?', e
a resposta veio: `Essas coisas acontecem por si mesmas, não precisa
ficar em dúvida a respeito', e foi só dizer isso que a mente largou
aquela dúvida. Depois de descansar por um momento, ela novamente
voltou-se para dentro. Não fui eu quem a fez entrar, ela foi sozinha.
Foi entrando, entrando, e tocou naquele mesmo interruptor -- não havia
nada que alcançasse aquele ponto. Ela permaneceu ali pelo tempo que quis
e então voltou à superfície quando chegou a hora de voltar. Foi tudo
automático, eu não determinei que fosse daquele jeito: `Seja assim, saia
desse jeito, entre desse jeito\ldots{}', não houve nada disso. Eu era
apenas a pessoa ciente, apenas o espectador. Ela voltou ao seu estado
normal, mas dessa vez não surgiu dúvida alguma, apenas continuei
sentado, contemplando e ela foi para dentro novamente. Nessa terceira
vez o mundo estilhaçou-se por completo: a terra, o chão, as plantas, as
árvores, as montanhas -- tudo se tornou elemento espaço. Não havia
pessoas, tudo desapareceu. Nessa última vez não sobrou nada.

Ela permaneceu ali de acordo com a vontade dela. Não sei como ela
estava, era difícil ver, difícil falar a respeito. Não há com o que
comparar esse tipo de coisa. Ficou naquele estado mais tempo do que nas
ocasiões anteriores -- eu era apenas o observador -- e então emergiu e
voltou ao normal. Qual o nome desses três eventos? Quem sabe! Que nome
poderia dar?

Tudo que contei aqui foi sobre a mente natural e nada mais. Não falei
sobre \emph{citta}\footnote{A mente (pāli).} ou
\emph{cetasika}\footnote{Manifestações mentais (pāli).} -- não há
necessidade. Quem tiver fé, pratique de verdade, arrisque sua vida;
quando chegar a esse nível do qual falei há pouco, é como se este mundo,
esta terra, virasse de cabeça para baixo. Nosso entendimento, nosso modo
de enxergar o mundo, muda completamente. Naquela hora, se alguma outra
pessoa nos visse, ia achar que somos completamente loucos -- e se você
não controlar bem sua mente, talvez fique louco de verdade, porque nada
permanece como antes. Quando olha as pessoas no mundo, não enxerga como
antes, mas só nós vemos aquilo. Tudo muda. O modo de pensar do mundo
inteiro vai naquela direção, mas o nosso vem nessa outra. As outras
pessoas falam nessa direção; já nós falamos na outra. As demais pessoas
sobem por um caminho, e nós descemos por outro. Tudo em nós difere das
demais pessoas, e continua sendo assim daí em diante.''

Luang Pó permaneceu naquele monastério abandonado em Ban Kók Yau por 19
dias e então continuou sua peregrinação por outras partes. Ele sentia
que naquele momento sua capacidade de ensinar o Dhamma e resolver
problemas com a prática, tanto seus próprios como os das demais pessoas,
estava muito proficiente. No que diz respeito ao Dhamma, ele não tinha
dúvida sobre nenhum aspecto. Luang Pó caminhou em direção a Sri Sonkram,
em Nakhon Panom, até alcançar o rio Mekong, cruzou à outra margem e foi
prestar reverência a lugares sagrados no Laos. Terminado isso, cruzou de
volta à Tailândia e mais uma vez descansou em Ban Nong Ka.

Naquela época, sua tigela de esmolas era feita de ferro, mas já tinha
muitos anos de uso e estava tão enferrujada que apresentava buracos,
então os monges do monastério local lhe deram uma outra que também era
usada, mas em melhor condição que a dele. Essa foi mais uma oportunidade
para ele contemplar sua ganância por utensílios e mostrar-lhe que sua
prática ainda não estava firme o suficiente. Mesmo tendo tido uma
experiência tão profunda de \emph{samādhi} em Ban Kók Yau apenas alguns
dias antes, ele novamente estava sendo assolado pela \emph{kilesa} da
ganância:

``Eles haviam me oferecido uma tigela, mas ela tinha um buraco e não
tinha tampa. Eu lembrei da época em que era garoto e cuidava dos búfalos
e via meus amigos pegarem junco e trançarem em forma de chapéu. Então
pedi que trouxessem junco para que trançasse. Fiz uma peça plana e uma
redonda como um anel, depois trancei ambas juntas e até consegui uma
tampa para a tigela, só que parecia mais um desses cestos de colocar
arroz. Quando saía em \emph{pindapāta} todos olhavam minha tigela com
estranheza e no final começaram a me chamar de `o monge da tigela
grande',\footnote{Essa é uma expressão para um monge que tem o hábito de
  pedir com frequência, que nunca está satisfeito, que é ``pidão''.}
mas eu não dava bola.

Tentei fazer novamente. Trabalhava dia e noite, era um esforço incorreto
porque havia muito desejo. À noite acendi uma tocha e trabalhava sozinho
na floresta. Fui trançando, trançando e acabei esbarrando na tocha --
brasas caíram na minha mão e me queimaram, a pele descascou por
completo, ainda hoje tenho a cicatriz. Então me dei conta: `Ei, que
estou fazendo? Estou pensando errado! Por acaso me ordenei por causa de
utensílios, mantos e tigelas? Estou me esforçando a ponto de não dormir,
estou desejando muito essa tampa de tigela! Esse é um esforço
incorreto.'

Larguei aquilo e sentei para refletir. Então pratiquei meditação
andando, mas enquanto andava comecei a pensar na tampa da tigela
novamente. Voltei a trabalhar como antes, trabalhei movido por desejo
até o sol raiar. Fiquei cansado e me sentei para meditar, para pensar.
`Isso está errado\ldots{}' comecei a adormecer e logo vi uma imagem de
um Buddha gigantesco, que me disse: `Venha aqui, eu vou te ensinar
algo.' Eu fui prestar reverência e ele ensinou sobre o uso desses
utensílios: `S\emph{abbe ime parikharara pañca khandhana
parivārayeva!}'\footnote{``Todos esses utensílios são apenas adornos para
  os cinco khandhas!'' (pāli)} Me assustei e acordei tremendo da cabeça
aos pés. Aquela voz permanece na minha memória até hoje.

Fiquei com medo de seguir adiante, então parei. Desejava tanto a ponto
de perder noção de mim mesmo. Então parei e passei a trabalhar em
períodos: trabalhava um pouco, caminhava um pouco, praticava meditação
um pouco. Esse é um ponto muito importante, se ainda não terminamos
qualquer trabalho, quando colocamos de lado e vamos praticar meditação,
a mente ainda fica presa àquela tarefa, não consegue abrir mão, não
solta, não importa quanto puxe. Então tomei isso como uma forma de
treinar minha própria mente, treinar em largar, em abrir mão: quando
fazia algo, não me apressava em terminar. Larguei a tampa da tigela e
fui praticar meditação, mas a mente ainda estava pensando na tampa. Fui
praticar meditação andando e ela ainda estava fixa naquela tampa de
tigela.

Então vi que é muito difícil para esta mente abrir mão das coisas; ela
se apega com muita força, mas graças a isso encontrei um fundamento para
praticar: quando fazia algo, não me apressava em terminar. Fazia um
pouco, colocava de lado e olhava para minha própria mente. Ia sentar em
meditação e a mente ainda rodava ao redor do trabalho inacabado, mas eu
só observava. Aí ficou divertido, comecei a lutar desse jeito. Decidi
treinar nisso até aprender. Queria ser capaz de trabalhar, mas também de
largar quando chegasse a hora: queria que fossem duas coisas
completamente independentes, que não gerassem sofrimento. Mas era muito
difícil mudar esse comportamento, é difícil largar esses apegos, difícil
abrir mão.

As pessoas pensam em terminar rápido seu trabalho porque assim o assunto
se encerra e não vão mais precisar se preocupar com aquilo. Pensar assim
também está certo, mas se pensarmos do ponto de vista do Dhamma, não
está realmente correto, porque não existe nada capaz de se encerrar se
nossa mente não largar. Pensei sobre as sensações, felicidade e
sofrimento: `Como vou conseguir largar disso, como abrir mão? Me deixo
levar por ambos, assim como me deixei levar por aquela tampa de tigela.
Se eu enxergar bem aqui, vou enxergar ali também; se treinar nessa
ocasião, vou vencer naquela outra também.'

Então encontrei um fundamento para a prática: quando fazia algo, não me
apressava em terminar. Fazia um pouco e largava, ia praticar meditação
andando. Se a mente ainda se preocupava com o trabalho, eu dava bronca
nela, dava bronca em mim mesmo, alertava a mim mesmo, treinava a mim
mesmo, conversava sozinho na floresta, lutava desse jeito. Em seguida
ficou leve. Eu queria treinar até conseguir -- quando fosse hora de
largar, tinha que conseguir largar. Queria que fossem duas coisas
independentes -- conseguir trabalhar, mas também conseguir largar,
queria que fossem independentes uma da outra. Fui treinando e aos poucos
ficou mais leve, mais fácil. Então soube que é assim mesmo que se tem
que fazer.

Desse momento em diante, quando tinha que costurar alguma coisa, fazer
uma capa para minha tigela ou qualquer outra coisa, eu aproveitava para
treinar a mim mesmo dessa forma. Conseguia fazer, mas também conseguia
largar. Então soube que a causa do sofrimento é essa mesmo, já conhecia
a causa do sofrimento e então o Dhamma surgiu graças a esse
conhecimento. Já tinha visto: nasce desse jeito mesmo. Deste ponto em
diante, quer estivesse em pé, andando, sentado ou deitado, era sempre
divertido e agradável. Mas quando acabei de fazer a tampa para a tigela
e saí em \emph{pindapāta,} eles ainda estranhavam: `Por que a tigela
desse monge é desse jeito?'

Em seguida pensei em algo mais para fazer, e então decidi usar seiva de
seringueira para revestir minha tigela.\footnote{Isso é feito para
  proteger contra ferrugem e dar uma cor escura à tigela.} Me lembrava
de ter visto monges fazendo isso quando era noviço. Primeiro pensei em
usar óleo de seringueira, mas achei que seiva era mais transparente.
Então fui para Ban Kôk em Yasoton, distrito de Leng Nok Ta -- para os
lados de lá tem muito óleo de seringueira. Revesti a tigela inteira e
também a tampa. Após terminar, um leigo disse para colocar num cesto e
deixar de molho no lago para que esfriasse, assim ia secar mais rápido.
Após três dias ainda não estava seco. Esperei um mês inteiro e ainda não
secava, não podia ir em \emph{pindapāta}, não podia ir a lugar algum
porque a tigela ainda não estava seca. Sentava em meditação e me
preocupava, olhava todo dia para ver se já tinha secado. Sofrimento de
verdade.

No final decidi que, mesmo que esperasse um ano, ainda não estaria seca,
então disse para o leigo trazer um papel e colei do lado de fora da
tigela. Assim pelo menos dava para ir em \emph{pindapāta}. Não tinha
coragem de pedir uma tigela nova ao leigo, tinha medo que fosse mau
\emph{kamma}. Apenas aguentava em silêncio. A tigela não tinha tampa,
então lembrei das travessas de metal que tinha visto quando me ordenei
em Wat Ban Kó Nók. Pensei em pegar uma travessa e cortar uma parte
plana, martelar outra em forma de anel e soldar, assim daria para usar
como tampa. Então foi o que fiz, não pensei em pedir a ninguém. Eu era
estranho, não gostava de pedir nada. Quando a tigela secou, ficou
completamente negra, tanto a tigela como a tampa.''

\emph{Yonisomanasikāra} -- ter inteligência ou pensar com sabedoria --
era uma característica forte da prática de Ajahn Chah desde o começo.
Era como se houvesse um intercâmbio automático de perguntas e respostas
ocorrendo dentro dele até que o problema fosse resolvido. Outra ocasião
em que Luang Pó se beneficiou dessa habilidade em utilizar situações e
coisas comuns e transformá-las em Dhamma dentro de sua mente ocorreu
nesse mesmo ano quando estava viajando sozinho em busca de reclusão:

Durante uma época em que estava morando sozinho numa montanha, Luang Pó
ficou tão doente que sequer conseguia ficar de pé. A febre, aliada ao
fato de que não tinha se alimentado por vários dias, o fez sentir-se tão
fraco que pensou que daquela vez realmente fosse morrer: ``Se eu morrer
aqui no meio da floresta deste jeito e eventualmente encontrarem meu
cadáver, eles vão enviar a notícia para minha família e eles terão o
fardo de viajar até aqui para organizar um funeral.''

Ele pensou dessa forma e então tirou sua carteira de identidade de
dentro de sua sacola. Ele queria tê-la à mão caso começasse a sentir que
estava realmente a ponto de morrer, para poder queimá-la e assim
destruir qualquer possibilidade de identificarem seu corpo. Enquanto
pensava dessa forma ouviu o som alto de um veado bramando na montanha, o
que o fez pensar e perguntar a si mesmo:

``Veados e demais animais da floresta ficam doentes?''

``Também ficam doentes porque possuem corpos, assim como eu aqui.''

``Eles têm remédio para tomar, têm médico para dar injeção, ou não?''

``Não, não têm. Provavelmente usam quaisquer brotos de árvore ou folhas
que conseguirem encontrar.''

``Os animais da floresta não tomam remédio e não têm médico para cuidar
deles, mas, ainda assim, têm vários filhotes e conseguem dar
continuidade à espécie, não é?''

``Sim, está correto.''

Pensando assim, sentiu-se encorajado e fez um esforço para levantar-se e
arrastar-se até onde estava seu cantil d'água e beber. Então sentou-se
em meditação até que a febre começou a regredir pouco a pouco. Quando
amanheceu, já tinha força suficiente para ir ao vilarejo recolher
esmolas e pôde alimentar-se novamente. Ele uma vez falou sobre seu modo
de vida durante esse período:

``Antigamente não tinha sequer um filtro para água, porque naquela época
era difícil obter as coisas. Eu só tinha uma cumbuca pequena de alumínio
à qual era muito apegado. Antigamente eu ainda fumava cigarros, mas não
tinha fósforos, tinha apenas uma pedra que batia para criar faíscas.
Usava metade de um limão seco como tampa para meu porta fumo. À noite,
quando me cansava de praticar meditação andando, sentava e batia a pedra
para acender o cigarro. Eu acho que o som das batidas -- `pók, pók' --
no meio da noite deve ter assustado até os fantasmas!

Pensando sobre a época em que praticava sozinho\ldots{} Prática é algo
sofrido e muito difícil, mas, ao mesmo tempo, muito divertido. É ao
mesmo tempo divertido e sofrido. É igual a comer pasta de pimenta com
gengibre, ainda mais se adicionar \emph{kampong} assado: é ao mesmo
tempo gostoso e ardido, vamos comendo e o nariz começa a escorrer. Não
conseguimos parar de comer porque é gostoso, então vamos reclamando e
comendo ao mesmo tempo. É esse o propósito desse prato.

Eu digo que quem pratica o Dhamma tem que aguentar muita coisa, porque
não é algo leve -- é pesado! Pode-se dizer que se tem que estar disposto
a morrer: se um tigre vier comer ou um elefante vier esmagar, tem que
estar disposto a morrer. Tem que pensar assim: `Boa hora para morrer!'
Se nossa \emph{sīla} é impecável, não teremos que nos preocupar com mais
nada: morrer é como não morrer. Dessa forma não há medo, é uma arma do
Dhamma, a arma é o Dhamma.

Já estive em montanhas por toda parte, mas minha arma era só o Dhamma,
só havia renunciar, abrir mão, coragem, estar disposto a morrer,
renunciar à vida. Pensando nisso vejo que a arma do Buddha é melhor que
a do caçador, pois ela faz a força do nosso coração ficar mais firme.
Quando contemplamos algo, olhamos, pensamos e entendemos tudo. Quando
vemos algo, enxergamos por completo. `Sofrimento é assim, sofrimento
cessa assim', então é tranquilo. Quem enxerga o sofrimento, mas não
enxerga por completo, apenas alcança paz, não consegue um meio para
conhecê-lo. Aquele que não tem medo de morrer, que está disposto a
morrer, acaba não mais morrendo.\footnote{Porque alcança a iluminação.}
Sofra até ultrapassar os limites do sofrimento, pois o sofrimento acaba
lá, continue até enxergar de verdade, até ver o Dhamma. Isso tem muito
valor, traz muita energia para a mente, você não vai mais ter medo de
pessoas, da floresta ou de animais. O coração vai ficar firme e forte se
você pensar dessa forma.

O coração de um monge \emph{kammatthāna} é resoluto. Qualquer pessoa que
se dedique à \emph{kammatthāna}, se sua mente chegar ao ponto de estar
disposta a renunciar à própria vida, vai achar mais difícil matar uma
galinha do que sacrificar a si mesmo. Fica resoluto a esse ponto se
enxergar (as consequências de) mau \emph{kamma}. Vai ter um coração
grande, elevado e firme. Se pensar de acordo com o Dhamma, renunciar é o
que há de mais sublime.

Teremos uma arma melhor que a do caçador na floresta, teremos a arma do
Dhamma. Isso se chama \emph{vitakka},\footnote{Pensamento (pāli).} quer
dizer, a mente traz um assunto à tona e \emph{vicāra}\footnote{Investigação
  intelectual (pāli).} contempla tudo aquilo. \emph{Vitakka} e
\emph{vicāra} fazem seu trabalho continuamente até conseguirem enxergar
por completo aquele assunto. Surge êxtase, aqui os pelos do corpo
arrepiam, quando caminhando em meditação os pelos arrepiam, pensando no
Buddha, no Dhamma, os pelos arrepiam, surge êxtase, o corpo se sente
refrescado. Continuamos sentados e eles vão trabalhando, \emph{vitakka}
e \emph{vicāra}. \emph{Vitakka} gera êxtase, satisfação em nossa
prática, em vencer obstáculos; os pelos do corpo se arrepiam, as
lágrimas rolam. Ganhamos ainda mais ânimo para lutar, não recuamos não
importa o que aconteça. \emph{Vitakka}, \emph{vicāra} e então êxtase,
felicidade, surge. É uma felicidade munida de sabedoria, e nos apoiamos
em \emph{vitakka} e \emph{vicāra}, nos apoiamos naquela felicidade
firme. Naquele momento, é dito que nos apoiamos na força de
\emph{jhāna}\footnote{Estados profundos de samādhi (pāli).} -- é o que
dizem, mas eu não sei. Aquilo é do jeito que é, quem quiser pode chamar
de \emph{jhāna}. \emph{Vitakka\ldots{}} mais um pouco e \emph{vitakka}
desaparece, em seguida \emph{vicāra} desaparece, não há mais êxtase:
surge \emph{ekaggatā}\footnote{Estado profundo de unificação mental
  (pāli).} -- a mente se unifica, se apoia em \emph{samādhi}, se apoia
na firmeza de \emph{samādhi}. Então surge paz e ela nos serve de
fundação; a paz é a fundação de onde a sabedoria brotará.

Eu então compreendi que só através da prática é possível saber, ver de
verdade. Estudar e pensar sozinho é uma coisa completamente diferente.
Quem quiser pode ficar pensando `isso deve ser aquilo', mas eu digo que
no final tudo isso vai culminar nesse mesmo local. Aí fica tranquilo,
nosso corpo pode ser gordo ou magro e ainda estamos tranquilos. Mesmo
doentes ainda estamos tranquilos. Nunca pensei: `Onde estará minha mãe,
aquele parente, aquele amigo, onde estarão?', nunca pensei nisso.
Praticava disposto a morrer, só isso. Não tinha nenhuma preocupação. Era
firme desse jeito. E a mente, que antes vacilava, muda e fica firme e
resoluta em praticar.''

