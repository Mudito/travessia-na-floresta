\chapter{O monge sem raiva}

Quando o começo da estação chuvosa se aproximava, Luang Pó chegou a um
monastério construído dentro de um antigo cemitério de floresta em
Nakhon Panom. Caminhando por uma pequena trilha na floresta ele chegou a
um salão onde encontrou Luang Ta Pum, o líder da comunidade daquele
monastério, sentado conversando com os demais monges. Luang Pó saudou a
todos e explicou sua razão para estar ali e a conversa seguiu adiante,
mas a certa altura Luang Ta Pum disse algo que deixou Ajahn Chah muito
intrigado. Ele declarava que seu nível espiritual era tal que já não
havia mais raiva dentro de si. Era a primeira vez que Luang Pó ouvia um
praticante do Dhamma falar dessa forma, e sentiu vontade de investigar
melhor para descobrir até que ponto aquilo era verdade.

Luang Pó pediu permissão para passar o \emph{vassa} naquele monastério
com eles, mas seu pedido foi recusado -- Luang Ta Pum não aceitava
estranhos no monastério facilmente. Principalmente por Luang Pó ter
vindo só e de surpresa, Luang Ta Pum ficou desconfiado, pois não tinha
como ter certeza de quem ele de fato era e quais eram suas reais
intenções em estar ali. Os demais monges também eram contrários à
presença de Luang Pó no monastério, porém no final o autorizaram a
passar o \emph{vassa} no cemitério de floresta, do lado de fora do
monastério.

Quando faltavam apenas alguns dias para o começo do \emph{vassa}, Luang
Ta Pum mandou um dos monges convidar Luang Pó para passar o \emph{vassa}
dentro do monastério, porque um respeitado Ajahn da região o havia
alertado: ``Não é apropriado pedir que um monge sênior como aquele passe
o \emph{vassa} fora do monastério. Talvez ele tenha um alto grau de
desenvolvimento na prática do Dhamma e você estaria agindo de forma
desrespeitosa.''

Por ter a contragosto autorizado Luang Pó a passar o \emph{vassa} dentro
do monastério, Luang Ta impôs várias regras para que ele seguisse:

\begin{enumerate}
\item
  Ele não deveria receber as oferendas que os leigos traziam ao
  monastério: deveria esperar outro monge receber e então compartilhar
  com ele, caso o monge julgasse apropriado.
\item
  Ele não participaria da recitação do \emph{pātimokkha}; ao invés
  disso, deveria esperar a recitação acabar para então entrar e declarar
  sua pureza.\footnote{O Buddha estabeleceu que a cada 15 dias os monges
    devem se reunir e recitar a regra monástica (pātimokkha). Só podem
    participar da recitação monges que não tenham pendente nenhuma
    ofensa da regra monástica. Por isso, é comum os monastérios só
    permitirem a participação de monges cujo bom comportamento seja
    conhecido deles. O fato de Luang Ta Pum somente dar permissão para
    que Luang Pó viesse ao fim da recitação e declarasse não ter
    quebrado nenhuma regra monástica mostra que não estava disposto a
    aceitá-lo como um verdadeiro membro da comunidade, mas que ainda
    assim esperava que ele observasse todas as regras de conduta.}
\item
  Durante a refeição, em vez de sentar-se em ordem, de acordo com a
  quantidade de anos de vida monástica, Luang Pó deveria sentar-se ao
  fim da fila, atrás do monge mais novo do monastério.
\end{enumerate}

Ajahn Chah concordou em seguir todas essas regras, mesmo sabendo que era
mais sênior do que todos os monges ali, incluindo Luang Ta Pum, pois já
tinha mais de dez anos de vida monástica. Ele refletiu e decidiu colher
os benefícios que essa prática lhe traria. Luang Pó disse a si mesmo que
Luang Ta Pum e os demais apenas o estavam testando; que não importava
onde se sentasse durante a refeição, da mesma forma que um diamante não
perde seu valor dependendo do local onde foi posto. Além do mais, essa
prática iria ajudá-lo a reduzir sua vaidade.

O \emph{vassa} com Luang Ta Pum prosseguiu em paz, em grande parte
graças à humildade e sabedoria de Ajahn Chah, que ajustou seu
comportamento à situação e continuou se dedicando à sua prática regular
do Dhamma como sempre fez. Ele procurava falar o mínimo possível e,
quando alguém o admoestava, tomava aquilo como objeto de contemplação
para desenvolver sabedoria e melhorar a si mesmo. Ele escolheu apenas
guardar as coisas boas do modo de prática daquele monastério e usá-las
como ensinamentos para si mesmo.

Ao mesmo tempo, Luang Ta Pum e os demais monges vigiavam Luang Pó
constantemente para saber se estava fazendo algo errado, mas Luang Pó
não se incomodava e não mostrava irritação; pelo contrário, sentia
gratidão por todos: ``Eles me ajudavam não me deixando ficar descuidado
e agir errado, como alguém impedindo que algo me sujasse.''

Durante aquele vassa, a mente de Luang Pó estava sempre pacífica e
firme. Ele praticava constantemente, seu comportamento era comedido e
agradável em todas as situações, tal como o Buddha elogiava, e isso fez
que os monges do monastério aos poucos relaxassem a apreensão que
sentiam com relação a ele. Certo dia, saíram todos de barco para
procurar lenha e Luang Pó foi junto. Quando chegaram a uma área deserta,
começaram a recolher pedaços de madeira e deixá-los à margem do rio, e a
Luang Pó foi designada a tarefa de colocá-los dentro do barco. A certa
altura ele notou um pedaço de carvalho com cerca de dois metros de
comprimento que havia sido esculpido em um formato arredondado. Luang Pó
pensou que certamente esse pedaço de madeira tinha um dono, pois era uma
madeira nobre e havia sido trabalhada por alguém. Se colocasse aquela
peça no barco, seria como se estivesse roubando o que não lhe pertencia
e, portanto, realizando uma das transgressões mais graves do Vinaya. Por
essa razão, recusou-se até mesmo a tocar aquela madeira. Quando o
trabalho estava terminado e era hora de voltar ao monastério, Luang Ta
Pum viu a madeira no chão e perguntou com irritação:

``Tahn Chah, por que não colocou essa madeira no barco?''

``Achei que não seria correto. Deve ter dono, pois a madeira foi
trabalhada.'', respondeu com educação. Ao ouvir essa resposta, Luang Ta
Pum hesitou por um instante e então mandou os monges subirem no barco,
tentando esconder seu embaraço e abandonando a madeira à margem do rio.

Alguns dias mais tarde um evento similar ocorreu: pessoas do vilarejo
vieram ao monastério para preparar arroz-doce assado no bambu para
oferecer aos monges. Passando pela cozinha, Luang Ta Pum notou que o
fogo estava alto e o bambu onde o arroz estava assando estava começando
a pegar fogo. Não havia ninguém vigiando o fogo e ele deve ter sentido
pena de deixar a comida estragar, mas não conseguia pensar em como
remediar a situação, já que naquele monastério entendiam que era uma
transgressão de Vinaya se um monge tocasse uma comida que ainda não
houvesse sido formalmente oferecida e como punição, nenhum dos monges
poderia consumi-la.

Luang Ta Pum permaneceu indeciso por um instante, então olhou para os
dois lados para ter certeza de que ninguém estava olhando e usou sua mão
para mover o bambu para longe do fogo. O que ele não havia notado é que
Luang Pó estava em sua cabana, que ficava perto da cozinha, e por acaso
viu o ocorrido. Quando chegou a hora da refeição Luang Ta Pum notou que
Luang Pó não estava comendo o arroz e perguntou: ``Tahn Chah, não vai
comer o arroz?'' Luang Pó apenas disse: ``Não.'', mas foi suficiente
para que Luang Ta Pum adivinhasse o que ocorreu e exclamasse em voz
alta: ``Cometi uma transgressão!'' Ao fim da refeição ele veio a Luang
Pó e pediu para realizar a cerimônia de confissão, mas Luang Pó disse:
``Não precisa, apenas seja mais cuidadoso de agora em diante.''

A partir de então, graças a seu bom comportamento, todos os monges do
monastério começaram a respeitar e admirar Ajahn Chah. Luang Ta Pum e os
demais decidiram então remover as regras extras que haviam imposto e
pediram que ele assumisse a posição de monge mais sênior do monastério,
mas Luang Pó recusou dizendo que não seria correto fazer isso: as regras
que haviam sido impostas estavam boas e ele as continuaria obedecendo
até o final do \emph{vassa}.

Os dias se passaram e, quando o fim do \emph{vassa} já estava próximo,
Luang Pó finalmente teve a chance de averiguar a veracidade da
declaração que Luang Ta Pum havia feito no primeiro dia em que se
encontraram, quando disse já ter erradicado toda a raiva de sua mente. A
verdade veio à tona confirmando a sabedoria das palavras do Buddha: ``É
através de convívio que a pureza de \emph{sīla} de uma pessoa pode ser
conhecida, e isso somente após um longo período, não um período curto;
somente para aqueles que são atentos, não para os desatentos; somente
para aqueles munidos de sabedoria, não para os desprovidos de
sabedoria.'' (Thāna Sutta, Anguttara Nikaya, 4.192).

Ao fim do \emph{vassa} fortes chuvas diárias acabaram por causar
inundação por toda área ao redor do monastério. O monastério em si era
localizado ao topo de um pequeno morro e não foi inundado, mas os
aldeões não tinham onde se abrigar e estavam preocupados, porque os bois
e as vacas que criavam não tinham como se alimentar. Sendo o monastério
a única área seca, o gado tinha que ir até as suas cercanias para
procurar grama e folhas para comer e assim conseguir sobreviver. Alguns
acabavam entrando no monastério, e isso desagradava Luang Ta Pum. Ele
mandou que todos os monges ajudassem a tocar o gado para fora, mas após
ter sido expulsa do local, uma pobre vaca, com fome, colocou a cabeça
através da cerca do monastério para comer algumas plantas. Luang Ta Pum
já estava munido de uma vara e começou a bater nela sem piedade. A vaca
mugia com dor e tentava puxar a cabeça para fora da cerca, mas seus
chifres ficaram enganchados no arame. Levou algum tempo para que
conseguisse se desvencilhar e durante todo o período em que esteve presa
teve que aguentar os golpes cheios de ira do velho monge. Luang Pó
estava por perto e assistiu com tristeza todo o evento, lembrando-se de
como há pouco tempo essa mesma pessoa se vangloriava de ter uma mente
livre de raiva.

Durante aqueles últimos dias do \emph{vassa,} Luang Pó se dedicou à
prática da contemplação da morte e costumava ir praticar meditação no
cemitério perto do monastério, sob um pequeno abrigo construído entre as
várias fileiras de túmulos. Era um lugar pacífico e recluso, muito
adequado à prática de meditação e à contemplação da verdade sobre a vida
e a morte. Certo dia os aldeões trouxeram o corpo de um garoto de 13
anos que havia morrido de febre e o enterraram bem ao lado do abrigo
onde Luang Pó costumava se sentar. Alguns dias mais tarde trouxeram mais
um, dessa vez o irmão mais velho daquele mesmo garoto, que havia morrido
da mesma doença. Mais alguns dias e trouxeram o corpo da irmã mais
velha, que também havia sucumbido à enfermidade.

A mãe, o pai e demais parentes dos três irmãos estavam inconsoláveis.
Vendo a situação deles, Luang Pó sentiu ainda mais desencanto com a
realidade do mundo e utilizou aquela experiência para motivar a si mesmo
a não ser negligente. Ele contemplou o sofrimento e pesar que nasce
justamente daquilo que mais amamos e tomamos como mais precioso. A
verdade do Dhamma que ele viu naquele cemitério aumentou seu senso de
urgência em praticar ainda mais, e ele aumentou o tempo que passava
praticando e diminuiu o tempo que descansava. Luang Pó estava
absolutamente resoluto em sua prática e, mesmo quando garoava,
continuava praticando meditação andando sob a chuva.

Certo dia, ele experienciou uma \emph{nimitta}\footnote{Imagens (ou
  sons, ou sensações) que às vezes surgem durante a prática de meditação
  (pāli).} em que viu um homem velho deitado, doente e gemendo de dor,
com a aparência de quem não demoraria muito para morrer. Luang Pó ficou
de pé contemplando aquela cena e então caminhou adiante. Ele então
deparou-se com mais um enfermo que já estava quase morto deitado ao lado
do caminho, seu corpo muito emaciado e sua respiração quase
imperceptível. Luang Pó parou para olhar e então seguiu adiante. Não
muito longe, ele encontrou um cadáver, deitado de costas, os olhos
esbugalhados, a língua e a pele inchada, o corpo coberto de larvas.
Luang Pó teve um senso muito profundo de desapego e, quando emergiu
daquela meditação, a imagem do cadáver continuou vívida em seus olhos e
não desaparecia. Ele sentiu desencanto com a vida e desejou logo escapar
de todo esse sofrimento do \emph{samsāra}.

Após vários dias Luang Pó mudou seu local de prática e começou a
frequentar o topo de um morro das redondezas, mas lá havia o problema da
falta de água. Ele então lembrou-se de um tipo de sapo que hiberna
dentro de um buraco no chão e que sobrevive bebendo apenas sua própria
urina. Luang Pó pensou em experimentar viver desse jeito, mas logo
descobriu que não era possível: após beber sua própria urina várias
vezes em seguida, assim que o líquido atingia seu estômago, era
imediatamente expelido pelo canal urinário. Ele então fez outro
experimento, decidiu fazer jejum. Durante quinze dias ele experimentou
comer uma única refeição a cada dois dias, mas durante todo o período
sentia seu corpo queimando como se estivesse em chamas; havia tanta
agitação em seu corpo e mente que ele não conseguia aguentar.

Tudo isso o fez lembrar das \emph{apannaka patipadā} -- as práticas que
o Buddha declarou nunca estarem erradas:

\begin{itemize}
\item
  \emph{Bhojane mattaññutā}, ter moderação no consumo. Comer na medida
  certa, nem muito, nem pouco.
\item
  \emph{Indrya samvara}, restringir as faculdades dos olhos, ouvidos,
  olfato, paladar, tato e mente para que não sejam sobrecarregadas com
  estímulos que alimentam \emph{kilesas}.
\item
  \emph{Jāgariyānuyoga}, praticar continuamente, com firmeza e energia,
  sem preguiça ou dormindo em excesso.
\end{itemize}

Quando se lembrou desses três ensinamentos do Buddha e tendo observado
que a prática de jejum não era compatível com sua personalidade,
abandonou-a e voltou a comer normalmente, uma vez ao dia. A partir de
então sua prática fez mais progresso e sua mente permanecia livre de
\emph{nīvaranas};\footnote{As cinco obstruções à obtenção dos jhānas:
  Desejo sensual, má vontade, torpor e preguiça, inquietação e
  ansiedade, dúvida (pāli).} sempre que contemplava o Dhamma, tudo lhe
parecia claro como cristal.

Ao chegar o fim do \emph{vassa}, Luang Ta Pum já tinha ganhado grande
admiração por Ajahn Chah graças à sua conduta, humildade e dedicação à
prática do Dhamma. Ele o convidou a atravessar o rio Mekong junto com
ele e os demais monges para visitar monastérios no Laos, mas Luang Pó
educadamente recusou. Sendo assim, ao fim do \emph{vassa}, Luang Ta Pum
e os demais partiram em direção ao Laos e Ajahn Chah permaneceu mais uma
semana sozinho no monastério antes de seguir adiante com suas
peregrinações. Desta vez ele foi em direção a Pu Lanka, em Nakhon Panom,
em busca de solução para um problema que estava enfrentando em sua
prática: seu progresso em \emph{samādhi} lhe parecia obstruído. Ele
relatou em detalhes o que ocorria:

``\ldots{} ia só até ali e então parava, não ia mais adiante. Fazendo
uma comparação, era como se eu andasse até aqui e empacasse, não ia mais
adiante, então voltava\ldots{} Estou falando sobre minha sensação, estou
falando sobre minha mente, entende? Eu ia praticando, mas quando chegava
a este ponto, não havia mais para onde ir, então parava -- às vezes era
desse jeito. Outras vezes era assim: andava até dar de cara com uma
barreira, então voltava. Já outras era como se não houvesse o que me
obstruir, então eu caía.

Continuava praticando meditação sentado e andando, mas cedo ou tarde eu
acabava nesse mesmo ponto. `Que é isso?', eu me perguntava. `Não
importa!', era a resposta que surgia. Seguia em frente por um longo
tempo e então parava; quando tentava avançar mais um pouco, dava
meia-volta e retornava. `O que é isso?' era a questão que não saía da
minha cabeça.

Mesmo quando não estava praticando, essa dúvida persistia e deixava
minha mente confusa. Então pensei: `O que é isso? Isso que acontece no
meu \emph{samādhi\ldots{}} estou empacado nesse \emph{samādhi}.' Mesmo
caminhando, a dúvida persistia `O que é isso?' A questão surgia
frequentemente, porque eu ainda não conseguia resolver aquilo, ainda não
havia chegado ao nível em que se é capaz de largar das coisas, portanto
estava empacado.

Pensei nos monges daquela época e me perguntei quem poderia me ajudar.
Pensei em Ajahn Wang. Ele morava no alto de uma montanha, Pu Lanka, com
dois noviços. Ele vivia lá longe, no alto daquela montanha, eu nunca
havia me encontrado com ele, mas pensei que esse monge devia ter algo de
especial já que conseguia viver daquele jeito\ldots{}''

Luang Pó subiu a montanha e permaneceu por três noites para poder contar
o problema que havia encontrado e pedir ajuda a Ajahn Wang. Anos mais
tarde ele relatou um resumo da conversa que tiveram naquela ocasião:

(Ajahn Wang) -- Uma vez, quando estava praticando meditação andando,
parei e foquei minha mente em meu corpo e ele afundou no chão de uma só
vez.

(Ajahn Chah) -- O senhor estava ciente de si?

-- Sim, por que não estaria? Ele afundou, foquei minha mente e ele
continuou afundando, foquei em apenas estar ciente dele afundando e
deixei acontecer. Foi afundando até chegar ao limite -- onde era esse
limite eu não sei, mas tendo chegado lá começou a subir. Subiu de vez. O
chão fica aqui, mas meu corpo subiu lá em cima, não ficou no chão. Um
pouco mais e subiu ainda mais longe e foi lá no alto. Eu estava ciente
de tudo. Estava estupefato: `Como isso é possível?' Ele então continuou
subindo até chegar ao topo das árvores e explodiu. Bum! Meus intestinos,
grosso e delgado, ficaram pendurados, enroscados nos galhos das árvores.

-- Tahn Ajahn, não foi um sonho?

-- Não foi sonho! Se não estivermos atentos, esse tipo de coisa pode nos
fazer perder o equilíbrio. Foi de verdade daquele jeito. Naquele momento
eu acreditava que aquilo era de verdade, mesmo agora eu ainda penso que
foi mesmo daquele jeito -- chega a esse ponto! Nem precisa me dizer
sobre \emph{nimitta} alguma, é possível chegar mesmo a esse ponto que
acabei de relatar. Se seu corpo explodisse -- bum! -- como ia se sentir,
os intestinos enroscados nos galhos daquele jeito? É uma sensação muito
forte, mas entendemos que isso é uma \emph{nimitta}. Ficamos firmes,
acreditamos que não há perigo algum. Sendo assim, olhamos para dentro,
olhamos até chegar à mente. Ela permanece assim por um instante e então
desaparece. Então a gente senta e se pergunta: `Que diabos foi isso?'

-- Eu vim prestar reverência a Tahn Ajahn porque já não sei mais o que
fazer. Comigo não acontece como o senhor descreveu, é de outro jeito. É
como se andasse sobre uma ponte; é como se essa ponte fosse alta, sobre
um rio. Vou andando e então paro, não tem mais para onde ir e então, o
que fazer? É assim, dou meia-volta e retorno. Às vezes ando até lá
novamente -- isso acontece dentro de \emph{samādhi} -- chego ao mesmo
ponto e acaba, pois não há mais para onde ir, então volto. Tento ir
(mais adiante), mas não consigo. Às vezes vou e é como se algo me
obstruísse e bato de cara -- pum! -- e não tenho mais para onde ir. Isso
já vem acontecendo há muito tempo. O que é, Tahn Ajahn?

-- Isso é o fim, é o limite de \emph{saññā}. Sendo assim, para onde
poderia ir? Fique de pé bem ali! Foque sua mente bem ali, se ficarmos de
pé bem ali, \emph{saññā} se desfaz. Ela se transformará sozinha, você
não precisa forçá-la. Olhe com a atitude: `Isto é assim, e sendo assim,
sinto algo acontecendo com minha mente?' Saiba que é daquele jeito,
esteja ciente. Se você permanecer ciente, num minuto ela se
transformará, mudará de \emph{saññā}. É como se fossem \emph{saññā} de
criança e \emph{saññā} de adulto -- ela se transforma em \emph{saññā} de
adulto. Por exemplo, as crianças gostam de brincar com um tipo de coisa,
mas quando crescem passam a enxergar que essas coisas já não são mais
divertidas e vão brincar com outras coisas, pois houve uma mudança.

-- Oh! Entendi.

-- Não fale muito, é um assunto que nunca se encerra. Entenda que, no
que diz respeito a \emph{samādhi}, tudo é possível. Mesmo assim, não dê
importância, não fique criando dúvidas a respeito. Se tivermos essa
sensação, não demora muito e aquilo perde valor -- é apenas \emph{citta
sankhāra}, não tem nada de mais. Se formos olhar, ela se transforma num
pato, então o pato se transforma numa galinha, se olharmos para a
galinha ela se transforma num cachorro, olhamos para o cachorro e ele se
transforma num porco, e então ficamos nessa confusão sem fim. Esteja
ciente daquilo, foque bem aqui, mas não pense que já acabou -- logo ela
vem novamente, mas não agarre -- tome ciência em sua mente e deixe
passar, continuamente. Dessa forma, não haverá perigo algum. Foque dessa
forma para que haja uma fundação, não fique correndo atrás dessas
coisas. Se conseguir resolver isso, vai poder prosseguir, haverá uma
abertura para passar. Passado e futuro também serão dessa forma, talvez
mais forte ou fraco; mas mesmo que sejam fantásticos ou maravilhosos,
não importa a que ponto, terá que ser daquele jeito. Entenda dessa
forma, de verdade.

-- Por que algumas pessoas não têm problemas (em suas práticas) e não
sofrem? Não têm coisas obstruindo; têm bem-estar no corpo e mente; não
têm problemas?

-- Isso é por causa do seu \emph{kamma}. Em horas como essa você tem que
lutar, a mente se unifica e há uma briga pelo trono aqui dentro. As
coisas com as quais brigamos não são apenas coisas ruins, também temos
que lutar com coisas boas e adoráveis -- todas são perigosas, não se
apegue a nenhuma delas.''

Após aquela conversa com Ajahn Wang, a compreensão sobre o caminho de
prática de Luang Pó aumentou consideravelmente e, tendo consultado-o
sobre diversos aspectos do Dhamma, ele então se despediu e retornou ao
local onde estava hospedado. Durante sua estadia em Pu Lanka, ele
acelerou sua prática ainda mais, descansando muito pouco e não mais
diferenciando dia e noite -- ele apenas praticava continuamente.
Passados três dias, Luang Pó se despediu de Ajahn Wang.

``Eu desci de Pu Lanka e cheguei a um monastério ao pé da montanha. Bem
naquela hora começou a chover e fui me sentar no salão para fugir da
chuva. Minha mente contemplava tudo aquilo, e de repente ela se firmou e
então mudou. Eu sentia como se estivesse em outro mundo: tudo que olhava
estava transformado. Quando olhava o bule d'água\footnote{Na época não
  havia garrafas plásticas; cada monge tinha um bule de metal que enchia
  de água e levava consigo para beber.} sentia que não era um bule
d'água. O cesto de lixo mudou, a tigela mudou, tudo mudou de natureza.
Eram completamente diferentes, como a palma e as costas da mão; como o
sol saindo detrás das nuvens que obstruíam sua luz. Minha mente mudou
num piscar de olhos, se firmou e então mudou. Via a garrafa, mas não era
uma garrafa; olhava e não era nada, eram apenas elementos, uma convenção
fabricada, não era uma garrafa de verdade, não era um cesto de lixo de
verdade, não era um copo de verdade, tudo mudou. Ia mudando e mudando e
então voltei minha atenção para mim mesmo e vi tudo em meu corpo não
sendo meu -- eram apenas convenções.''

Concluindo esse relato, Luang Pó disse a seus discípulos: ``\ldots{} não
vacilem em sua prática, dediquem-se de corpo e alma. Tenham firmeza em
suas mentes, pratiquem! Não importa onde vocês ouvem ensinamentos ou
estudam livros; aquilo é conhecimento, mas não é um conhecimento que
alcança a verdade. Se conhecem, mas ainda não alcançam a verdade, vão
continuar vacilando e em dúvida. Se conhecem de verdade, o assunto se
encerra. Podem dizer o que quiserem, a verdade continua sendo daquele
jeito, é certa daquele jeito. Quando nossa mente volta ao normal, não
importa se rirem, chorarem, ficarem felizes ou tristes -- ela não mais
vacila perante qualquer coisa neste mundo.''

E sobre a importância de ter acesso a bons mestres, ele uma vez disse:
``\ldots{} É possível praticar sozinho, mas para alguns isso vai
resultar em progresso lento, ficarão andando em círculos. Isso é algo
com o qual fazemos contato com nossas mentes; por isso, se há quem nos
indique o caminho, vamos mais rápido, pois teremos uma abertura para
contemplar. Todas as pessoas são assim: quando atolam, atolam de
verdade\ldots{}''

Após deixar Pu Lanka, Luang Pó foi até Wat Pah Nong Hi prestar
reverência a Luang Pu Kinari e naquele encontro Luang Pu lhe disse:

``Tahn Chah, já viveu em peregrinação tempo suficiente. Você deveria
agora procurar um local para tomar residência, numa área plana.''

Luang Pó então disse: ``Eu decidi voltar para minha terra natal, em
Ubon.''

Luang Pu Kinari respondeu com um curto alerta: ``Está voltando para casa
porque está com saudades de alguém? Se sente saudades de alguém, por
causa daquela pessoa, máculas surgirão em você.''
