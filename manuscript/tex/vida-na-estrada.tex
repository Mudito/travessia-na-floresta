\chapter{Vida na estrada}

Em 1946 Ajahn Chah começou sua vida na estrada, tendo um monge chamado
Tawan como companhia. Caminharam em direção à região central da
Tailândia, passando pelas montanhas de Dong Phaya Yen e chegando a Ban
Yang Yu, em Sara Buri. Após uma curta estada nesse local, decidiram que
já haviam caminhado sem destino por tempo suficiente e que agora seria
melhor se começassem a procurar um local apropriado para praticar, onde
houvesse um bom professor de meditação.

Caminharam até Lop Buri e chegaram a Wat Khao Wong Kot, monastério de
Luang Pó Pao, mas infelizmente Luang Pó Pao já havia falecido. No lugar
dele encontraram Ajahn Wanna, um discípulo de Luang Pó Pao que então
cumpria a tarefa de cuidar do monastério e ensinar Dhamma. Graças a ele,
ambos os monges tiveram a oportunidade de estudar o modo de prática
deixado por Luang Pó Pao e também ler as pequenas frases de Dhamma que
ele escrevera na entrada de cavernas e outros locais pelo monastério,
para alertar e encorajar os praticantes -- muitos anos mais tarde, Luang
Pó Chah seguiria esse exemplo e faria o mesmo em seu monastério, Wat
Nong Pah Pong.

Durante sua estadia, Ajahn Chah também teve a oportunidade de estudar em
detalhe a regra monástica e ganhar uma compreensão mais profunda dela,
através do Visuddhimagga e Pubbasikkhā Vannanā\footnote{Dois antigos
  comentários sobre os ensinamentos do Buddha (pāli).}, mas também
recebeu instrução de um monge do Camboja que possuía profundo
conhecimento tanto das escrituras budistas como da prática de meditação.
Ele havia vindo à Tailândia buscando estudar a edição tailandesa do
Tipitaka e possuía uma capacidade excepcional para lembrar-se de todas
as regras do Vinaya\footnote{A regra monástica criada pelo Buddha
  (pāli).} em seus mínimos detalhes. Além disso, apesar de ser um perito
nas escrituras, vivia a vida de um monge \emph{kammatthāna} -- gostava
de viver de forma simples e em locais afastados, em florestas e
montanhas.

Certo dia, Luang Pó havia estudado várias regras de Vinaya com esse
monge, mas por engano o monge lhe explicou uma delas de forma
equivocada. Naquela época Luang Pó tinha adquirido o hábito de toda
noite subir um morro para praticar meditação num ambiente mais recluso.
Terminada a lição daquele dia e tendo ajudado com a limpeza do
monastério, Luang Pó voltou ao seu local de prática no alto do morro. Já
bem tarde da noite, por volta das dez horas, enquanto praticava
meditação, Luang Pó ouviu o som de passos sobre gravetos e folhas secas
se aproximando, mais e mais. Ele pensou que talvez fosse algum animal
procurando por comida, mas quando o som ficou cada vez mais alto e
próximo ele pôde notar que na verdade se tratava daquele monge do
Camboja. Luang Pó perguntou surpreso:

``Tahn Ajahn, o que o traz aqui a uma hora dessas da noite?''

``Eu lhe expliquei errado uma regra de Vinaya.''

``Tahn Ajahn não deveria se incomodar tanto, você sequer tem uma
lanterna! Não poderia esperar e me dizer amanhã pela manhã?''

``Não, não! Se eu morresse hoje à noite, você no futuro também ensinaria
errado a outras pessoas. Isso seria um \emph{kamma} ruim
desnecessário.'' Ele então explicou novamente a tal regra monástica de
forma correta. Tendo terminado sua explicação, deu meia-volta e desceu o
morro.

Esse encontro fez Luang Pó sentir muita admiração pela bondade daquele
monge e também por sua capacidade de se preocupar tão sinceramente com
os demais: ele não negligenciava nem um pequeno engano, não admitia
esperar até o dia seguinte para corrigir seu erro. Luang Pó sentiu que
esse era um exemplo a ser emulado, algo realmente admirável e digno de
elogio.

Na época em que viva em Wat Khao Wong Kot, Ajahn Chah ainda não era
muito hábil na prática de meditação, e durante sua estadia experimentou
com vários métodos. Certo dia lembrou-se da ocasião em que ainda era um
noviço em Wat Ban Kó Nók e viu um monge \emph{kammatthāna} com um
\emph{māla}\footnote{Um cordão de contas, similar ao rosário comum na
  religião católica. Tradicionalmente um māla possui 108 contas e na
  maior parte das vezes é utilizado para auxiliar a marcar a contagem do
  número de recitações de uma frase utilizada como objeto de meditação
  (pāli).} pendurado no pescoço. Pensou que também gostaria de ter um
\emph{māla} para experimentar praticar utilizando-o, mas não havia como
obtê-lo. Ele então notou que a árvore chamada de \emph{tabék} na
Tailândia possui sementes grandes e redondas que poderiam ser utilizadas
como contas de um \emph{māla}, mas Luang Pó não se atrevia a colhê-las,
pois isso seria uma transgressão de Vinaya (a regra proíbe os monges de
danificarem plantas).

Um dia, um grupo de macacos passou pelo monastério e começou a quebrar
os galhos de uma dessas árvores e arrancar as sementes. Luang Pó pegou
um punhado delas do chão, mas não tinha um cordão para amarrá-las, então
simplesmente segurava as sementes numa mão e, toda vez que terminava a
recitação de um \emph{parikamma}\footnote{Em geral, uma frase recitada
  continuamente para servir como objeto de meditação (pāli).}\emph{,}
colocava uma semente dentro de uma lata até completar 108. Ele praticou
dessa forma por três dias, até decidir que ela não era compatível com
sua personalidade e resolver parar. Luang Pó também experimentou com a
prática de \emph{ānāpānasati}:

``Mesmo quando tinha dúvidas sobre o que era \emph{samādhi}, eu ia
pensando, praticando meditação, e minha mente ficava ainda mais confusa,
pensava ainda mais. Mas quando não estava praticando, não era tão ruim
assim\ldots{} Puxa, era muito difícil! Mas mesmo assim eu não parava, ia
praticando daquele jeito mesmo. Quando estava à toa era tranquilo, mas
quando tentava fazer a mente unificar-se, era ainda pior. `Por que isso?
Por que é desse jeito?' Em seguida pensei: `Será que não é como minha
respiração? Se determinar para que ela seja curta, longa ou média, fica
muito difícil. Mas, quando estou andando, sequer noto o ar entrando e
saindo, e nessa hora não há dificuldade alguma.' Então percebi: `Oh!
Deve ser isso: quando estou apenas caminhando, não forço a respiração.
Normalmente, alguém já sofreu por causa da respiração? Ninguém. É muito
fácil. Mas se nos sentamos e a forçamos a se pacificar, isso é adicionar
desejo e apego. Queremos que a respiração seja curta ou longa, e então
não acontece como queríamos. A mente sofre ainda mais que antes. Por
quê? Porque nossa intenção se torna desejo e apego, por isso não dá em
nada; é difícil porque adicionamos desejo ao processo.'''

Em 1946, durante sua estadia em Wat Khao Wong Kot, outro evento
inusitado fez com que Luang Pó ganhasse uma compreensão maior sobre o
caminho do Dhamma. Certa noite, após ter terminado sua prática de
meditação sentada e andando, Luang Pó preparou-se para dormir. Mas, por
ter sempre tido muito medo de fantasmas, e apesar de ser corajoso o
suficiente para passar a noite no alto daquele morro sozinho, ele sempre
sentia a necessidade de recitar encantos e mantras antes de dormir, pois
acreditava que ajudariam a protegê-lo de fantasmas e maus espíritos.
Porém, naquela noite, estava seguro de que seu comportamento e
observância da regra monástica eram impecáveis e que por essa razão não
tinha nada a temer e decidiu então que não recitaria os mantras
costumeiros.

Quando estava a ponto de cair no sono, ele sentiu algo apertando seu
pescoço cada vez mais forte, até quase não conseguir mais respirar. É
impossível dizer ao certo se isso foi resultado do apego psicológico que
ele tinha em recitar mantras antes de dormir -- e não tê-lo feito
naquela noite -- ou se de fato se tratava de um evento paranormal, mas
Luang Pó ainda teve presença mental suficiente para começar a recitar
mentalmente o \emph{parikamma} ``Buddho'' até que o aperto ao redor do
seu pescoço começasse a enfraquecer. Após algum tempo conseguiu abrir os
olhos, mas ainda não podia mover seu corpo, e ainda assim continuou
recitando ``Buddho, Buddho\ldots{}'' mentalmente até conseguir mover
seus membros. Após um período ainda não conseguia se levantar, mas
conseguia passar a mão sobre seu corpo e nesse momento sentiu como se
não fosse si mesmo. Ele continuou recitando ``Buddho'' até que
conseguisse se levantar.

Luang Pó afirmou estar certo de que só conseguiu sobreviver àquilo
graças ao \emph{parikamma} ``Buddho'' e concluiu que esse evento também
estava relacionado com sua pureza em \emph{sīla} e que esse tipo de
fenômeno só apresenta perigo para aqueles desprovidos de \emph{sīla.}
Desse ponto em diante ele decidiu que mantras mágicos não eram
necessários, eram apenas superstições, e que o que é realmente
importante é desenvolver sua conduta e sua mente em linha com os
princípios do Dhamma.

A partir daquele dia ele passou a ter ainda mais cuidado em resguardar a
pureza de sua conduta, sempre respeitando os princípios de \emph{sīla.}
Renunciou à posse de dinheiro e a todos os itens que possuía que haviam
sido obtidos de formas proibidas pelo Vinaya\footnote{Uma vez que a
  posse de dinheiro é ilícita a um bhikkhu, a regra monástica também
  proíbe aos monges possuírem ou fazerem uso de quaisquer itens
  adquiridos com aquele dinheiro. Além disso, existem limites para o que
  e quando um monge pode pedir aos leigos. Qualquer item obtido em
  detrimento a esses limites também é inapropriado à posse de um monge.}.
Luang Pó decidiu-se a não mais receber qualquer coisa que não estivesse
de acordo com a regra monástica: ele não mais quebraria a regra ou se
envolveria em qualquer atividade do tipo. Antes de se tornar um monge
\emph{kammatthāna,} Luang Pó ainda possuía dinheiro e fazia uso dele.
Ele contou como foi o momento em que tomou a decisão de renunciar ao
dinheiro por completo:

``No que diz respeito a Vinaya, se a pessoa não enxergar em seu coração,
é muito difícil. Muitos anos antes de vir morar em Wat Nong Pah Pong eu
decidi abandonar o uso de dinheiro. Eu passei o \emph{vassa} inteiro
pensando a respeito, mas não conseguia tomar uma resolução. No final eu
simplesmente peguei minha carteira e fui procurar um certo
``mahā''\footnote{Um título dado a monges que alcançam uma certa
  graduação em estudos da língua pāli.} que costumava viajar comigo, mas
agora mora em Wat Rakan. Eu joguei minha carteira para ele e disse:

`Aqui, Mahā, seja minha testemunha sobre este dinheiro. Deste dia em
diante eu não vou mais pegar, não vou mais tocar (dinheiro) -- exceto
caso largue a vida monástica. Seja minha testemunha nisto.'

Tahn Mahā não queria pegar a carteira\ldots{} estava com vergonha:
`Guarde! Use para pagar seus estudos. Tahn Ajahn, por que você está
jogando fora centenas (de bahts) desta forma?', ele estava envergonhado.

`Não se preocupe comigo, ontem à noite eu tomei minha decisão, já me
decidi.'

Daquele momento, quando ele pegou o dinheiro, em diante, foi como se não
mais nos conhecêssemos. Se conversássemos, não nos entendíamos. Mas
ainda hoje ele é minha testemunha -- eu nunca fiz novamente, nunca mais
utilizei dinheiro, nunca permutei, nunca mais fiz nada do tipo.

\ldots{} Se seguimos essa regra, desenvolvemos \emph{pāramī}. Os leigos
nos veem e sentem admiração, sentem-se inclinados a nos ajudar. O
importante é não pedir. Devemos estar prontos a passar sem, em qualquer
situação. Essa é uma regra que ajuda a desenvolver frugalidade como uma
fundação para nossa vida.''

Obedecer à regra monástica nutriu a fé de Luang Pó, mas também lhe
trouxe desafios:

``Eu era tão tolo que uma vez quase larguei o manto quando vi os
defeitos nas minhas ações, na minha prática, nos meus professores e em
tudo mais. Era como algo me queimando, não conseguia dormir, era um
problema sério, minha cabeça estava carregada de dúvidas. Mas quanto
mais duvidava, mais eu praticava, mais me esforçava. Não importa o
quanto tivesse dúvidas, continuava praticando. Bem ali surgiu sabedoria
e tudo foi mudando. Antes, não sabia nada sobre as transgressões mais
leves e não queria ouvir falar delas. Quando comecei a compreender o
Dhamma de verdade, esse modo de prática, as transgressões mais leves se
tornaram tão importantes como as mais pesadas.''
