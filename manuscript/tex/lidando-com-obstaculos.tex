\thistitleoffsettrue
\chapter{Lidando com obstáculos}

Falta de recursos e doenças não eram os únicos obstáculos que Luang Pó e
seus discípulos tiveram que enfrentar. Apesar de estar estabelecendo um
monastério em sua cidade natal, nos primeiros anos Ajahn Chah teve que
lidar como todo tipo de problemas vindos de falta de comunicação,
mal-entendidos e de pessoas da região imaginado coisas sozinhas, algo
que é normal e ocorria tanto entre os leigos como entre a sangha
monástica. Por exemplo, esta história que Luang Pó contou à comunidade
leiga de um monastério que foi visitar:

``\ldots{} Eu fui viver em Wat Pah Pong -- imaginem, no meu próprio
vilarejo! -- e fiz um bom trabalho cavando um poço na floresta. Pus um
aviso que dizia `Este poço é para uso exclusivo dos monges. Os leigos
não estão autorizados a utilizá-lo.' Ainda assim, nós tirávamos água e
colocávamos num jarro grande para que os leigos pudessem usar. Mas as
pessoas ainda reclamaram -- o pior foi Pó Nu. Quando uma mulher lhe
contou sobre o poço, ele prontamente exclamou: `Então não devemos beber
água?!' Ele foi com força total: `O quê?! De onde saíram esses monges?
Por que decidiram que os leigos não podem beber água? Eles se deram a
todo esse trabalho de estudar e virar monges e ainda se comportam desse
jeito?! Eles por acaso não comem o que oferecemos como esmola ou de onde
é que estão tirando comida, a ponto de decidirem que leigos não podem
beber água?'

Vejam! Isso é por não saber. Nós deixávamos eles beberem, mas
normalmente os leigos que vinham ao monastério usavam o balde com o qual
pegavam sapos, peixes, girinos e sabe-se lá mais o que, e usavam aquele
mesmo balde para tirar água do poço, deixando-o sujo. Eu apenas não
queria que fizessem isso, eu queria que bebessem, é por isso que
tirávamos água e colocávamos num jarro à parte para eles. Mas a única
coisa que diziam era: `Ele nos proibiu de beber água!', eles não
contavam a história inteira.

Pó Nu ouviu sobre o poço e ficou com raiva, saiu de casa imediatamente
para expulsar os monges: `Que tipo de monges são esses? De onde eles
saíram? Eles vivem às custas dos leigos ou o quê?!' Ele ouviu a história
apenas uma vez, mas os pensamentos dele já foram lá na frente. Levou um
bom tempo até conseguirmos nos entender, mas quando o fizemos, tornou-se
meu discípulo\ldots{}''

De fato levou um bom tempo. Após a disputa sobre o poço, a raiva de Pó
Nu diminuiu um pouco, mas ele não desistiu. De vez em quando vinha
discutir com Luang Pó e, em um desses encontros, Ajahn Chah lhe fez um
convite aberto:

``Se tem algo que queira dizer, fale abertamente, não se preocupe
comigo. Desse jeito vamos nos entender de uma vez por todas.''

``Se eu falar de forma aberta, temo que Luang Pó vá ficar com raiva de
mim, porque não falo da mesma maneira que as demais pessoas.''

``Pode falar, não vou ficar bravo.''

Na verdade, Pó Nu estava esperando justamente uma oportunidade como essa
e pensou ``Ele me deu permissão para falar abertamente. Agora, se eu
falar de maneira bem incisiva e ele ficar com raiva, vou poder acusá-lo
e dizer: `Que espécie de monge \emph{kammatthāna} é você que não
consegue aguentar nada?'\thinspace ''

Então Pó Nu começou seu argumento: ``Luang Pó uma vez disse que estou no
caminho errado, mas eu digo que é você quem está no caminho errado, mais
do que eu, porque todas as religiões são falsas, elas são apenas coisas
que as pessoas inventam, convenções inventadas para enganar as pessoas.
Todas as coisas neste mundo são apenas convenções. Por exemplo, um
búfalo: inventamos um nome para ele, mas se quiséssemos poderíamos
chamá-lo de `porco', ele não iria reclamar\ldots{} `Pessoa', `animal',
tudo isso são apenas convenções, mesmo as religiões são apenas
convenções.

Por que você tem tanto medo de demérito e mau \emph{kamma}, a ponto de
ter que se esconder na floresta? Isso nada mais é que atormentar seu
próprio corpo inutilmente. Eu não acredito que mérito e demérito existam
realmente: eles provavelmente são histórias para enganar crianças. Vá
embora e viva num monastério na cidade como os outros monges fazem! Ou,
melhor ainda, largue a vida monástica -- assim você vai sentir prazer e
experimentar o gosto da sensualidade. Isso seria melhor do que viver
atormentando a si mesmo dessa forma -- eu não enxergo benefício algum.
Eu penso assim. Qual sua opinião sobre tudo isso?''

Após deixar Pó Nu falar à vontade para fisgar o interesse dele e após
ter ouvido em silêncio por um bom tempo, Luang Pó disse: ``O Buddha
descartava pessoas que pensam como você, ele não conseguia ajudá-las.
Elas são como flores de lótus que nascem afundadas na lama\ldots{} Se
você não acredita que mérito e demérito realmente existam, por que não
tenta sair por aí roubando, furtando ou matando as pessoas só para ver o
que acontece? Você vai ver como é.''

``O quê? Como você pode me dizer para matar pessoas? Se eu fizesse isso,
os parentes deles viriam atrás de mim para me matar ou eu acabaria na
cadeia!''

``Pois é. Você não vê que esse é o resultado do demérito?''

Pó Nu vacilou quando posto diante dessa linha de raciocínio, mas para
não dar o braço a torcer, insistiu: ``Talvez demérito exista, mas e
mérito? Luang Pó vive na floresta atormentando a si dessa forma, mas eu
não vejo mérito algum nisso.''

``Você pode pensar dessa forma se quiser, mas eu vou fazer uma
comparação: se não há mérito e demérito e pratico de acordo com o
\emph{Dhamma-Vinaya}, não perco nada. Mas se de fato há mérito e
demérito, então estou levando vantagem. Qual dos dois está em melhor
situação: uma pessoa que está levando vantagem ou nem ganhando nem
perdendo, ou uma pessoa que só está perdendo?

Pó Nu começou a abrir sua mente um pouco e falou sobre seus verdadeiros
sentimentos sobre esse assunto: ``Essa história de mérito e demérito é
algo sobre o qual penso a respeito já faz muito tempo, a ponto de ficar
com dor de cabeça. Eu ainda não tenho certeza se acredito. Nenhum monge
a quem perguntei conseguiu me explicar de uma forma que me fizesse
entender. Mas consigo aceitar um pouco do raciocínio que você usou. Mas,
supondo que mérito e demérito não existam, o que devo fazer?''

``Faça o que quiser.''

``Se não existir de verdade, após eu morrer vou voltar para te dar um
chute, pode ser?''

``Pode! Mas uma coisa importante é que você tem que abandonar o demérito
e realmente praticar o mérito se quiser saber se mérito e demérito
realmente existem.''

``Nesse caso, que Luang Pó me ensine para que eu saiba como praticar
mérito.''

``Não é possível ajudar uma pessoa com pontos de vista incorretos como
você. É uma perda de tempo.''

Quando pressionado dessa forma por Ajahn Chah, Pó Nu respondeu com
eloquência: ``Quem neste mundo é o mais sábio de todos? O~Buddha, não é?
Você é um discípulo do Buddha, eu sou um discípulo de Māra.\footnote{Uma
  personagem que surge frequentemente nas escrituras budistas tentando
  persuadir o Buddha e seus discípulos a desistirem da prática do Dhamma
  e voltar a viver no mundo da sensualidade (pāli).} Se o Buddha é
realmente tão especial, mas você é incapaz de me ensinar, mostra que na
verdade o Buddha não é superior a Māra.''

``Se é mesmo isso que você quer, então preste atenção, vou lhe dar um
pouco de Dhamma e você pode levar, contemplar e tentar praticar para ver
os resultados. Não importa se você acredita ou não, faça primeiro e veja
o que acontece. Então, você não acredita no que os monges lhe disseram
sobre o \emph{Buddha Sāsanā}, correto?''

``Sim, eu não quero acreditar em ninguém a não ser em mim mesmo.''

``Se você não quer acreditar nos outros, também não deve acreditar em si
mesmo. Você não é carpinteiro? Você alguma vez já cortou madeira
errado?''

``Sim, já ocorreu.''

``É a mesma coisa. Você sempre entende as coisas certo, faz tudo certo
ou às vezes entende mal, faz errado?''

``Sim, eu já fiz coisas de maneira correta e errada.''

``Nesse caso você ainda não pode confiar em si mesmo, porque você ainda
se leva a fazer errado, falar errado, pensar errado\ldots{}''

Pó Nu ficou satisfeito com essa linha de raciocínio e pediu por
conselho: ``Como devo agir?''

``Não fique duvidando demais e também não fale muito.''\footnote{Estas são duas
  expressões tailandesas que possuem ambos um significado de face, imediatamente
  percebível, e também um mais sutil. A primeira é uma admoestação para que Pó
  Nu não fique pensando demais, tentando adivinhar o significado do que não
  sabe, ou, como dizem no Brasil: `ficar criando minhocas na cabeça'. A segunda,
  além de recomendar falar menos, também indica a necessidade de humildade --
  não ficar falando sobre algo sobre o qual não temos real conhecimento.}

Após essa ocasião, onde quer que Pó Nu estivesse, esse conselho de Luang
Pó Chah sempre ia junto. Mesmo que tentasse não pensar a respeito, ele
não conseguia evitá-lo e quanto mais pensava a respeito, mais via a
verdade no ensinamento de Ajahn Chah e do Buddha: ``Tudo que o Buddha
ensinou está de fato correto e não é possível contradizê-lo.''

Mais tarde ele abandonou seu orgulho e arrogância, abandonou seu
comportamento teimoso, tornou-se discípulo de Luang Pó e começou a
observar os cinco preceitos com diligência, assim como praticar
meditação. No final, se tornou uma nova pessoa, agradável e sempre
prestativo em todas as atividades do monastério, com energia e
dedicação.

Outro problema que tiveram que enfrentar era o de pessoas que só queriam
levar vantagem e obter o que desejavam. Algumas viam verduras e frutas
crescendo na floresta do monastério e queriam colhê-los. No começo
pediam uma ou duas; mais tarde começaram a roubar algumas e no final a
ganância os sobrepujou e começaram a vir ao monastério carregando cestos
para encher de frutas. Uns dos alvos preferidos eram os diversos pés de
mamão do monastério, e inicialmente Luang Pó não fez nada, mas como os
roubos começaram a ficar maiores e mais frequentes, ele pensou em uma
estratégia para acabar com esse problema de uma vez por todas.

Luang Pó mandou que um anagārika cortasse três moitas de espinhos e as
escondesse ao lado do caminho. Normalmente os ladrões vinham por volta
das sete ou oito da noite, que era o horário em que os monges estavam
reunidos no salão principal realizando a \emph{pūja} vespertina. Naquela
noite, Luang Pó formou três grupos de monges: ele permaneceu com o
primeiro perto do local onde os ladrões pegavam o mamão; o segundo ficou
colocado a meio caminho e o terceiro perto de onde as moitas de espinhos
estavam escondidas.

Quando chegou o horário costumeiro, os ladrões chegaram pelo caminho que
normalmente usavam, carregando seus cestos. Enquanto colhiam as frutas,
Luang Pó deu o sinal para que os espinhos fossem retirados do
esconderijo e fossem postos obstruindo o caminho que os ladrões usariam
ao partir. Quando os cestos dos ladrões estavam quase cheios, Luang Pó
limpou sua garganta, fazendo um barulho não muito alto porque não queria
assustar demais os ladrões por medo de que eles saíssem correndo e
deixassem para trás o produto do roubo. Quando ouviram o som, os ladrões
se assustaram e se apressaram em colocar os cestos aos ombros e, meio
andando, meio correndo, sem saberem o que os aguardava, foram direto ao
local onde o segundo grupo de monges estava esperando. Nesse momento
Luang Pó perguntou em voz alta:

``Vocês viram alguém indo naquela direção?''

``Onde? Onde? Cadê eles?'', os monges responderam falando alto e de
forma ameaçadora.

Os ladrões ouviram aquele barulho e ficaram ainda mais assustados.
Começaram a correr pela floresta até que chegaram ao terceiro ponto, e
os monges escondidos ali começaram a fazer barulho ao mesmo tempo em que
o primeiro e segundo grupo chegavam por detrás, e o som de vozes ficou
ainda mais alto, preenchendo a floresta inteira. Os ladrões ficaram
brancos de medo, largaram seus cestos e correram direto para as moitas
de espinhos que obstruíam o caminho. O primeiro deve ter se machucado
mais do que o segundo, porque após ele mesmo ter caído de corpo inteiro
em meio aos espinhos, ainda teve que aguentar o peso de seu parceiro que
caiu por cima dele logo em seguida, empurrando-o ainda mais para dentro
dos espinhos. Ambos ficaram completamente enroscados nas moitas,
gritando de dor, mas graças ao desespero em que estavam, ainda
conseguiram atravessar os galhos repletos de espinhos e desaparecer na
noite.

Luang Pó mandou que o anagārika colhesse os cestos. Ele sabia que seria
fácil achar os ladrões, porque eles teriam que ficar de cama e pedir que
seus familiares removessem os espinhos de seus corpos, um por um, tarefa
que levaria no mínimo três dias para completar. Luang Pó chamou o chefe
do vilarejo e disse: ``Diga a eles que venham coletar seus cestos e as
frutas; eles os perderam aqui. Diga que não precisam ter medo, podem vir
e recolher os pertences deles comigo. Se demorarem muito, as frutas vão
estragar e eles não vão conseguir vendê-las.''

Quando um dos ladrões veio se encontrar com Luang Pó, ao invés de lhe
dar uma bronca, ele lhe ensinou com bondade: ``Não faça isso novamente.
Essas frutas que você roubou conseguem alimentá-lo por no máximo um ou
dois dias. Vá procurar uma profissão honesta e dessa forma você será um
bom exemplo para seus filhos.'', e desse dia em diante não houve mais
roubos.

Luang Pó também enfrentou problemas porque parte dos moradores da região
costumava se beneficiar da floresta cortando lenha, caçando e usando
como pasto para o gado, mas agora que ela havia sido transformada em
monastério, eles não podiam mais fazer nada disso. Alguns decidiram
intimidar os monges utilizando uma mulher para difamá-los. Quando o
monge que normalmente caminhava em \emph{pindapāta} naquela área ouviu
as notícias, ficou preocupado e consultou Ajahn Chah, que então se
voluntariou a ir em vez dele. Os leigos desaconselharam, dizendo: ``Tahn
Ajahn, não vá em \emph{pindapāta} sozinho, eles vão arrumar uma garota
para abraçá-lo e então acusá-lo de tê-la estuprado.'' Mas Luang Pó não
estava preocupado, ele apenas disse: ``Deixa ela vir! Desde que nasci
nunca abracei uma mulher. Diga para ela vir, vou gostar muito!'' Mas no
final nada aconteceu e o murmúrio cessou sozinho. Desde então a
atmosfera começou a melhorar e o monastério ganhou a simpatia de todos
os moradores da região.

Mais um exemplo do modo ao mesmo tempo gentil e firme com que Luang Pó
lidava com conflitos foi a ocasião em que uma pessoa, querendo se livrar
de um cachorro, abandonou-o no monastério. Luang Pó resolveu o problema
explicando de forma racional e evitando criar ressentimentos: ``Um
monastério de floresta não gosta de cães e gatos. Nós já temos muitos
animais aqui, como esquilos e galinhas selvagens. Se eu estiver aqui,
essas espécies terão continuidade; caso contrário, elas desaparecerão.
Eu quero deixar algumas para que as próximas gerações possam ver. Se
você trouxer cães para abandonar aqui, eles irão matar esses animais.
Não faça novamente. Quem for o dono daquele cachorro, leve-o. Se você
não o quer, eu encontro outra pessoa para levá-lo, mas não faça
novamente. Os esquilos aqui são meus, mas também seus; as galinhas são
minhas, mas também suas. Quando seus filhos crescerem, eles terão a
chance de ver essas espécies no monastério. Se estivessem fora daqui,
elas desapareceriam; portanto, me ajude a protegê-las.''

Mas não eram apenas com os leigos que eles tinham problemas: o mesmo
ocorria com os demais monges da região -- algo normal, uma vez que os
pontos de vista de diferentes pessoas não são sempre iguais. Mesmo Luang
Pu Man, quando veio residir em Ubon Ratchathani, foi expulso da região
pelos monges dos monastérios de cidade, cujo modo de vida é diferente
dos monges da floresta. Os monges da cidade em geral pensam que os
monges da floresta não conhecem os verdadeiros princípios do
\emph{Buddha Sāsanā} por não terem passado pelo curso formal de estudo
das escrituras budistas. Eles afirmam que as práticas ascéticas, como
comer apenas uma refeição ao dia, são \emph{attakilamathānuyogo}, ou
seja, a prática de atormentar a si mesmo de forma inútil. Eles clamam
que os monges \emph{kammatthāna} têm visão equivocada
(\emph{micchāditthi}) e, sendo assim, como poderiam ensinar à população
de forma correta? Tudo que ensinam também deve estar errado. Quando
ensinam, eles apenas falam sem ler dos livros e escrituras, como se não
precisassem de um ponto de referência e simplesmente inventassem o que
dizer.

Na época em que Luang Pó veio morar em Wat Nong Pah Pong, era sabido de
todos que monastérios e templos budistas eram deficientes em disciplina.
Neles realizavam-se regularmente muitas atividades diretamente proibidas
pela regra monástica como, por exemplo, shows de música, bailes, jogos,
apresentações de teatro, etc. Os monges eram relapsos em moralidade e
alguns ganhavam dinheiro como médiuns, astrólogos e videntes ou
fabricando remédios milagrosos -- todas atividades expressamente
proibidas pelo Buddha à ordem monástica. Naquela época tudo isso era
considerado normal, pois as pessoas não tinham um ponto de referência
para julgar o comportamento dos monges e, por isso, ninguém reclamava.

Mas quando Luang Pó e os monges de Wat Nong Pah Pong chegaram e passaram
a servir como exemplo de monges que praticam estritamente seguindo o
Vinaya, o resultado foi que a fé dos habitantes dos vilarejos e da
cidade começou a ir em direção a eles, em detrimento dos monges da
cidade. Isso fez com que alguns monges se sentissem um tanto
incomodados. Alguns poderosos dentro dos círculos eclesiásticos olhavam
Luang Pó e Wat Nong Pah Pong com rancor, outros com incompreensão,
outros ainda com inimizade aberta. Certa vez um monge importante e
respeitado chegou a chamar Ajahn Chah de ``monge louco''. Luang Pó
apenas tolerava sem se ofender; ele jamais retaliava ou respondia aos
insultos. Ele tentava não ter atritos com nenhuma parte da sociedade
dentro ou fora do monastério, tendo fé que ``o Dhamma protege aqueles
que protegem o Dhamma.''

Todos os anos, ao começo do \emph{vassa}, ele levava seus monges para
prestar reverência aos monges da cidade mais seniores, mesmo sabendo
que, chegando lá, seria recebido com acusações e ofensas. Como sempre,
ele se comportava de forma impecável e com tamanha humildade que era
difícil acharem alguma abertura para criticá-lo. Um exemplo disso pode
ser visto neste relato de um discípulo sobre a ocasião em que foram
prestar reverência ao monge mais sênior do distrito:

``\ldots{} No começo de todo \emph{vassa}, Ajahn Chah ia prestar
reverência a todos os monges seniores da região. E este monge em
particular, todas as vezes que íamos visitá-lo, criticava Ajahn Chah.
Levávamos o monastério inteiro, dezessete monges, e prestávamos
reverência, mas é óbvio, o interesse dele era em erudição, portanto
costumava menosprezar Ajahn Chah o tempo todo. Eu me lembro de ir com
ele uma vez, e esse monge disse: `O que você está ensinando a essas
pessoas? Você está ensinando esses monges a apenas sentar de olhos
fechados, eles não conseguem ver nada, não conseguem ler nada, como vão
conhecer a verdade com os olhos fechados? Como você vai conhecer a
verdade se apenas senta com os olhos fechados e medita? Ninguém consegue
ver nada, só há escuridão.' E então disse a Ajahn Chah que ele estava
ensinando todos a serem ignorantes e tolos. Ele o criticou por uma hora
e, enquanto estávamos ali sentados, Ajahn Chah escutava as críticas.
Então, logo ao final, seu único comentário foi: `Saberemos no dia da
nossa morte quem é verdadeiro, quem está correto.' e em seguida liderou
o grupo a realizar a cerimônia de pedir perdão.\footnote{Essa cerimônia
  consiste em pedir perdão por quaisquer ofensas que possam ter cometido
  para aquela pessoa, e a mesma pessoa então responde pedindo que eles
  também o perdoem por qualquer ofensa que possa ter cometido. É
  realizada como forma de prestar reverência a monges mais velhos e
  reforçar laços de amizade dentro da sangha.}''

Mesmo quando tratado dessa maneira, sempre que havia itens que poderiam
ser úteis em um monastério da cidade e estavam sobrando em Wat Nong Pah
Pong, Luang Pó os doava a eles. Além disso, nos primeiros anos, ele
organizava aulas para que seus discípulos estudassem o currículo oficial
de Dhamma, e ele mesmo atuava como professor. Como resultado, todos os
monges de Wat Nong Pah Pong possuíam a graduação de Nak Thamm Ek. Nos
anos posteriores, apesar de o monastério não mais organizar aulas
formais, Luang Pó ainda dava permissão e suporte para que aqueles que
estivessem interessados pudessem cursar Nak Thamm.

Com o passar do tempo, alguns monges da cidade tornaram-se discípulos de
Ajahn Chah, o que fez sua reputação melhorar entre eles, até que ele se
tornou respeitado por todo o país. Pessoas de perto e de longe começaram
a fluir em direção a Wat Nong Pah Pong e todos os sentimentos de
inimizade foram abandonados. Além disso, é bem possível que toda aquela
pressão vinda de fora tenha sido útil -- além das excelentes qualidades
de Ajahn Chah -- em fomentar a harmonia que sempre foi uma das
características mais marcantes da sangha de Wat Nong Pah Pong e que a
fez prosperar por tantos anos, até os dias de hoje.
