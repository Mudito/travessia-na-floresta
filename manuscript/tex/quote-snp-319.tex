
\mbox{}\vfill

{\centering

\begin{minipage}{0.85\linewidth}%
\alegreyaSansLightFont
\fontsize{10}{14}%
\setlength{\parindent}{0pt}%
\setlength{\parskip}{5pt}%

``Como poderia alguém que caído em um rio de forte correnteza e sendo carregado
pelas águas, ajudar os demais a cruzar à outra margem?

Da mesma forma, como poderia alguém que não conhece o Dhamma, não escutou as
explicações dos Sábios, é ignorante e coberto de dúvidas, ajudar os demais a
realizarem aquele mesmo Dhamma?

Aquele que sobe num barco robusto, equipado com bom leme e bons remos, pode
ajudar muitos a cruzar à outra margem, graças à sua habilidade, consideração e
experiência.

Da mesma forma, aquele que é sábio e desenvolveu a si mesmo, que é munido de
conhecimento e mente estável, que compreende o Dhamma, pode ajudar outros a
realizá-lo se estes ouvirem com atenção.

Portanto, busque a companhia dos Nobres, daqueles que são sábios e possuidores
de conhecimento, que compreendem o significado do Dhamma. Seguindo o caminho e
realizando o Dhamma, você atingirá a felicidade.''

\bigskip

{\raggedleft
  Sutta Nipāta, 319-323
\par}
\end{minipage}

}

\vfill\mbox{}
