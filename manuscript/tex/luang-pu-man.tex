\thistitleoffsettrue
\chapter{Luang Pu Man Bhūridatto}

Ainda em Wat Khao Wong Kot, Ajahn Chah ouviu falar que Luang Pu Man
Bhūridatto -- uma pessoa que diziam ter alcançado um alto nível de
realização espiritual, perito tanto em \emph{samādhi} como em
\emph{vipassanā}, admirado e respeitado por muitos como um
\emph{arahant} -- estava passando o \emph{vassa} daquele ano em Wat Pah
Nong Pue Nanai, em Sakhon Nakhon. Um discípulo leigo de Wat Khao Wong
Kot trouxe a notícia e aconselhou Ajahn Chah a ir visitá-lo, porque ele
mesmo uma vez havia se hospedado naquele monastério praticando com Luang
Pu Man e ficado muito impressionado com seu modo de prática.

Encerrado o período do \emph{vassa}, Luang Pó refletiu e concluiu que
seu amigo, Tahn Tawan, estava mais interessado em estudar escrituras e,
portanto, seria melhor deixá-lo ir estudar em Bangkok. Desta forma os
dois amigos, que haviam viajado juntos desde Wat Ban Kó, decidiram
separar-se e seguir caminhos diferentes. Luang Pó começou sua jornada em
direção a Luang Pu Man, e em seu grupo havia um total de sete pessoas
viajando juntas: um noviço, dois leigos e quatro monges (dois da região
central da Tailândia e dois da região nordeste). Eles primeiro
caminharam de volta até Ubon Ratchatani e descansaram em Wat Ban Kó Nók
por um período curto antes de começarem a caminhada em direção a Sakhon
Nakhon.

Na décima noite chegaram a Prah Taht Panom, prestaram reverência à
famosa estupa -- que, diz-se, abriga uma relíquia de osso da região do
peito do Buddha -- descansaram uma noite e no dia seguinte continuaram
sua marcha em direção ao distrito de Naké. Lá visitaram Ajahn Son em Pu
Kó para estudar o modo de prática dele e ali permaneceram duas noites
antes de seguir adiante. Desse ponto em diante separaram-se em dois
grupos, porque Luang Pó queria visitar outros mestres e estudar seus
estilos de prática antes de conhecer Luang Pu Man, para poder comparar
os métodos de cada um.

De Pu Kó em diante eles enfrentaram muito cansaço e dificuldades. O
noviço e os dois leigos que os acompanhavam sentiram que não seriam
capazes de completar a jornada e decidiram voltar para casa. Já Ajahn
Chah e os outros dois monges continuaram em frente sem aceitar a
possibilidade de desistir. No final, o grupo chegou até o monastério de
Luang Pu Man, Wat Nong Pue Nanai, em Sakhon Nakhon.

Assim que entrou no monastério, Luang Pó ficou impressionado com o
ambiente pacífico, coberto de árvores nativas e impecavelmente limpo.
Ele também notou o comportamento e as boas maneiras dos monges de lá,
que o deixaram impressionado e mais satisfeito, mais do que em qualquer
outro monastério que tivesse visitado até então. No começo da noite,
juntou-se aos demais monges residentes do monastério para ir prestar
reverência a Luang Pu Man e ouvir seus ensinamentos. Ao encontrá-lo,
Luang Pu Man fez muitas perguntas sobre sua origem, em quais monastérios
tinha morado e praticado e quantos anos de vida monástica possuía.

Ajahn Chah lhe disse que vinha do monastério de Luang Pu Pao e lhe
mostrou a carta que o discípulo leigo de lá havia lhe dado para que
entregasse a Luang Pu Man. Ao receber a carta ele disse: ``Tahn Pao era
um dos verdadeiros monges da Tailândia.'' e então lhe falou sobre os
dois sectos de budismo tailandês -- Dhammayuta e Mahā Nikāya -- que de
fato era um assunto que vinha preocupando Ajahn Chah já fazia algum
tempo. Luang Pu Man explicou que, em termos de prática, se tomarmos o
Dhamma-Vinaya como princípio, não precisamos ter dúvidas sobre secto
algum. Portanto, não era necessário que Ajahn Chah mudasse de escola e
se juntasse ao secto Dhammayuta, como era comum a monges discípulos de
Luang Pu Man fazerem naquela época. Além disso, ele disse, o secto
Mahā Nikāya também precisava de monges \emph{kammatthāna}.

Ajahn Chah se interessava pelo estudo do Vinaya desde quando ainda
morava em monastérios de cidade. Após começar a viver como monge
\emph{kammatthāna}, frequentemente conversava e trocava conhecimentos
sobre as regras de Vinaya com os demais monges. Não foi diferente quando
se encontrou com Luang Pu Man: uma das perguntas que mais queria fazer
era justamente sobre Vinaya, e a explicação que Luang Pu Man lhe deu foi
um momento muito importante na vida de Ajahn Chah, que serviu como ponto
de referência em seu coração durante toda sua prática do Dhamma. Luang
Pó contou detalhes sobre a conversa:

``Uma vez fui prestar reverência e consultar Tahn Ajahn Man. Naquela
época eu tinha acabado de começar a praticar e tinha lido um pouco do
Pubbasikkhā e o tinha compreendido bem. Aí fui ler o Visuddhimagga, que
contém Sīlaniddesa, Samādhiniddesa, Paññāniddesa\footnote{Diferentes
  seções do livro abordando sīla, samādhi e paññā.}\ldots{} Minha cabeça
quase explodiu. Concluí que estava além da capacidade de seres humanos
praticar daquele jeito. Mas refleti um pouco mais e pensei que o Buddha
não ensinava aquilo que está além da capacidade de seres humanos, não
ensinava e não declarava porque aquilo não seria útil para ele mesmo ou
para as demais pessoas. Ele não ensinava coisas que pessoas não
conseguem fazer. O Sīlaniddesa é muito detalhado, o Samādhiniddesa
também, Paññāniddesa ainda mais -- eu sentei e pensei que não ia
aguentar, que não tinha como seguir em frente. Era como se não houvesse
mais caminho.

Naquela época eu estava batalhando em minha prática, estava empacado.
Por sorte tive a oportunidade de visitar Tahn Ajahn Man e lhe perguntei:

`O que devo fazer? Comecei a praticar recentemente, mas não sei como
devo prosseguir. Tenho muitas dúvidas, ainda não consegui encontrar um
ponto de apoio para tocar minha prática.'

Ele disse: `O que está acontecendo?'

`Estava procurando um caminho e então peguei o livro Visuddhimagga para
ler. Tenho a sensação de que não vou conseguir, porque o que está
escrito no Sīlaniddesa, Samādhiniddesa, Paññāniddesa me parece já não
estar mais dentro da capacidade dos seres humanos. Eu tenho a opinião de
que nenhuma pessoa neste mundo é capaz de fazer aquilo, é duro, é
difícil estar sempre atento a todas as regras, sem exceções. É
impossível, está além da capacidade de qualquer um.'

Então ele disse: `Tahn\ldots{} É verdade que é muita coisa, mas, na
verdade, é muito pouco. Se fomos lembrar de todas os itens do
Sīlaniddesa, é verdade que será muito difícil, muito duro\ldots{} Mas,
na realidade, o que se chama Sīlaniddesa é uma explicação que se
originou na mente humana. Se treinarmos nossa mente a ter vergonha, ter
medo de fazer tudo aquilo que é errado, seremos uma pessoa atenta e
cuidadosa graças àquele medo.

Quando for assim, isso será causa para sermos pessoas frugais. Não
seremos pessoas sem moderação, porque assim não conseguiríamos respeitar
nossos preceitos. Quando formos assim, \emph{sati} ficará mais forte,
teremos \emph{sati} ao estar em pé, andando, sentado ou deitado, teremos
que estar sempre resolutos em suster \emph{sati} o tempo todo,
plenamente. Você passará a ser cuidadoso.

Quando tiver dúvida se algo é correto ou errado, não o diga, não o faça!
Se não sabemos algo, primeiro perguntamos ao professor; tendo perguntado
e ouvido a explicação, apenas ouvimos e guardamos aquela informação;
ainda não temos certeza, pois ainda não é algo que nasceu de dentro de
nós mesmos.

Se formos lembrar de todos os detalhes, vai ser difícil. Mas será que já
conseguimos aceitar que fazer o errado é incorreto e fazer o certo é
correto? Já conseguimos aceitar isso ou não?' -- esse ensinamento dele é
algo importante; não precisamos cuidar de toda e cada regra, cuidamos só
da mente e já é suficiente.

`Tudo aquilo sobre o que você foi ler nasce da mente. Se você não
treinar sua mente para que tenha sabedoria e pureza, vai continuar
duvidando infinitamente, vai ser sempre vítima de
\emph{vicikicchā}.\footnote{Dúvidas e incerteza sobre o caminho para
  iluminação. Um dos cinco obstáculos à concentração mental (nīvaranas).}
Portanto, traga todo o ensinamento do Buddha de volta à sua
mente,\footnote{Tradução alternativa: `unifique o ensinamento do Buddha
  em sua mente'} esteja atento à mente. Quando algo surgir, se ficar em
dúvida, apenas renuncie àquilo -- se ainda não tiver certeza, não faça,
não fale. Por exemplo: `isso está certo ou errado?' -- se estivermos
pensando assim, significa que ainda não sabemos de acordo com a verdade
e, portanto, não devemos fazer aquilo, não devemos dizer aquilo, não
devemos cruzar aquela linha.'

Eu estava sentado ouvindo isso e vi que era um Dhamma que estava de
acordo com o ensinamento do Buddha que diz: `Qualquer ensinamento que
leve a acumular \emph{kilesas}, qualquer ensinamento que leve a gerar
sofrimento, qualquer ensinamento que leve a poluir a mente com desejos
sensuais, qualquer ensinamento que encoraje a falta de frugalidade,
qualquer ensinamento que encoraje a ambição, qualquer ensinamento que
encoraje juntar-se em grupos,\footnote{Em oposição a viver em reclusão.}
qualquer ensinamento que encoraje a preguiça, qualquer ensinamento que
leve alguém a ser uma pessoa difícil de satisfazer -- um ensinamento que
estiver de acordo com essas oito características, não estará de acordo
com o ensinamento do Buddha. Caso contrário, estará correto.'

Se realmente estivermos interessados, nossa mente terá que ser a mente
de uma pessoa que tem vergonha de fazer o mal, que tem medo do que é
errado. Saiba na sua mente e, caso esteja em dúvida, não faça aquilo,
não fale aquilo. Samādhiniddesa é a mesma coisa, é apenas um texto. Por
exemplo, quando escrito em um livro, \emph{hiri-ottappa}\footnote{Hiri:
  senso de pudor, vergonha. Ottappa: medo, receio. (pāli)} é de um
jeito, mas quando passa a existir em nosso coração, é de outro
jeito\ldots{}''

Após explicar isso, Luang Pu Man falou sobre \emph{sīla, samādhi} e
\emph{paññā} até que todos estivessem satisfeitos e não tivessem mais
dúvidas. Explicou também sobre os cinco poderes de um praticante
(\emph{pañca bālā}) e as quatro bases para o poder mental (\emph{catu
iddhipāda}). Todos os discípulos escutavam com atenção, interesse e com
uma atitude pacífica. Ajahn Chah disse que, apesar de ter vindo
caminhando de muito longe e estar se sentindo cansado, assim que ouviu o
ensinamento de Luang Pu Man sentiu todo cansaço desaparecer e sua mente
entrar em \emph{samādhi}. Sentiu como se estivesse flutuando no ar. Eles
ouviram Luang Pu ensinar até a meia-noite e então encerraram a reunião.

Na segunda noite, Luang Pu Man falou sobre diferentes aspectos do Dhamma
e ao final do ensinamento Ajahn Chah sentiu que já não tinha mais
dúvidas sobre como praticar. Ele sentiu uma quantidade de êxtase e
deleite que jamais havia experimentado e sentiu inspiração e coragem
para continuar trabalhando sem desistir até alcançar \emph{nibbāna}. O
ensinamento que Luang Pu Man enfatizou naquela ocasião foi
``\emph{sakkhiputtha}'', ser sua própria testemunha, ou seja, enxergar e
saber por si mesmo. Outro ensinamento que impressionou Ajahn Chah foi
sobre a diferença entre a mente e as manifestações da mente:

``Com relação a essas manifestações, Tahn Ajahn Man disse que são apenas
manifestações. Nós não entendemos isso e então achamos que são todas
verdadeiras, pensamos que são nossa mente, mas na verdade são apenas
manifestações. Assim que ele disse que são apenas manifestações, tudo
ficou claro. Por exemplo, felicidade existe na mente, mas é apenas uma
manifestação, é algo distinto da mente propriamente dita, existe num
nível diferente dela. Se conhecermos essa verdade, abandonamos, largamos
(o apego à felicidade). Essa é uma realidade convencional, mas ao mesmo
tempo transcendental. É desse jeito. Algumas pessoas juntam tudo e dizem
que aquilo é a mente, mas na verdade são apenas manifestações mentais e
ciência fazendo contato dentro da mente. Se sabemos disso, vemos que não
tem nada de mais.''

No terceiro dia, Ajahn Chah se despediu de Luang Pu Man e começou a
caminhar em direção a Na Ké, em Nakhon Panom. Ele uma vez revelou a seus
discípulos sobre a importância daquele encontro com Luang Pu Man:

``Eu ter adquiro sabedoria e inteligência a ponto de hoje poder
partilhá-las com todos os senhores foi possível por ter prestado
reverência a Luang Pu Man, por tê-lo encontrado e visto o monastério
dele. Mesmo não sendo bonito, era um lugar muito limpo. Havia 50 ou 60
monges, mas o silêncio era absoluto! A ponto de que, quando queriam
lascar madeira para ferver e fazer tintura para os mantos, levavam a
madeira lá longe, porque tinham medo de perturbar a paz dos demais.
Tendo realizado suas tarefas, cada um voltava para sua cabana para
praticar meditação andando e não se ouvia som algum, a não ser o som de
passos. Por volta das sete da noite íamos prestar reverência a Luang Pu
Man e ouvir Dhamma. Terminado isso, por volta das dez ou onze horas,
voltávamos às nossas cabanas levando o Dhamma que havíamos ouvido para
contemplar. Quando ouvíamos o ensinamento dele, nos sentíamos
satisfeitos; quando praticávamos meditação sentados ou andando não
sentíamos cansaço, tínhamos muita energia. Quando se encerrava a
reunião, era um silêncio absoluto. Às vezes, dois monges moravam
próximos e um notava que o outro praticava meditação andando o dia e a
noite inteira, a ponto de querer ir olhar quem era aquela pessoa. Por
que pratica meditação andando sem parar para descansar? É porque a mente
dele está energizada\ldots{}''

Muitos perguntam por que Ajahn Chah permaneceu com Luang Pu Man apenas
três dias; se estava em busca de um professor, por que não permaneceu
com Luang Pu Man? Luang Pó respondia a essa pergunta com um símile:
``Uma pessoa com bons olhos, quando encontra uma lâmpada, vê a luz. Já
uma pessoa cega, mesmo que se sente bem em frente à lâmpada, não vê
nada.'' Após esse encontro com Luang Pu Man, a fé de Ajahn Chah no
ensinamento do Buddha ficou ainda mais forte, e ele agora estava
disposto a arriscar sua vida no esforço para alcançar \emph{nibbāna},
porque o caminho já estava claro para ele. Durante todo aquele período,
quer Luang Pó estivesse andando ou sentado em meditação, ele tinha a
sensação de que Luang Pu Man o estava acompanhando, dando-lhe conselhos,
alertando-o o tempo todo.

Após deixar Luang Pu Man, Ajahn Chah e seus companheiros continuaram
viajando em peregrinação, eventualmente parando para praticar em
montanhas e florestas que encontravam no caminho. Quando alcançaram Na
Ké, Tahn Bun Mi separou-se do grupo, que agora era composto apenas de
Ajahn Chah, Tahn Luem e Anagārika Kéu. Um dia, Luang Pó e seus
companheiros chegaram a uma floresta ao pé de uma montanha e, como já
estava escurecendo, decidiram passar a noite ali. Por volta das nove da
noite uma matilha de cães selvagens veio correndo de dentro da floresta.
Quando viram Ajahn Chah, se aproximaram rosnando, demonstrando intenção
de atacá-lo. Luang Pó foi pego de surpresa e não sabia o que fazer,
então apenas permaneceu sentado dentro de seu \emph{glot},\footnote{Uma
  espécie de guarda-chuva grande e robusto que os monges utilizam quando
  acampam na floresta. Por cima dele é posta uma tela contra mosquitos
  para proteger o monge durante a prática de meditação.} estabeleceu
\emph{sati} em sua mente e fez uma determinação: ``Eu não vim aqui para
agredir ninguém, vim por desejar praticar o bem; meu propósito é
transcender \emph{dukkha}, se no passado eu alguma vez os agredi, que
vocês agora me mordam até a morte para que assim se encerre esse
\emph{kamma} antigo. Mas se nunca houve inimizade entre nós, que vocês
me deixem em paz.''

Então sentou em meditação com os olhos fechados, disposto a renunciar à
sua vida naquele instante. Os cães corriam ao redor dele latindo e
rosnando, às vezes ameaçando invadir o \emph{glot}. Ajahn Chah começou a
sentir muito medo, mas após um instante meditando viu uma imagem de
Luang Pu Man munido de uma lanterna andando em sua direção. Quando
chegou ao local em que estava Ajahn Chah, gritou: ``Fora! O que vocês
querem com ele?'', e ergueu uma vara ameaçando bater nos cães. A matilha
dissipou-se com medo e fugiu. Ajahn Chah pensou que Luang Pu Man
realmente tivesse vindo salvá-lo e abriu os olhos esperando vê-lo à sua
frente, mas não havia ninguém. Os cães também haviam desaparecido, sem
sobrar nenhum.

