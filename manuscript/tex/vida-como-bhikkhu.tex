\chapter{Vida como bhikkhu}

Após completar 21 anos de idade e ser dispensado do serviço militar,
Luang Pó decidiu ordenar-se. Com a aprovação e deleite de seus pais, a
data da cerimônia foi marcada para o dia 26 de abril de 1940, em Wat Ban
Kó Nai. Prah Kru Intara Sankhun atuou como \emph{upajjhāya}. Luang Pó
recebeu o nome monástico ``Subhaddo'', que significa ``aquele
desenvolvido de forma correta''.

``Quando me ordenei, não pensava que viveria desse jeito, não pensava em
ser o `mestre' de ninguém, não pensava que escaparia deste
\emph{samsāra}. Ordenei-me como os demais, mas, uma vez ordenado,
estudei, pratiquei, e fé nasceu em mim. A primeira vez que senti isso
foi quando estava pensando sobre a vida de todos os seres neste
mundo\ldots{} É tão triste! Nascemos e num instante já estamos mortos,
deixando nossa casa e todas nossas posses para os filhos, só para que
eles então comecem a brigar pelos bens. Quando vi isso, senti desencanto
pela vida do rico e do pobre, do tolo e do esperto -- todos eles vivem
neste mundo e estão todos sujeitos a essa mesma realidade.

Meu coração estava sempre repleto desse sentimento e eu não sabia com
quem conversar a respeito. Quando uma mente desperta, muitas coisas
despertam em seguida e passamos a ver muitas coisas que nos fazem
despertar ainda mais. Frequentemente eu pensava: `Tive a oportunidade de
conhecer o Dhamma, ou seja, ganhei uma luz que me permite ver muitas
coisas. Agora tenho o Dhamma.' e sentia meu coração encher-se de
felicidade. Quanto mais praticava e estudava, mais satisfação encontrava
e mais firme ficava minha fé; ganhava cada vez mais paz e sabedoria.''

Após ordenado, Luang Pó passou dois anos em Wat Ban Kó Nók. Durante esse
tempo ele prestou o exame e atingiu a graduação `Nak Thamm Tri' do
currículo oficial de estudo do Dhamma. Luang Pó falou a seus discípulos
sobre como foram seus primeiros anos de monasticismo:

``No começo, quando me ordenei, não praticava nada, mas tinha fé --
talvez fosse parte da minha natureza, não sei dizer. Ao fim do
\emph{vassa}\footnote{Uma vez por ano, durante as monções, todos os
  monges devem permanecer numa única residência por três meses. Em pāli
  esse período se chama `vassa', mas no ocidente criou-se o hábito de
  chamá-lo de `retiro das chuvas' ou `retiro das monções' (pāli).}\emph{,}
todos os demais monges e noviços que se ordenaram comigo voltaram à vida
laica. Eu olhava e pensava: `Eh\ldots{} Qual o problema desses bobos?'
Mas não tinha coragem de falar para eles, porque ainda não confiava nos
meus próprios sentimentos. Eles estavam alegres, mas eu pensava: `É
muita burrice. É difícil se ordenar, mas é fácil abandonar a vida
monástica. Esses têm pouco mérito, não têm muito mérito. Eles acham que
o caminho mundano é mais útil do que o caminho do Dhamma.' Eu pensava
desse jeito mas não dizia nada, só olhava para minha própria mente.

Vi meus amigos que se ordenaram junto comigo indo embora um por um. Às
vezes eles vestiam uma roupa nova e vinham se exibir no monastério. Eu
achava que eles eram doidos da cabeça aos pés, mas eles achavam que
aquilo era bom, bonito. Eles haviam voltado à vida laica e tinham que
fazer isso e aquilo -- eu só olhava para minha mente, não tinha coragem
de dizer a eles que pensar daquela maneira estava errado. Eu não tinha
coragem de falar porque ainda não tinha como ter certeza por quanto
tempo minha fé ia durar. Não tinha coragem de dizer nada a ninguém,
apenas pensava sobre isso na minha mente.

Após todos irem embora, fiquei desamparado. Já que não tinha ninguém por
perto, peguei a regra monástica e comecei a ler, e então memorizei a
regra monástica sem muitas dificuldades -- não havia quem viesse me
distrair com nada. Eu estava resoluto, mas não dizia nada porque
praticar até o fim dos dias, até os setenta, oitenta anos de idade, com
inteligência, sem relaxar o esforço, sem perder a fé, com continuidade
-- é muito difícil, portanto não me atrevia a dizer nada. Quem queria
virar monge virava, quem queria largar o manto largava -- eu apenas
observava, não dizia nada para quem ficasse ou fosse embora. Eu via os
amigos indo embora, mas no meu coração a sensação que eu tinha era de
que aquele pessoal não enxergava muito bem.''

Na tradição Theravada os monges ainda seguem o exemplo do Buddha e só
fazem uma ou duas refeições ao dia, entre o período do nascer do sol ao
meio-dia, e durante a tarde e a noite eles permanecem em jejum. Por isso
é normal monges recém-ordenados sofrerem com a fome, e Luang Pó não era
exceção. Ele contou sobre sua experiência com esse assunto:

``Não pense que praticar não é sofrimento, tem que sofrer. Nos primeiros
um ou dois anos de vida monástica é ainda mais sofrido. Monges jovens e
noviços pequenos\footnote{Para receber ordenação como monge (bhikkhu) é
  necessário ter pelo menos 20 anos de idade. Aqueles de menor idade
  podem se ordenar noviços (sāmanera).} sofrem ainda mais. Eu mesmo
sofri muito, especialmente com comida. Eu tinha vinte anos quando me
ordenei, estava acostumado a comer e dormir à vontade. Que dizer? Às
vezes sentava sozinho pensando em comida e coisas que queria comer --
\emph{tam gluei tani, som tam malakó}\footnote{Nome de pratos típicos
  tailandeses.}\emph{,} todo tipo de coisa. A saliva fluía e pingava da
minha boca. Era uma tortura, nada era fácil\ldots{}

Mas, se olhar de novo, prática é isso mesmo -- quanto mais difícil,
melhor. Para que fazer o que é fácil? Qualquer um consegue fazê-lo.
Aquilo que é difícil fazer é onde está nossa prática e continuamos até
vencermos. Algumas pessoas me dizem: `Se me ordenasse desde jovem, como
você fez, e não tivesse família, ia ser fácil praticar. Eu não teria
muito com que me preocupar.' Isso é o que dizem, mas é melhor não falar
isso perto de mim ou vão levar uma bengalada. Ora, eles falam como se eu
não tivesse um coração! A prática do Dhamma vai contra nossa própria
mente, não é uma tarefa pequena. Envolve nossa vida inteira.''
