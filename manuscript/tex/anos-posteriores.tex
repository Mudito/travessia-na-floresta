\chapter{Anos posteriores}

Durante os primeiros dez anos, Wat Nong Pah Pong era apenas uma
residência monástica na floresta: havia poucas edificações permanentes e
os monges viviam de maneira simples em pequenas cabanas com telhado de
palha. Luang Pó liderava seus discípulos a praticar de forma rigorosa e
não se preocupava com o desenvolvimento material do monastério; além do
mais, naqueles dias não havia fundos suficientes para tocar projetos de
construção. Com o tempo, a reputação de Luang Pó começou a se espalhar,
tanto em Ubon como em outras províncias, e o número de monges e monjas
cresceu, assim como a quantidade de leigos que frequentavam o
monastério. Em apenas alguns anos Wat Nong Pah Pong tornou-se um grande
monastério e matriz de um grande número de monastérios
filiados,\footnote{Na Tailândia, é comum as pessoas oferecem terra a um
  mestre para que ele estabeleça um monastério. Ele então envia um de
  seus discípulos para morar lá e construir o local, e o monastério é
  entendido como uma filial do monastério original.} o que criou a
necessidade de adaptar o local à sua nova situação.

Por essa razão, Wat Nong Pah Pong teve que desenvolver a estrutura
física necessária para poder servir de centro para as centenas de monges
que compunham sua própria sangha, assim como para os membros do
crescente número de monastérios afiliados. Atualmente, existem cerca de
70 cabanas disponíveis para monges e 60 para monjas, há um templo
principal, um salão para usos diversos, um refeitório, um salão de
\emph{uposatha}, uma estupa (onde estão guardados os restos mortais de
Ajahn Chah), um museu e muitas outras edificações de suporte como
banheiros e lavatórios, reservatório d'água, cozinha, dormitórios para
visitantes leigos e monásticos, escritórios, etc. A atual área do
monastério é de 30 hectares.

Luang Pó não gostava de construções luxuosas, extravagantes ou
supérfluas. Se uma edificação era necessária ao monastério, ela deveria
ser feita de forma adequada e apenas o que era realmente útil deveria
ser construído. A edificação deveria se integrar com a natureza do local
tanto quanto possível, de forma a não destruir a atmosfera da floresta.
Deveriam ser simples e baratas, como é apropriado a um monastério de
floresta. Um princípio importante do qual Luang Pó nunca abria mão era o
de nunca financiar nada fazendo pedidos à comunidade leiga, nem mesmo de
maneira indireta. Por causa disto, o desenvolvimento material de Wat
Nong Pah Pong ocorreu de forma espontânea, sem que nunca se organizassem
campanhas ou eventos para angariar recursos. Luang Pó sempre dizia que
era algo deplorável ver monges correndo atrás de dinheiro para construir
monastérios: monges deveriam construir pessoas, e essas pessoas então se
encarregariam de construir o monastério.\footnote{Significa que os monges
  deveriam ensinar o Dhamma à população, e aqueles que entendessem o
  valor desse ensinamento voluntariamente se encarregariam de criar as
  condições físicas para que ele prosperasse.}

Uma monja falou sobre o processo de construção em Wat Nong Pah Pong:
``Luang Pó proibia a organização de campanhas para angariar fundos, ele
só construía o que fosse possível. Quando o dinheiro acabava,
simplesmente interrompíamos a construção temporariamente. Para
economizar dinheiro e ajudar a desenvolver resiliência, todos os
projetos de construção utilizavam, em grande parte, os monges e noviços
como operários. Era uma forma de desenvolver \emph{sati} em todas as
posturas e fomentar a harmonia dentro do grupo. Além disso, uma vez que
os recursos para financiar a construção eram limitados, tínhamos que
economizar. O que podíamos fazer para economizar dinheiro, fazíamos. Às
vezes os aldeões das cercanias também sacrificavam um pouco do tempo
deles para vir ajudar.''

Mas não só as características físicas do local se transformaram com o
passar do tempo: o número de leigos e monásticos que vinham a Wat Nong
Pah Pong em busca de treinamento, o tipo de pessoas e até mesmo as
nacionalidades que compunham esses grupos foi se alterando e, com tudo
isso, um fenômeno talvez inesperado para muitos ocorreu: a personalidade
de Luang Pó Chah também mudou para se ajustar à nova situação.

Luang Pó viveu em Wat Nong Pah Pong por quase 40 anos. Quando
estabeleceu o monastério, tinha 36 anos de idade e era monge havia 17
anos -- ainda era jovem, estava plenamente concentrado em desenvolver
sua prática do Dhamma e almejava alcançar a realização mais elevada.
Aqueles que o conheceram naquela época tinham a memória de um monge
jovem e de poucas palavras, severo e resoluto, um trabalhador vigoroso e
ágil, magro e de pele castigada pelo sol. Ele enfrentou todos os
obstáculos daquela época com grande paciência e perseverança.

Nos anos posteriores, tendo Luang Pó Chah alcançado um nível que julgava
satisfatório em sua prática e ganhado mais experiência em administrar e
ensinar uma comunidade de discípulos, pessoas de todas as direções
começaram a vir para Wat Nong Pah Pong e tornaram-se discípulos. A
imagem que essa geração de pessoas tem de Luang Pó já é completamente
diferente da geração anterior: eles o viam como uma pessoa calorosa e
informal, cheia de bondade e bom humor. Essa era indubitavelmente a
característica pessoal que chamava a atenção dos presentes quando ele
interagia com leigos e monges. A única diferença era que Luang Pó tendia
a ser mais rigoroso com os monges, uma vez que estes eram discípulos
mais próximos, como filhos, o que o fazia se sentir mais à vontade em
exigir mais deles. Além disso, por levar um estilo de vida monástico,
eles estavam muito mais capacitados a colocar os ensinamentos dele em
prática e colher os resultados esperados. Seus discípulos monásticos
estavam sempre alertas e cientes de que, por trás daquele bom humor e
aparente inofensividade, estava escondido um tigre feroz, pronto a
atacar assim que um discípulo agisse de forma desatenta ou inapropriada
à linhagem de monges \emph{kammatthāna}.

Um monge contou como era trabalhar como atendente dele durante aqueles
anos: ``Luang Pó ia dormir por volta das três da manhã todo dia. Ele
descansava nesse horário e às cinco da manhã acordava. Quando acordava,
eu oferecia água morna e água fria (para ele lavar o rosto) e oferecia a
escova de dentes. Inicialmente eu oferecia água para lavar o rosto e
escovar os dentes; nos anos posteriores, passei a pedir licença para
lavar a dentadura dele. Lavava o manto dele, ajudava ele a se vestir,
preparava a cama dele, entre outras coisas. Se ele visse que a pessoa
fazia um bom trabalho, ele deixava fazer; se fizesse malfeito, ele se
irritava -- às vezes tocava a pessoa para fora da cabana. Normalmente
ele não dava ordens para que todas essas coisas fossem feitas, porque
depende de cada um. Ele não dizia: `Você tem que lavar minha dentadura,
tem que esvaziar meu penico, tem que lavar meus pés, lavar meu manto,
etc.' Ele nunca falava isso, mas se o discípulo tivesse fé e desejasse
servir ao mestre de acordo com o que está especificado no Vinaya, e se
aquela pessoa fosse suficientemente capaz, ele dava permissão para que o
fizesse.''

Muitos discípulos notam que, de 1967 em diante, os ensinamentos de Luang
Pó começaram a se suavizar e provavelmente havia várias razões para
isso, a primeira seria o fato de que ele estava envelhecendo e a força
de seu corpo, diminuindo. A segunda razão talvez fosse uma mudança de
público: pessoas de classe média, educadas em universidades, começaram a
vir tanto de Bangkok e outras áreas urbanas da Tailândia como do
ocidente, interessadas em receber seus ensinamentos -- especialmente
após uma biografia ter sido preparada e publicada em 1968 -- e é
possível que essa tenha sido uma razão pela qual Luang Pó mudou sua
maneira de falar para melhor comunicar a esse novo grupo de pessoas. Um
discípulo conta sobre essa mudança:

``Desse ponto em diante, os ensinamentos ficaram mais suaves.
Transmitiam um deleite no Dhamma, e quem os ouvia se sentia bem. Quem
tinha sabedoria se beneficiava muito, porque os ensinamentos eram leves
e agradáveis. Anteriormente, se a pessoa não gostasse de ouvir (os
ensinamentos duros que Luang Pó dava), não conseguia ficar mais que três
noites. Simplesmente não conseguia ficar. Mandavam vir ouvir o
ensinamento três noites; se não fosse compatível com o temperamento
deles, não conseguiam ficar. Doía e machucava, se ofendiam logo na
primeira noite. Isso era uma técnica que Luang Pó utilizava.

Mas quando ele ficou mais velho, com idade mais avançada, já não
conseguia manter aquele nível de intensidade, e portanto mudou (de
estilo). O conteúdo dos ensinamentos continuou o mesmo, mas o sabor
melhorou, ficou mais divertido, mais agradável de ouvir. Porém, acho que
as pessoas tinham mais dificuldades em entender, porque se entretinham
ouvindo. Antes, quem ouvia logo entendia, porque era algo que doía e
ardia, machucava e ofendia. Quem não gostava não conseguia ficar, mas
quem conseguia ficar três noites permanecia dali em diante.''

Nos primeiros anos, Luang Pó costumava ensinar usando a língua do
Laos,\footnote{O laociano é uma língua próxima ao tailandês, assim como o
  português é próximo ao espanhol.} pois a região nordeste da Tailândia
faz fronteira com o país e ambos partilham muitos aspectos culturais.
Como resultado, pessoas das demais regiões da Tailândia e os ocidentais
que vinham a Wat Nong Pah Pong tinham a impressão de que Luang Pó estava
com raiva, pois falava rápido, e numa época em que ainda não havia
alto-falantes, ele se expressava com voz alta e enfática que podia ser
ouvida de muito longe. Mesmo se você saísse do salão onde o ensinamento
estava sendo dado e andasse uma longa distância, ainda conseguia ouvir o
som dele discursando. Sua voz tinha força, ela empolgava quem ouvia,
fazia-os sentir como se uma grande onda d'água os tivesse lavado e
carregado consigo; era como um general comandando suas tropas na guerra
contra as trevas do \emph{samsāra}.

Já quando alcançou idade avançada, Luang Pó passou a em geral utilizar a
língua tailandesa e falar com voz suave e calorosa, como o som de um pai
ensinando a seus filhos. Sua voz ainda possuía a mesma força, mas agora
o fluxo de Dhamma que emergia de seu coração estava repleto de humor e
tolerância. Mas, mesmo nessa época, ele às vezes ainda falava em
laociano, especialmente nas noites em que o fluxo de Dhamma jorrava com
força de seu coração e ele, sem perceber, acabava voltando a falar na
língua do Laos.

Outro discípulo dessa mesma época falou sobre essa atmosfera que
começava a mudar e tornar-se mais relaxada, mas ainda poderosa: ``Luang
Pó ensinava várias formas de prática; ele enfatizava a rotina monástica,
o modo de prática, as atividades da sangha e assim por diante. Se você
ouvisse o som do sino, que era um sinal de aviso de que havia chegado a
hora de uma atividade coletiva, você tinha que se apressar. Mesmo que
estivesse costurando ou tingindo um manto, fabricando escovas de dentes
ou realizando qualquer outra atividade pessoal, você tinha que deixar
tudo de lado e dar prioridade à atividade da sangha, porque era mais
importante. Isso também demonstrava que ter harmonia dentro do grupo era
uma fundação importante.''

Algo que Ajahn Chah também enfatizava era a necessidade de humildade,
especialmente para os monges recém-ordenados. É comum encontrar monges
que ouvem apenas algumas palavras do ensinamento do mestre, leem um
pouco as escrituras, ainda não fizeram grande progresso com a prática de
meditação, mas logo estão ansiosos para andar por aí ensinando aos
outros e se \mbox{vangloriando} de qualquer experiência insignificante que
tenham tido quando sentados em meditação. Querem contar suas
experiências para todos e até mesmo durante \emph{pindapāta} não
conseguem parar de falar. Para exemplificar, Luang Pó contava a história
do genro:

``O genro, recém-casado, foi morar com os pais de sua esposa, mas,
sempre que fazia algo, costumava contar vantagem. Ele queria que o sogro
pensasse que ele era uma pessoa capaz, diligente e forte, sem preguiça,
hábil em ganhar a vida. Ele mudou-se para a casa do sogro que ficava
perto de um riacho. No final da tarde, o sogro levou o genro à beira do
riacho para pegar peixes usando uma armadilha de bambu. Ele pegou um
cupinzeiro, com cupins ainda dentro, e o colocou dentro da armadilha
para atrair os peixes, colocou uma pedra pesada sobre a armadilha para
que ela não fosse levada pela correnteza durante a noite e disse para o
genro vir verificar na manhã seguinte.

Na manhã seguinte o genro viu a armadilha cheia até a borda de peixe
baiacu\footnote{No original tailandês o tipo de peixe é outro, mas
  também com um nome engraçado.} -- os peixes pulavam e se contorciam
dentro da cesta. Ele ficou muito feliz porque havia apenas baiacu; não
havia nenhum outro tipo de peixe, apenas baiacu. Naquela época o
desenvolvimento ainda não havia chegado ao campo e era difícil obter
tecido, por isso as pessoas normalmente tiravam as calças e as deixavam
à beirada quando tinham que entrar n'água. Ele então colocou a armadilha
nos ombros com alegria e se esqueceu de vestir as calças. Caminhou de
volta para casa cantando `Olha o baiacu! Aqui só tem baiacu!' e subiu as
escadas para mostrar à sua esposa. A esposa estava agachada ao chão
preparando arroz para cozinhar quando ouviu o barulho alto: `Olha o
baiacu! Chegou o baiacu!', ela levantou o rosto e ouviu o grito `Só tem
baiacu! Nenhuma tilápia, dá só uma olhada!' A esposa ao mesmo tempo
falou e apontou para o corpo nu do marido, que finalmente se deu conta
da situação, olhou para si mesmo e saiu correndo para buscar as calças
que havia esquecido à beira do riacho.

Alguns dos nossos monges recém-ordenados são assim. Eles se ordenam e
assim que experienciam ou veem qualquer coisinha já querem contar
vantagem. Eles se deixam levar, querem se exibir para os outros, mas
para sua própria feiura eles não olham, não enxergam. Igual ao genro que
queria contar vantagem para os sogros, e por causa disso não conseguia
ver onde ele mesmo estava incompleto. Alguns aqui não conseguem sequer
acompanhar a rotina diária do monastério e já querem contar vantagem
para os amigos ouvirem -- aonde quer que vão eles dão sermão; aonde quer
que vão, só sabem falar.

\ldots{} Ao praticar, não o faça com o nariz empinado, não se apegue a
ter alcançado isso ou aquilo; pratique para pacificar, cessar. Não é
necessário ficar agitado querendo atingir alguma realização espiritual.
Algumas pessoas, assim que começam a praticar e veem qualquer coisa, já
pensam que aquela é a `coisa de verdade', que já conseguiram, já
alcançaram algo. Isso não é correto.

Uma vez ocorreu no monastério de Luang Pó Pao de uma monja vir
procurá-lo e dizer: `Luang Pó, já alcancei \emph{sotāpanna}!' Ele ouviu
a monja dizer aquilo e então respondeu: `Eh\ldots{} Um pouco melhor que
ser uma cadela, não?' Foi só dizer isso e aquela `\emph{sotāpanna'}
perdeu a calma e foi embora. É assim mesmo, está completamente errado.

Quando estiver praticando, não deixe o nariz ficar empinado. O que quer
que você pense ser, deixe estar. Se for \emph{sotāpanna}, deixe estar;
se for \emph{arahant}, deixe estar. Seja uma pessoa simples, construa
bondade sem parar. Onde quer que esteja, seja uma pessoa normal; não
precisa se exibir, pensando `eu alcancei, eu sou isso ou aquilo'. Mas
hoje em dia, se alguém é \emph{arahant}, não vive feliz. Pensa `eu sou
um \emph{arahant}' e fica o tempo todo anunciando para que os outros
saibam. Não consegue viver em lugar algum. Na época do Buddha, quem era
\emph{arahant} vivia com simplicidade -- não como o que vemos nos dias
de hoje.''

Um monge contou um pouco mais: ``Respeito também era muito importante.
Por exemplo, lavar os pés dos monges mais seniores\footnote{Pindapāta é
  feita com pés descalços; por isso, quando os monges mais seniores
  voltam ao monastério, alguns monges júniores recebem a tigela deles
  enquanto outros os ajudam a lavar e secar seus pés.} e coisas do tipo
-- ele dizia para dar atenção especial a isso. É algo muito benéfico,
pois ajuda a diminuir a arrogância e vaidade, algo difícil de fazer. Era
o que ele dizia. Não é fácil encontrar quem consiga fazer essas coisas,
ajudar a lavar os mantos e realizar outras tarefas. Mesmo na hora de
lavar os pés, ele dizia para antes fazer \emph{añjali} e pedir licença.
Tínhamos que ter humildade, modéstia e agir de forma respeitosa.''

Viver com simplicidade e frugalidade era outro traço característico de
Luang Pó Chah. O interior de sua habitação era quase que completamente
vazio, porque ele não guardava itens supérfluos. Ele dormia num quarto
pequeno, onde havia apenas uma cama e alguns utensílios. Luang Pó não
tinha conta bancária: todos os bens e recursos financeiros que lhe
ofereciam eram repassados para monastérios filiados ou partilhados com a
comunidade em Wat Nong Pah Pong.

Certa vez um monge que tinha um gosto por arte estava procurando uma
oportunidade para olhar dentro da cabana de Luang Pó, porque notava que
frequentemente leigos traziam estátuas do Buddha e outros objetos de
arte religiosa para oferecer a ele. O monge imaginava que Luang Pó os
mantinha guardados em sua cabana, mas quando enfim conseguiu uma
desculpa para entrar e olhar, ficou surpreso em encontrar o local vazio
-- Luang Pó não guardava nada, não importa quão valioso fosse o objeto
oferecido; tudo que recebia era doado para outras pessoas.

Muitas vezes os discípulos leigos reclamavam que já haviam oferecido
\emph{pavāranā}\footnote{Um convite aberto para que o monge peça algo em
  caso de necessidade. Por exemplo, um leigo pode colocar um valor à
  parte e avisar ao monge que peça caso um dia esteja precisando de algo
  dentro daquele valor. O leigo então se encarrega de providenciar o
  artigo especificado (pāli).} várias vezes, mas ele nunca fazia uso
delas; pelo contrário, ele frequentemente dizia a seus monges: ``Quanto
mais eles oferecem \emph{pavāranā}, mais eu os evito.'' Certa vez, uma
leiga lhe ofereceu \emph{pavāranā} de 90.000 bahts (que na época
representava um valor muito maior do que hoje), mas o proibiu de
utilizá-los para qualquer outro propósito a não ser para benefício
próprio. Ele aceitou e deu a seguinte ordem a seus discípulos que
estavam perto: ``Se um dia alguma necessidade relacionada a mim surgir,
usem todo esse dinheiro, não deixem sobrar nada.'' Como resultado, só
após sua morte o dinheiro pôde ser utilizado, porque ele jamais o fez em
vida. O fundo foi usado para custear a impressão das milhares de cópias
da biografia de Luang Pó Chah que foram distribuídas gratuitamente
durante seu funeral.

Um discípulo conta esta história: ``Uma vez um leigo de Ubon comprou um
carro para lhe oferecer e disse que não aceitaria um `não' como
resposta. Ele estacionou o carro bem ao lado da cabana de Luang Pó,
colocou as chaves dentro da sacola dele e foi embora feliz e sorridente.
Luang Pó jamais sequer olhou para o carro. Quando ele descia de sua
cabana, ele o fazia por outra saída. Quando tinha que ir à cidade,
utilizava o carro de alguma pessoa. Ele jamais sequer olhou para ver
qual carro era, que cor tinha. Após sete dias Luang Pó chamou uma pessoa
e disse: `Vá e diga àquele homem que estou devolvendo o carro. Ele veio
oferecer e eu já recebi. Ele já fez o mérito dele, mas agora vou
devolver porque não é algo apropriado para um monge possuir.''

Certa ocasião, quando Luang Pó tinha que ir até a filial de Wat Tam Séng
Pet, havia uma fila de leigos, todos donos de carros de luxo, competindo
pela honra de ser aquele que ia lhe dar carona. Luang Pó olhou ao redor
por um instante, apontou para um carro velho que estava estacionado e
disse: `Ah! Eu vou neste aqui!' O dono do carro velho mal podia conter
sua alegria enquanto abria a porta para que Luang Pó entrasse. Naquele
dia a viagem a Wat Tam Séng Pet levou um pouco mais do que o normal,
porque todos aqueles carros de luxo tiveram que ir mais devagar para
conseguir permanecer atrás do carro velho que ia à frente, levando Ajahn
Chah.

Em outra ocasião, durante uma viagem com várias paradas, Luang Pó se
hospedou por três noites na casa de uma afluente família em Somut
Songkram. No terceiro dia disse ao leigo que viajava com ele que
preparasse as bagagens e chamasse um táxi para que continuassem a
viagem. O leigo ficou preocupado:

``O senhor não vai primeiro avisar o dono da casa?''

``Para quê? Quer informemos ou não, ainda temos que partir. Eu não
prometi vir visitá-los, apenas disse que estava vindo. Para que perder
tempo avisando e depois ter que aguentá-los tentando nos impedir de
partir?''

\enlargethispage{\baselineskip}

O leigo começou a preparar as bagagens e chamou o táxi. Um dos
empregados da casa deve ter notado a movimentação e telefonou para seu
patrão em Bangkok. Quando o táxi chegou, Luang Pó deu ordens para ir a
Aranya Pratet, mas após viajar apenas alguns quilômetros a dona da casa
veio dirigindo atrás deles, os ultrapassou e obstruiu a rua com seu
automóvel. Ela então saiu do carro, prestou reverência e chorou
reclamando: ``Luang Pó veio se hospedar em minha casa e eu ainda não
tive a oportunidade de falar com o senhor. E agora está nos abandonando
desse jeito!'' Não demorou muito e o marido também chegou e também
obstruiu a rua com seu carro. Mas, mesmo com ambos insistindo e
implorando, Luang Pó não dava ouvidos, ele apenas continuava dizendo:
``Deixe para outra ocasião.'', até que o marido também começou a chorar.
O marido perguntou ao motorista do táxi qual o valor da viagem e, ao
ouvir a resposta, a esposa colocou o valor em cima do capô do automóvel.
Luang Pó não se interessou, apenas disse ao motorista que seguisse em
frente. Ao final, não restou muito ao casal a não ser se ajoelhar no
chão com mãos em \emph{añjali} enquanto o táxi partia levando Ajahn
Chah.\footnote{É comum pessoas ricas convidarem monges famosos para se
  hospedar em suas casas apenas como uma forma de ostentação e status. A
  impressão que a história passa é de que após três dias o casal de
  ricos ainda não havia sequer tido interesse em ir dizer olá a Ajahn
  Chah, imaginando que ele permaneceria com eles o tempo que lhes fosse
  conveniente. Mas, tendo descansado, Luang Pó simplesmente continuou
  sua viagem, sem se deixar intimidar pela forma agressiva com que
  tentaram forçá-lo a permanecer.}

Luang Pó era sempre muito cuidadoso em não deixar que o relacionamento
com a comunidade leiga fugisse do controle. Ele nunca dava oportunidade
para ninguém se tornar muito íntimo, o que poderia causar inveja nas
demais pessoas. Isso era especialmente verdade com relação a discípulas
leigas. Ele também tentava evitar muito contato com pessoas ricas e
poderosas na sociedade para evitar criar um laço de dependência que no
futuro poderia colocá-lo na situação de ser obrigado ou a ceder aos
desejos deles ou a ofendê-los. Ainda assim ele se esforçava em tratar
todos com a mesma bondade, tomando o Dhamma-Vinaya como ferramenta para
proteger sua liberdade, o que fazia com que muitos dissessem que
``ninguém era dono de Luang Pó Chah''. Ele ensinava seus discípulos a
agirem da mesma forma; ele os alertava sobre não ter moderação ao
aceitar as oferendas feitas pelos leigos:

``Não ajam dessa maneira! Apesar de eles terem oferecido
\emph{pavāranā}, vocês ainda têm que ter consideração por eles. Eles têm
uma família para sustentar, é difícil ganhar a vida. Vocês, veneráveis,
nunca tiveram a experiência de ser chefe de uma família; como podem
saber se aquelas pessoas não estão passando por dificuldades? Mesmo que
eles tenham oferecido \emph{pavāranā}, temos que ter sensibilidade. Não
é só eles oferecerem \emph{pavāranā} e nós podemos então pedir qualquer
coisa que nos der na telha -- de repente, o que não era necessário se
torna necessário! Não ter consideração é agir movido por cobiça, é ser
desleixado, é uma calamidade.

Aonde quer que forem, sejam como um boi selvagem, não como um boi
domesticado. O boi selvagem é livre, ninguém o prende pelo
nariz,\footnote{Na Tailândia é comum o uso de argolas no nariz do gado
  para controlá-lo.} mas o boi domesticado fica amarrado ao poste.
Aonde quer que forem, não deixem os leigos os servirem até que vocês
fiquem presos a eles; não deixem que eles os prendam pelo nariz como um
boi. Sejam livres como o boi selvagem e assim poderão ir e vir quando
quiserem.''

Outro exemplo foi na ocasião em que dois discípulos saíram em
peregrinação pela região sul do país. A certa altura um grupo de leigos
ficou sabendo da presença de monges discípulos de Ajahn Chah na região e
foram até lá prestar reverência. Eles ofereceram \emph{pavāranā} aos
monges para que pedissem qualquer coisa que quisessem. Os monges então
pediram duas passagens de avião para retornar a Wat Nong Pah Pong. Na
época, mais do que hoje, viajar de avião era um luxo impensável para a
maioria das pessoas e teria sido mais apropriado se os monges tivessem
pedido uma passagem de ônibus ou trem. De qualquer forma, os leigos,
ainda que constrangidos, compraram as passagens mas também mandaram
notícias a Ajahn Chah sobre o ocorrido. Quando os dois monges enfim
chegaram a Wat Nong Pah Pong, Luang Pó os estava esperando com uma longa
e apimentada bronca para lhes dar como forma de boas-vindas.

No que diz respeito ao ritmo de prática no monastério, apesar de
continuar intensivo, Luang Pó começou a dar mais tempo para que os
monges relaxassem. Um dos recursos que ele utilizava para isso era criar
oportunidades para que os monges pudessem conversar com ele sobre Dhamma
num ambiente mais informal na área aberta na parte de baixo de sua
cabana. Nessas ocasiões, ele às vezes distribuía cigarros (na época
ainda era permitido fumar no monastério). Ele explicou: ``Mesmo uma
grande represa precisa de uma válvula de escape para evitar que
arrebente. Prática de meditação é a mesma coisa: você tem que relaxar e
aliviar a pressão, mas isso deve ser feito de uma maneira apropriada a
um monge.''

As ocasiões informais em que sentavam e escutavam o Dhamma de Luang Pó
eram como uma reunião de família, era um ambiente animado em que os
monges se sentiam alegres e se deleitavam enormemente no Dhamma. Lá
podiam presenciar de forma vívida toda a sagacidade da sabedoria de
Luang Pó, além de seu antigo bom humor, que agora voltava a fazer parte
de sua personalidade diária. Apesar de ser respeitado e temido por seus
discípulos, ele era ao mesmo tempo o centro da afeição de todos eles --
todos queriam estar perto dele e ouvir suas palavras, absorver sua
sabedoria. Alguns se beneficiavam mais e outros menos, de acordo com a
capacidade de cada um, mas todos levavam consigo algo de bom daqueles
encontros.

Alguns discípulos comentam sobre como Luang Pó era capaz de mudar de
personalidade quando ensinava o Dhamma, tal qual um ator escolhe uma
máscara diferente para adequar-se ao papel que representa. Além disso,
ele era extremamente hábil em escolher a personalidade correta para cada
ocasião, dependendo das pessoas com as quais estava interagindo. Por
essa razão, pessoas diferentes tinham percepções diferentes de quem e
como ele era, tudo dependendo da ``máscara'' que ele estivesse usando
quando se encontraram com ele. Por exemplo, aqueles que estavam sofrendo
e vinham visitá-lo ficavam impressionados com sua bondade e com o
conforto que suas palavras ofereciam. Já aqueles que eram orgulhosos e
arrogantes, ou que vinham desafiá-lo e testá-lo, eram recebidos com um
ensinamento forte e áspero que logo os traziam de volta à razão. Os
discípulos tailandeses em geral o viam como uma pessoa séria e rigorosa;
já os discípulos ocidentais o viam como uma pessoa relaxada e sorridente
-- era difícil dizer com certeza qual era sua real personalidade.

Luang Pó também teve que se adaptar à fama e as desvantagens que ela
traz. Quando seu renome começou a se espalhar, pessoas de toda parte
passaram a vir procurá-lo em Wat Nong Pah Pong. Entre essas pessoas era
comum encontrar aqueles interessados em forças ocultas e poderes
paranormais que contavam relatos dos milagres realizados por Ajahn Chah,
da mesma forma como comumente faziam em relação a outros mestres
tailandeses. Na maioria dos casos essas histórias vinham de gente com
imaginação fértil ou eram simplesmente inventadas do nada (ou as duas
coisas ao mesmo tempo\ldots{}).

No que diz respeito a Ajahn Chah, ele não dava importância alguma a esse
tipo de coisa e, assim, não falava sobre tais assuntos -- mas também não
negava sua existência. Se alguém o perguntasse sobre esses temas, ele ou
encerrava a conversa ou mudava o assunto para a prática do Dhamma e o
caminho para transcender \emph{dukkha}. Uma vez lhe perguntaram:

``Dizem que o Luang Pó é um \emph{arahant}. É verdade que você é capaz
de levitar no ar?''

``Levitar no ar não é importante; até mesmo besouros `rola-bosta'
conseguem voar!''

Em outra ocasião, quando um professor de escola perguntou se eram
verdadeiras as histórias de \emph{arahants} flutuando e caminhando no ar
que ele havia encontrado nas escrituras budistas, Ajahn Cahah respondeu:

``Sr. professor, sua pergunta está longe demais de si mesmo. Seria
melhor se você perguntasse sobre assuntos mais baixos -- da mesma
estatura que nós aqui\ldots{}''

Ainda assim, algumas histórias relacionadas a Ajahn Chah parecem
verídicas, especialmente aquelas que contam sobre sua capacidade de ler
os pensamentos das pessoas, das quais existem muitos exemplos como este:

Uma vez, um monge saiu em \emph{pindapāta} e, enquanto andava pelo
vilarejo, pensava consigo mesmo: ``Hoje estou com muita fome! Vou comer
muito, vou ter que comer uma bola de arroz do tamanho da minha cabeça
para poder ficar satisfeito!'' Quando voltou ao monastério, enquanto
caminhava pelo portão, Ajahn Chah o surpreendeu e perguntou ``Você está
mesmo com tanta fome? A ponto de ter que fazer uma bola de arroz do
tamanho da sua cabeça?!'' O monge não sabia o que dizer, então apenas
continuou caminhando em silêncio, envergonhado do fato de que Ajahn Chah
estava ciente do que ele pensava.

Outra história semelhante é a do monge que havia escondido um ovo dentro
de sua tigela. Normalmente toda a comida obtida como esmola era dividida
entre todos e o sistema utilizado era oferecer primeiro a comida aos
monges mais velhos e então para os demais, de acordo com o tempo de vida
monástica de cada um. Para esse propósito, quando voltavam de
\emph{pindapāta}, todos os monges esvaziavam o conteúdo de suas tigelas
em bacias, mantendo consigo somente a arroz puro. Naquele dia, um jovem
monge havia recebido um ovo cozido como esmola e provavelmente não quis
reparti-lo com os demais, então resolveu escondê-lo dentro de uma bola
de arroz grudento.\footnote{Khao Niao: uma variedade de arroz que quando
  cozida ganha uma consistência `al-dente' e grudenta. É comido pegando
  pequenas porções com as mãos.} Curiosamente, justamente naquele dia
Luang Pó resolveu verificar o conteúdo de todas as tigelas antes da
refeição. Ele caminhou pelo refeitório abrindo a tampa de cada tigela
para ver o que havia dentro e, quando chegou à tigela daquele monge,
tirou a bola de arroz grudento de dentro da tigela e a quebrou ao meio,
expondo o ovo cozido. Ele então perguntou em voz alta: ``De quem é esse
arroz!? Ele acaba de botar um ovo!!!''

Em outra ocasião, um discípulo leigo próximo a Ajahn Chah que já havia
testemunhado esse tipo de acontecimento várias vezes perguntou a ele
como era possível fazer tal coisa. Ajahn Chah apenas respondeu:

``Doutor, isso tem a ver com \emph{samādhi}, não é nada muito profundo,
mas, ainda assim, não é bom conversar a respeito.''

\enlargethispage{\baselineskip}

Luang Pó não falava de forma aberta sobre que estágio de iluminação ou
capacidades especiais ele ou seus discípulos haviam realizado,
preferindo sempre apontar a pessoa que perguntava de volta a si mesma e
à sua própria realidade. Quando falava sobre si mesmo, sempre humilde,
ele invariavelmente focava a conversa em seus defeitos e nas
dificuldades que enfrentou, o que fazia com que aqueles que ouviam
sentissem ânimo ao saber que mesmo um grande mestre como ele teve que
enfrentar tais desafios, e inspiração por saber que ele conseguiu
vencê-los e, portanto, quem sabe, talvez conseguissem fazer o mesmo.

Ainda assim, com o passar do tempo, o entendimento geral entre as
pessoas passou a ser de que ele havia alcançado o estágio mais elevado
de iluminação espiritual e de que essa seria sua última vida. Em grande
parte esse consenso foi se manifestando em seus discípulos apenas por
causa da fé que tinham nele, mas também por terem pouco a pouco reunido
informações que transpareciam durante conversas em que ele explicava
aspectos mais sutis do Dhamma ou relatava experiências em sua prática.
Além disso, outros mestres da tradição da floresta eram bastante
explícitos em declarar que Ajahn Chah era de fato um \emph{arahant}. De
qualquer forma, como o próprio Ajahn Chah costumava dizer, tudo isso é
muito incerto. A única coisa que sabemos com certeza é que ainda temos
caminho pela frente, mas ganhamos certo alento em ouvir que mesmo hoje
ainda há aqueles que conseguem realizar o fruto mais elevado do caminho
e libertar-se deste ciclo de nascimento e morte, \emph{samsāra}.

